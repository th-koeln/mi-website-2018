\hypertarget{einfuxfchrungpathlabelmi-2017modulbeschreibungen-bachelor01-einfuehrung}{%
\chapter{Einführung\label{/mi-2017/modulbeschreibungen-bachelor/01-einfuehrung}}\label{einfuxfchrungpathlabelmi-2017modulbeschreibungen-bachelor01-einfuehrung}}

Der Medieninformatik Bachelor ist ein berufsqualifizierender,
grundständiger Studiengang. Die Regelstudienzeit des
anwendungsorientierten Informatikstudiengangs beträgt 6 Semester. Die
Einschreibung ist ausschließlich zum Wintersemester möglich.

Der Studiengang besteht im Kern aus zwei parallelen Strängen: einem
Informatik und einem Medien-Strang. Die verschiedenen Module lassen sich
mehr oder weniger gut diesen beiden Strängen zuordnen. Darüber hinaus
gibt es eine Reihe von Modulen, die der Querschnittsqualifikation
dienen, z.B. die Studierenden bei der Durchführung von Projekten oder im
Umgang mit betriebswirtschaftlichen Frage- und Problemstellungen
unterstützen.

Das Studium verfolgt drei grundlegende Ziele:

\begin{itemize}
\tightlist
\item
  Aufbau von Kommunikations- und Methodenkompetenz
\item
  Vermittlung eines umfassenden Technologieverständnis
\item
  kennenlernen von Geschäftsprozessen und Kernaktivitäten der
  Medienwirtschaft
\end{itemize}

Der zeitliche Ablauf des Studiums teilt sich in drei Abschnitte:
Grundlagen, Vertiefung und Spezialisierung. In den ersten beiden
Semestern überwiegen die Module aus dem Informatik Strang. Hier werden
die mathematischen, theoretischen und technischen Grundlagen mit
Lehrveranstaltungen wie Mathematik, Algorithmen und Programmierung,
Theoretische Informatik und Einführung in Betriebssysteme und
Rechnerarchitekturen vermittelt. Mit fortschreitender Fachsemesterzahl,
nehmen die Module aus dem Medienstrang zunehmend Raum ein.

Im vierten Semester kann im großen Vertiefungsmodul eine von drei
Vertiefungsrichtungen gewählt werden. Zur Auswahl stehen: Visual
Computing, Social Computing und Web-Development. Im Vertiefungsmodul ist
ein Projektanteil von etwa fünf Creditpoints vorgesehen. Das
Entwicklungsprojekt im fünften Semester kann entsprechend der fachlichen
Neigung der jeweiligen Studierenden ausgestaltet werden.

Das vierte Semester eignet sich aus verschiedenen Gründen gut für ein
Auslandssemester. Die Studierenden verfügen über ausreichende
Qualifikationen und Projekterfahrungen, um in verschiedenen Kontexten
handlungsfähig zu sein. Sie stehen aber noch vor dem
Spezialisierungsteil des Studiums und verfügen damit idealerweise über
die fachliche, mentale und organisatorische Offenheit für ein
Austauschsemester. Wegen der wenigen Module im vierten Semester und vor
allem wegen des großen Vertiefungsmoduls können im Ausland erworbene
Qualifikationen sehr flexibel anerkannt werden. Die Anerkennung erfolgt
auf Basis des ``Übereinkommen über die Anerkennung von Qualifikationen
im Hochschulbereich in der europäischen Region''.

Den Abschluss des Studiums bildet die Bachelorarbeit. In dieser
bearbeiten die Studierenden selbstständig eine praxisorientierte Aufgabe
aus einem gewünschten Fachgebiet. Diese können sie in Kooperation mit
einem Unternehmen schreiben und somit Kontakte zur Wirtschaft und damit
potentiellen Arbeitgebern knüpfen.

\begin{figure}
\centering
\includegraphics[width=\columnwidth]{../anhaenge/bilder/studienverlaufsplan-mi-bachelor.png}
\caption{Studienverlaufsplan Medieninformatik Bachelor}
\end{figure}

Weitere Informationen zum Bachelor Studiengang Medieninformatik finden
Sie unter
\url{https://www.medieninformatik.th-koeln.de/study/bachelor/}.

\hypertarget{audiovisuelles-medienprojekt-1pathlabelmi-2017modulbeschreibungen-bachelorba_avm}{%
\chapter{Audiovisuelles Medienprojekt
1\label{/mi-2017/modulbeschreibungen-bachelor/BA_AVM}}\label{audiovisuelles-medienprojekt-1pathlabelmi-2017modulbeschreibungen-bachelorba_avm}}

\begin{modulHead}
\textbf{Modulverantwortlich}: Prof.~Hans
Kornacher
\end{modulHead}
\begin{modulHead}
\textbf{Studiensemester}:
3
\end{modulHead}
\begin{modulHead}
\textbf{Sprache}:
deutsch
\end{modulHead}
\begin{modulHead}
\textbf{Kreditpunkte}:
5
\end{modulHead}
\begin{modulHead}
\textbf{Typ}:
Pflichtmodul
\end{modulHead}
\begin{modulHead}
\textbf{Prüfungsleistung}:
Projektarbeit
\end{modulHead}


\hypertarget{kurzbeschreibungpathlabelmi-2017modulbeschreibungen-bachelorba_avm}{%
\section*{Kurzbeschreibung\label{/mi-2017/modulbeschreibungen-bachelor/BA_AVM}}\label{kurzbeschreibungpathlabelmi-2017modulbeschreibungen-bachelorba_avm}}

Im Mittelpunkt dieses Moduls steht die digitale audiovisuelle
Medienproduktion in den Formaten Porträt- und Dokumentarfilm.

\hypertarget{lehrformswspathlabelmi-2017modulbeschreibungen-bachelorba_avm}{%
\section*{Lehrform/SWS\label{/mi-2017/modulbeschreibungen-bachelor/BA_AVM}}\label{lehrformswspathlabelmi-2017modulbeschreibungen-bachelorba_avm}}

4 SWS: Vorlesung 2 SWS; Projekt 2 SWS

\hypertarget{arbeitsaufwandpathlabelmi-2017modulbeschreibungen-bachelorba_avm}{%
\section*{Arbeitsaufwand\label{/mi-2017/modulbeschreibungen-bachelor/BA_AVM}}\label{arbeitsaufwandpathlabelmi-2017modulbeschreibungen-bachelorba_avm}}

Gesamtaufwand 150h, davon

\begin{itemize}
\tightlist
\item
  36h Vorlesung
\item
  36h Projektarbeit
\item
  78h Selbststudium
\end{itemize}

\hypertarget{angestrebte-lernergebnissepathlabelmi-2017modulbeschreibungen-bachelorba_avm}{%
\section*{Angestrebte
Lernergebnisse\label{/mi-2017/modulbeschreibungen-bachelor/BA_AVM}}\label{angestrebte-lernergebnissepathlabelmi-2017modulbeschreibungen-bachelorba_avm}}

\begin{itemize}
\tightlist
\item
  Die Studierenden kennen die grundlegenden Erzählformen und Formate
  audiovisueller Medien und können auf der Basis klassischer
  Erzählmuster eigene audiovisuelle Erzählformen entwickeln.
\item
  Die Studierenden können die in der audiovisuellen Produktion
  auftretende Problemstellungen selbstständig lösen und die verwendeten
  Medientechnologien, wie Videokamera, Tonaufnahmegeräte und
  Schnittsysteme technisch richtig und gestalterisch aussagekräftig
  einzusetzen.
\item
  Sie sind befähigt zur Analyse, Diskussion und zur kritischen
  Betrachtung audiovisueller Medieninhalte.
\item
  Die Studierenden können in den unterschiedlichsten Berufsfeldern
  digitaler audiovisueller Medien die Entwicklung und den Einsatz
  audiovisuellen Content beraten, planen, selbst durchführen und
  verantworten.
\end{itemize}

\hypertarget{inhaltpathlabelmi-2017modulbeschreibungen-bachelorba_avm}{%
\section*{Inhalt\label{/mi-2017/modulbeschreibungen-bachelor/BA_AVM}}\label{inhaltpathlabelmi-2017modulbeschreibungen-bachelorba_avm}}

Im Mittelpunkt dieses Moduls steht die digitale audiovisuelle
Medienproduktion.

Die praktische Umsetzung des Vorlesungsstoffes, die Kommunikation und
Zusammenarbeit im Team und die Präsentation von eigenen Ideen und
Projekten sind die Lerninhalte des Moduls „Audiovisuelles
Medienprojekt``. Neben der Fachkompetenz und Methodenkompetenz stehen in
diesem Modul gerade die sogenannten Softskills Teamfähigkeit und
Kommunikationsfähigkeit im Focus.

Die Projektarbeit gliedert sich dabei in die selbstständige Entwicklung,
Ausarbeitung und Präsentation eines Filmthemas, in die praktische
Umsetzung in einem Filmprojekt und in die Nachbearbeitung und Montage in
einer dramaturgischen Erzählform.

Begleitend zu der Produktion werden folgende fachspezifischen Inhalte
thematisiert und in Übungsaufgaben vertieft:

\begin{itemize}
\tightlist
\item
  Video- und Audioaufnahmetechnik
\item
  Filmsprache
\item
  Lichtsetzung
\item
  Tonaufnahme
\item
  Dokumentarfilm und Interview
\item
  Dramaturgie
\item
  Inszenierung
\item
  Schnitt und Montage
\end{itemize}

\hypertarget{medienformenpathlabelmi-2017modulbeschreibungen-bachelorba_avm}{%
\section*{Medienformen\label{/mi-2017/modulbeschreibungen-bachelor/BA_AVM}}\label{medienformenpathlabelmi-2017modulbeschreibungen-bachelorba_avm}}

\begin{itemize}
\tightlist
\item
  Vorlesungen mit Folienpräsentationen
\item
  Workshops zu Anwendungsprogrammen
\item
  Beispiele aus verschiedenen Medien: Filmbeispiele, Webvideos
\item
  Audiovisuelle Aufnahme- und Wiedergabegeräte
\item
  Lehrfilme und Video-Tutorials
\end{itemize}

\hypertarget{literaturpathlabelmi-2017modulbeschreibungen-bachelorba_avm}{%
\section*{Literatur\label{/mi-2017/modulbeschreibungen-bachelor/BA_AVM}}\label{literaturpathlabelmi-2017modulbeschreibungen-bachelorba_avm}}

\begin{itemize}
\tightlist
\item
  James Monaco, Film verstehen, Rowolth Taschenbuch Verlag Hamburg,
  1980, ISBN 3-499-162717
\item
  Syd Field, Drehbuchschreiben für Film und Fernsehen, München 2003,
  ISBN 354836473X
\item
  Steven D. Katz, Die Richtige Einstellung, Zweitausendeins, Frankfurt
  a.M.1998,ISBN 3-86150-229-1
\item
  David Lewis Yewdall, Practical Art of Motion Picture Sound, Focal
  Press, USA 2003, ISBN 0-240-80525-9
\item
  Hans Kornacher \& Manfred Stross, Dokumentarisches Videofilmen,
  Augustus Verlag, Augsburg, 1992, ISBN 3-8043-5474-2
\item
  Hans Beller Hg., Handbuch der Filmmontage, München: TR-Verlagsunion,
  1993, ISBN 3-8058-2357-6
\item
  Karel Reisz, Gavin Millar, Geschichte und Technik der Filmmontage,
  München: Filmlandpresse, 1988, ISBN 3-88690-071-1
\item
  Chris Vogler, Die Reise des Drehbuchschreibens, Verlag Zweitausendeins
\end{itemize}

\hypertarget{algorithmen-und-programmierung-1pathlabelmi-2017modulbeschreibungen-bachelorba_algorithmenundprogrammierung1}{%
\chapter{Algorithmen und Programmierung
1\label{/mi-2017/modulbeschreibungen-bachelor/BA_AlgorithmenundProgrammierung1}}\label{algorithmen-und-programmierung-1pathlabelmi-2017modulbeschreibungen-bachelorba_algorithmenundprogrammierung1}}

\begin{modulHead}
\textbf{Modulverantwortlich}: Prof.~Dr.~Frank
Victor
\end{modulHead}
\begin{modulHead}
\textbf{Studiensemester}:
1
\end{modulHead}
\begin{modulHead}
\textbf{Sprache}:
deutsch
\end{modulHead}
\begin{modulHead}
\textbf{Kreditpunkte}:
8
\end{modulHead}
\begin{modulHead}
\textbf{Typ}:
Pflichtmodul
\end{modulHead}
\begin{modulHead}
\textbf{Prüfungsleistung}:
Schriftliche Prüfung, sowie erfolgreiche Teilnahme am Praktikum als
Prüfungsvorleistung
\end{modulHead}


\hypertarget{lehrformswspathlabelmi-2017modulbeschreibungen-bachelorba_algorithmenundprogrammierung1}{%
\section*{Lehrform/SWS\label{/mi-2017/modulbeschreibungen-bachelor/BA_AlgorithmenundProgrammierung1}}\label{lehrformswspathlabelmi-2017modulbeschreibungen-bachelorba_algorithmenundprogrammierung1}}

6 SWS: Vorlesung 3 SWS; Übung 1 SWS; Praktikum 2 SWS

\hypertarget{arbeitsaufwandpathlabelmi-2017modulbeschreibungen-bachelorba_algorithmenundprogrammierung1}{%
\section*{Arbeitsaufwand\label{/mi-2017/modulbeschreibungen-bachelor/BA_AlgorithmenundProgrammierung1}}\label{arbeitsaufwandpathlabelmi-2017modulbeschreibungen-bachelorba_algorithmenundprogrammierung1}}

Gesamtaufwand 240h, davon

\begin{itemize}
\tightlist
\item
  54h Vorlesung
\item
  36h Praktikum
\item
  18h Übung
\item
  132h Selbststudium
\end{itemize}

\hypertarget{angestrebte-lernergebnissepathlabelmi-2017modulbeschreibungen-bachelorba_algorithmenundprogrammierung1}{%
\section*{Angestrebte
Lernergebnisse\label{/mi-2017/modulbeschreibungen-bachelor/BA_AlgorithmenundProgrammierung1}}\label{angestrebte-lernergebnissepathlabelmi-2017modulbeschreibungen-bachelorba_algorithmenundprogrammierung1}}

Die Studierenden sollen

\begin{itemize}
\tightlist
\item
  formale und algorithmische Kompetenzen im Bereich der
  Software-Entwicklung erlangen. Hierzu gehören insbesondere die
  Prinzipien der Objektorientierung und die der prozeduralen
  Programmierung.
\item
  die Kompetenz erlangen, strukturierte und unstrukturierte
  Problemstellungen zu analysieren, Lösungen modellbasiert zu entwickeln
  sowie prozedural und objektorientiert umzusetzen.
\item
  Systementwürfe evaluieren und bewerten können, insbesondere sollen sie
  die Arbeitsweise, die Randbedingungen und den Komplexitätsgrad von
  einfachen Algorithmen verstehen.
\item
  die Fähigkeit erlernen, algorithmische Entwurfsmuster zu erkennen und
  anzuwenden
\end{itemize}

\hypertarget{inhaltpathlabelmi-2017modulbeschreibungen-bachelorba_algorithmenundprogrammierung1}{%
\section*{Inhalt\label{/mi-2017/modulbeschreibungen-bachelor/BA_AlgorithmenundProgrammierung1}}\label{inhaltpathlabelmi-2017modulbeschreibungen-bachelorba_algorithmenundprogrammierung1}}

\begin{itemize}
\tightlist
\item
  Prozedurale Programmierung am Beispiel von C.
\item
  Objektorientierte Programmierung am Beispiel von Java.
\item
  Kontroll- und Datenstrukturen.
\item
  Modularisierungskonzepte.
\item
  Typkonzepte.
\item
  Grundmuster der objektorientierten Programmierung.
\item
  Elementare Algorithmen und Aufwandsschätzung.
\item
  Entwicklungsumgebungen.
\end{itemize}

\hypertarget{medienformenpathlabelmi-2017modulbeschreibungen-bachelorba_algorithmenundprogrammierung1}{%
\section*{Medienformen\label{/mi-2017/modulbeschreibungen-bachelor/BA_AlgorithmenundProgrammierung1}}\label{medienformenpathlabelmi-2017modulbeschreibungen-bachelorba_algorithmenundprogrammierung1}}

\begin{itemize}
\tightlist
\item
  Beamer-gestützte Vorlesungen (Folien in elektronischer Form)
\item
  Praktikum an Rechnern des Labors
\end{itemize}

\hypertarget{literaturpathlabelmi-2017modulbeschreibungen-bachelorba_algorithmenundprogrammierung1}{%
\section*{Literatur\label{/mi-2017/modulbeschreibungen-bachelor/BA_AlgorithmenundProgrammierung1}}\label{literaturpathlabelmi-2017modulbeschreibungen-bachelorba_algorithmenundprogrammierung1}}

\begin{itemize}
\tightlist
\item
  Vorlesungsunterlagen: Foliensammlung, ausformuliertes Skript,
  Beispiellösungen, Übungsklausuren mit Lösungen
\item
  Fachliteratur: Diverse C-Bücher, u.a.: Kernighan, B.W., Ritchie, D.M.:
  „Programmieren in C``
\item
  Diverse Java-Bücher, u.a.: Bishop, J.: „Java Lernen``
\item
  Sedgewick, R.: „Algorithmen in Java``
\end{itemize}

\hypertarget{algorithmen-und-programmierung-2pathlabelmi-2017modulbeschreibungen-bachelorba_algorithmenundprogrammierung2}{%
\chapter{Algorithmen und Programmierung
2\label{/mi-2017/modulbeschreibungen-bachelor/BA_AlgorithmenundProgrammierung2}}\label{algorithmen-und-programmierung-2pathlabelmi-2017modulbeschreibungen-bachelorba_algorithmenundprogrammierung2}}

\begin{modulHead}
\textbf{Modulverantwortlich}: Prof.~Dr.~Christian
Kohls
\end{modulHead}
\begin{modulHead}
\textbf{Studiensemester}:
2
\end{modulHead}
\begin{modulHead}
\textbf{Sprache}:
deutsch
\end{modulHead}
\begin{modulHead}
\textbf{Kreditpunkte}:
7
\end{modulHead}
\begin{modulHead}
\textbf{Typ}:
Pflichtmodul
\end{modulHead}
\begin{modulHead}
\textbf{Prüfungsleistung}:
Schriftliche Prüfung, sowie erfolgreiche Teilnahme am Praktikum als
Prüfungsvorleistung
\end{modulHead}


\hypertarget{lehrformswspathlabelmi-2017modulbeschreibungen-bachelorba_algorithmenundprogrammierung2}{%
\section*{Lehrform/SWS\label{/mi-2017/modulbeschreibungen-bachelor/BA_AlgorithmenundProgrammierung2}}\label{lehrformswspathlabelmi-2017modulbeschreibungen-bachelorba_algorithmenundprogrammierung2}}

6 SWS: Vorlesung 3 SWS; Übung 1 SWS; Praktikum 2 SWS

\hypertarget{arbeitsaufwandpathlabelmi-2017modulbeschreibungen-bachelorba_algorithmenundprogrammierung2}{%
\section*{Arbeitsaufwand\label{/mi-2017/modulbeschreibungen-bachelor/BA_AlgorithmenundProgrammierung2}}\label{arbeitsaufwandpathlabelmi-2017modulbeschreibungen-bachelorba_algorithmenundprogrammierung2}}

Gesamtaufwand 210h, davon

\begin{itemize}
\tightlist
\item
  54h Vorlesung
\item
  36h Praktikum
\item
  18h Übung
\item
  102h Selbststudium
\end{itemize}

\hypertarget{angestrebte-lernergebnissepathlabelmi-2017modulbeschreibungen-bachelorba_algorithmenundprogrammierung2}{%
\section*{Angestrebte
Lernergebnisse\label{/mi-2017/modulbeschreibungen-bachelor/BA_AlgorithmenundProgrammierung2}}\label{angestrebte-lernergebnissepathlabelmi-2017modulbeschreibungen-bachelorba_algorithmenundprogrammierung2}}

Die Studierende sollen Objektorientierung, die Prinzipien der
Algorithmenentwicklung und grundlegende Algorithmen verstehen und die
Grundstrukturen der Java-Bibliothek anwenden können.

\hypertarget{inhaltpathlabelmi-2017modulbeschreibungen-bachelorba_algorithmenundprogrammierung2}{%
\section*{Inhalt\label{/mi-2017/modulbeschreibungen-bachelor/BA_AlgorithmenundProgrammierung2}}\label{inhaltpathlabelmi-2017modulbeschreibungen-bachelorba_algorithmenundprogrammierung2}}

\begin{itemize}
\tightlist
\item
  Basisalgorithmen: Suchen u. Sortieren
\item
  Datenstrukturen
\item
  Dictionaries
\item
  Methodik des objektorientierten Programmierens
\end{itemize}

\hypertarget{medienformenpathlabelmi-2017modulbeschreibungen-bachelorba_algorithmenundprogrammierung2}{%
\section*{Medienformen\label{/mi-2017/modulbeschreibungen-bachelor/BA_AlgorithmenundProgrammierung2}}\label{medienformenpathlabelmi-2017modulbeschreibungen-bachelorba_algorithmenundprogrammierung2}}

\begin{itemize}
\tightlist
\item
  Beamer-gestützte Vorlesungen (Folien in elektronischer Form)
\item
  Praktikum an Rechnern des Labors
\end{itemize}

\hypertarget{literaturpathlabelmi-2017modulbeschreibungen-bachelorba_algorithmenundprogrammierung2}{%
\section*{Literatur\label{/mi-2017/modulbeschreibungen-bachelor/BA_AlgorithmenundProgrammierung2}}\label{literaturpathlabelmi-2017modulbeschreibungen-bachelorba_algorithmenundprogrammierung2}}

\begin{itemize}
\tightlist
\item
  Vorlesungsunterlagen: Foliensammlung, ausformuliertes Skript,
  Beispiellösungen
\item
  Fachliteratur: Bishop, J.: „Java Lernen``
\item
  Sedgewick, R.: „Algorithmen in Java``
\item
  Barnes, J., Kölling, M.: „Java Lernen mit BlueJ``, Verweise auf
  Onlinedokumente
\end{itemize}

\hypertarget{bwl-i---grundlagenpathlabelmi-2017modulbeschreibungen-bachelorba_bwl1}{%
\chapter{BWL I -
Grundlagen\label{/mi-2017/modulbeschreibungen-bachelor/BA_BWL1}}\label{bwl-i---grundlagenpathlabelmi-2017modulbeschreibungen-bachelorba_bwl1}}

\begin{modulHead}
\textbf{Modulverantwortlich}: Prof.~Dr.~Monika
Engelen
\end{modulHead}
\begin{modulHead}
\textbf{Studiensemester}:
5
\end{modulHead}
\begin{modulHead}
\textbf{Sprache}:
deutsch
\end{modulHead}
\begin{modulHead}
\textbf{Kreditpunkte}:
5
\end{modulHead}
\begin{modulHead}
\textbf{Typ}:
Pflichtmodul
\end{modulHead}
\begin{modulHead}
\textbf{Prüfungsleistung}:
Schriftliche Prüfung via ILIAS eAssessment
\end{modulHead}


\hypertarget{lehrformswspathlabelmi-2017modulbeschreibungen-bachelorba_bwl1}{%
\section*{Lehrform/SWS\label{/mi-2017/modulbeschreibungen-bachelor/BA_BWL1}}\label{lehrformswspathlabelmi-2017modulbeschreibungen-bachelorba_bwl1}}

4 SWS: Vorlesung 2 SWS; Übung 2 SWS

\hypertarget{arbeitsaufwandpathlabelmi-2017modulbeschreibungen-bachelorba_bwl1}{%
\section*{Arbeitsaufwand\label{/mi-2017/modulbeschreibungen-bachelor/BA_BWL1}}\label{arbeitsaufwandpathlabelmi-2017modulbeschreibungen-bachelorba_bwl1}}

Gesamtaufwand 150h, davon

\begin{itemize}
\tightlist
\item
  30h Vorlesung
\item
  30h Übung
\item
  90h Selbststudium
\end{itemize}

\hypertarget{angestrebte-lernergebnissepathlabelmi-2017modulbeschreibungen-bachelorba_bwl1}{%
\section*{Angestrebte
Lernergebnisse\label{/mi-2017/modulbeschreibungen-bachelor/BA_BWL1}}\label{angestrebte-lernergebnissepathlabelmi-2017modulbeschreibungen-bachelorba_bwl1}}

Die Studierenden verstehen die wichtigsten Entscheidungsbereiche
wirtschaftlichen Handeln und können diese anwenden. Sie können
grundlegenden Entscheidungen im Rahmen einer Unternehmensgründung, die
Aufgaben der Unternehmensführung wie die Konzeption einer tragfähigen
Strategie, und die Aufgaben der Teilbereiche Produktion, Absatz und
Marketing sowie Investition und Finanzierung beschreiben und beurteilen.
Investitionsentscheidungen können die Studierenden informationsgestützt
treffen indem sie die Kalkulationsverfahren der Investitionsrechnung
anwenden und auswerten. Die Veranstaltung bereitet die Studierenden für
weitere BWL-Veranstaltungen ihres Studiums, sowie darauf, in ihrem
Berufsleben wirtschaftliche Konzepte im Unternehmenskontext anzuwenden,
vor.

\hypertarget{inhaltpathlabelmi-2017modulbeschreibungen-bachelorba_bwl1}{%
\section*{Inhalt\label{/mi-2017/modulbeschreibungen-bachelor/BA_BWL1}}\label{inhaltpathlabelmi-2017modulbeschreibungen-bachelorba_bwl1}}

• Grundlagen

• Unternehmensführung 1: Ziele, Planung und Entscheidung

• Unternehmensführung 2: Ausführung und Kontrolle

• Investition und Finanzierung

• Konstitutive Entscheidungen

• Produktion

• Absatz und Marketing

\hypertarget{literaturpathlabelmi-2017modulbeschreibungen-bachelorba_bwl1}{%
\section*{Literatur\label{/mi-2017/modulbeschreibungen-bachelor/BA_BWL1}}\label{literaturpathlabelmi-2017modulbeschreibungen-bachelorba_bwl1}}

\begin{itemize}
\tightlist
\item
  Wöhe (2016): Einführung in die Allgemeine Betriebswirtschaftslehre,
  26. Aufl.
\end{itemize}

\hypertarget{bachelorarbeitpathlabelmi-2017modulbeschreibungen-bachelorba_bachelorarbeit}{%
\chapter{Bachelorarbeit\label{/mi-2017/modulbeschreibungen-bachelor/BA_Bachelorarbeit}}\label{bachelorarbeitpathlabelmi-2017modulbeschreibungen-bachelorba_bachelorarbeit}}

\begin{modulHead}
\textbf{Modulverantwortlich}: alle Professor:innen
der Lehreinheit Informatik der
F10
\end{modulHead}
\begin{modulHead}
\textbf{Studiensemester}:
6
\end{modulHead}
\begin{modulHead}
\textbf{Sprache}:
deutsch
\end{modulHead}
\begin{modulHead}
\textbf{Kreditpunkte}:
12
\end{modulHead}
\begin{modulHead}
\textbf{Typ}:
Pflichtmodul
\end{modulHead}
\begin{modulHead}
\textbf{Prüfungsleistung}:
Schriftliche Ausarbeitung, ggf. Projektarbeit mit entsprechenden
Artefakten.
\end{modulHead}


\hypertarget{lehrformswspathlabelmi-2017modulbeschreibungen-bachelorba_bachelorarbeit}{%
\section*{Lehrform/SWS\label{/mi-2017/modulbeschreibungen-bachelor/BA_Bachelorarbeit}}\label{lehrformswspathlabelmi-2017modulbeschreibungen-bachelorba_bachelorarbeit}}

Angeleitetes, eigenverantwortliches Arbeiten

\hypertarget{arbeitsaufwandpathlabelmi-2017modulbeschreibungen-bachelorba_bachelorarbeit}{%
\section*{Arbeitsaufwand\label{/mi-2017/modulbeschreibungen-bachelor/BA_Bachelorarbeit}}\label{arbeitsaufwandpathlabelmi-2017modulbeschreibungen-bachelorba_bachelorarbeit}}

360 Stunden

\hypertarget{angestrebte-lernergebnissepathlabelmi-2017modulbeschreibungen-bachelorba_bachelorarbeit}{%
\section*{Angestrebte
Lernergebnisse\label{/mi-2017/modulbeschreibungen-bachelor/BA_Bachelorarbeit}}\label{angestrebte-lernergebnissepathlabelmi-2017modulbeschreibungen-bachelorba_bachelorarbeit}}

Die Bachelorarbeit soll zeigen, dass der Prüfling befähigt ist,
innerhalb einer vorgegebenen Frist eine praxisorientierte Aufgabe aus
seinem Fachgebiet sowohl in ihren fachlichen Einzelheiten als auch in
den fachübergreifenden Zusammenhängen nach wissenschaftlichen,
fachpraktischen und gestalterischen Methoden selbständig zu bearbeiten.
Die Bachelorarbeit ist in der Regel eine eigenständige Untersuchung mit
einer Aufgabenstellung aus der Medieninformatik und einer ausführlichen
Beschreibung und Erläuterung ihrer Lösung. In fachlich geeigneten Fällen
kann sie auch eine schriftliche Hausarbeit mit fachliterarischem Inhalt
sein.

\hypertarget{inhaltpathlabelmi-2017modulbeschreibungen-bachelorba_bachelorarbeit}{%
\section*{Inhalt\label{/mi-2017/modulbeschreibungen-bachelor/BA_Bachelorarbeit}}\label{inhaltpathlabelmi-2017modulbeschreibungen-bachelorba_bachelorarbeit}}

Selbstständiges wissenschaftliches, fachpraktisches und gestalterisches
Bearbeiten einer Aufgabenstellung.

\hypertarget{bachelor-kolloquiumpathlabelmi-2017modulbeschreibungen-bachelorba_bachelorkolloquium}{%
\chapter{Bachelor
Kolloquium\label{/mi-2017/modulbeschreibungen-bachelor/BA_Bachelorkolloquium}}\label{bachelor-kolloquiumpathlabelmi-2017modulbeschreibungen-bachelorba_bachelorkolloquium}}

\begin{modulHead}
\textbf{Modulverantwortlich}: alle Professor:innen
der Lehreinheit Informatik der
F10
\end{modulHead}
\begin{modulHead}
\textbf{Studiensemester}:
6
\end{modulHead}
\begin{modulHead}
\textbf{Sprache}:
deutsch
\end{modulHead}
\begin{modulHead}
\textbf{Kreditpunkte}:
3
\end{modulHead}
\begin{modulHead}
\textbf{Typ}:
Pflichtmodul
\end{modulHead}
\begin{modulHead}
\textbf{Prüfungsleistung}:
Mündliche Prüfung, Vortrag, Fachgespräch
\end{modulHead}


\hypertarget{lehrformswspathlabelmi-2017modulbeschreibungen-bachelorba_bachelorkolloquium}{%
\section*{Lehrform/SWS\label{/mi-2017/modulbeschreibungen-bachelor/BA_Bachelorkolloquium}}\label{lehrformswspathlabelmi-2017modulbeschreibungen-bachelorba_bachelorkolloquium}}

Angeleitetes, eigenverantwortliches Arbeiten

\hypertarget{arbeitsaufwandpathlabelmi-2017modulbeschreibungen-bachelorba_bachelorkolloquium}{%
\section*{Arbeitsaufwand\label{/mi-2017/modulbeschreibungen-bachelor/BA_Bachelorkolloquium}}\label{arbeitsaufwandpathlabelmi-2017modulbeschreibungen-bachelorba_bachelorkolloquium}}

90 Stunden

\hypertarget{angestrebte-lernergebnissepathlabelmi-2017modulbeschreibungen-bachelorba_bachelorkolloquium}{%
\section*{Angestrebte
Lernergebnisse\label{/mi-2017/modulbeschreibungen-bachelor/BA_Bachelorkolloquium}}\label{angestrebte-lernergebnissepathlabelmi-2017modulbeschreibungen-bachelorba_bachelorkolloquium}}

Das Kolloquium dient der Feststellung, ob der Prüfling befähigt ist, die
Ergebnisse der Bachelorarbeit, ihre fachlichen Grundlagen, ihre
fachübergreifenden Zusammenhänge und ihre außerfachlichen Bezüge
mündlich darzustellen und selbständig zu begründen und ihre Bedeutung
für die Praxis einzuschätzen. Dabei soll auch die Bearbeitung des Themas
der Bachelorarbeit mit dem Prüfling erörtert werden.

\hypertarget{inhaltpathlabelmi-2017modulbeschreibungen-bachelorba_bachelorkolloquium}{%
\section*{Inhalt\label{/mi-2017/modulbeschreibungen-bachelor/BA_Bachelorkolloquium}}\label{inhaltpathlabelmi-2017modulbeschreibungen-bachelorba_bachelorkolloquium}}

Vortrag über das Thema der Bachelorarbeit, Fachdiskussion und mündliche
Verteidigung der Arbeit im Kontext der Medieninformatik.

\hypertarget{datenbankenpathlabelmi-2017modulbeschreibungen-bachelorba_datenbanken1}{%
\chapter{Datenbanken\label{/mi-2017/modulbeschreibungen-bachelor/BA_Datenbanken1}}\label{datenbankenpathlabelmi-2017modulbeschreibungen-bachelorba_datenbanken1}}

\begin{modulHead}
\textbf{Modulverantwortlich}: Prof.~Dr.~Johann
Schaible
\end{modulHead}
\begin{modulHead}
\textbf{Studiensemester}:
3
\end{modulHead}
\begin{modulHead}
\textbf{Sprache}:
deutsch
\end{modulHead}
\begin{modulHead}
\textbf{Kreditpunkte}:
5
\end{modulHead}
\begin{modulHead}
\textbf{Typ}:
Pflichtmodul
\end{modulHead}
\begin{modulHead}
\textbf{Prüfungsleistung}:
Schriftliche Prüfung, sowie erfolgreiche Teilnahme am Praktikum als
Prüfungsvorleistung.
\end{modulHead}


\hypertarget{lehrformswspathlabelmi-2017modulbeschreibungen-bachelorba_datenbanken1}{%
\section*{Lehrform/SWS\label{/mi-2017/modulbeschreibungen-bachelor/BA_Datenbanken1}}\label{lehrformswspathlabelmi-2017modulbeschreibungen-bachelorba_datenbanken1}}

4 SWS: Vorlesung 2 SWS; Übung 1 SWS; Praktikum 1 SWS

\hypertarget{arbeitsaufwandpathlabelmi-2017modulbeschreibungen-bachelorba_datenbanken1}{%
\section*{Arbeitsaufwand\label{/mi-2017/modulbeschreibungen-bachelor/BA_Datenbanken1}}\label{arbeitsaufwandpathlabelmi-2017modulbeschreibungen-bachelorba_datenbanken1}}

Gesamtaufwand 150h, davon

\begin{itemize}
\tightlist
\item
  36h Vorlesung
\item
  18h Praktikum
\item
  18h Übung
\item
  78h Selbststudium
\end{itemize}

\hypertarget{angestrebte-lernergebnissepathlabelmi-2017modulbeschreibungen-bachelorba_datenbanken1}{%
\section*{Angestrebte
Lernergebnisse\label{/mi-2017/modulbeschreibungen-bachelor/BA_Datenbanken1}}\label{angestrebte-lernergebnissepathlabelmi-2017modulbeschreibungen-bachelorba_datenbanken1}}

Die Studierenden können

\begin{itemize}
\tightlist
\item
  ein einheitliches und konsistentes Begriffsgebäude bezüglich der
  Datenbankthematik verwenden,
\item
  Erkenntnisse im Rahmen der Modellierung und Implementierung von
  Datenbankschemata praktisch anwenden,
\item
  relationale Datenbankschemata konzipieren, implementieren und
  validieren, insbesondere in Bezug auf die relationale Algebra, die
  Normalisierung sowie funktionale Abhängigkeiten,
\item
  komplexere Datenbankanfragen, Datendefinitionen und Datenänderungen
  über SQL programmieren,~
\item
  Datenbankabfragen durch Verwendung von Indexen und
  SQL-Statement-Tuning effizient gestalten und
\item
  den Transaktionsbegriff sowie die Mehrbenutzersynchronisation
  anwenden.
\end{itemize}

\hypertarget{inhaltpathlabelmi-2017modulbeschreibungen-bachelorba_datenbanken1}{%
\section*{Inhalt\label{/mi-2017/modulbeschreibungen-bachelor/BA_Datenbanken1}}\label{inhaltpathlabelmi-2017modulbeschreibungen-bachelorba_datenbanken1}}

\begin{itemize}
\tightlist
\item
  Grundbegriffe und Architektur von Datenbanken
\item
  Ein Vorgehensmodell zur Erstellung eines Datenbanksystems
\item
  Datenmodellierung (Entity Relationship Modell) und Implementierung am
  Beispiel eines relationalen Datenbanksystems
\item
  Grundlagen des relationalen Modells
\item
  Relationale Algebra
\item
  Datenbankerstellung, -manipulation und -abfragen mit SQL
\item
  Funktionale Abhängigkeiten
\item
  Datenintegrität
\item
  Normalisierung
\item
  Datenbanksprache SQL: DDL, DML, DAL, Integritätsbedingungen und
  Constraints unter dem jeweils aktuellen SQL-Standard
\item
  Transaktionskonzepte und Mehrbenutzersynchronisation
\end{itemize}

\hypertarget{medienformenpathlabelmi-2017modulbeschreibungen-bachelorba_datenbanken1}{%
\section*{Medienformen\label{/mi-2017/modulbeschreibungen-bachelor/BA_Datenbanken1}}\label{medienformenpathlabelmi-2017modulbeschreibungen-bachelorba_datenbanken1}}

\begin{itemize}
\tightlist
\item
  Folien gestützer Vortrag
\item
  I.d.R. erarbeiten der Theorie anhand von überschaubaren
  Problemstellungen und deren in der Veranstaltung entwickelten Lösungen
\item
  Fragen der Studierenden beantworten - sehr erwünscht!
\item
  Ilias zur Bereitstellung aller Informationen (Aktuelles, Links,
  Folien, Praktikums-/Übungsaufgaben, wie auch Lösungen)
\item
  Self-Assessment mit den Database Trainern von EILD
\item
  DB-Wiki, das Online Lexikon für Datenbank-Themen
\end{itemize}

\hypertarget{literaturpathlabelmi-2017modulbeschreibungen-bachelorba_datenbanken1}{%
\section*{Literatur\label{/mi-2017/modulbeschreibungen-bachelor/BA_Datenbanken1}}\label{literaturpathlabelmi-2017modulbeschreibungen-bachelorba_datenbanken1}}

\begin{itemize}
\tightlist
\item
  Date, C.J.: ``E. F. Codd and Relational Theory'', Technics
  Publications LLC, 2021 (engl.)
\item
  Elmasri, R., Navathe, S.B.: ``Fundamentals of Database Systems''.
  Addison Wesley, 2016 (2009 auch auf deutsch)
\item
  Jens Dittrich, Uni Saarland, Datenbank-Vorlesung, Unterlagen:
  http://datenbankenlernen.de
\item
  mehr als 70 Videos: https://www.youtube.com/user/jensdit
\item
  Faeskorn-Woyke, H., Bertelsmeier, B., Riemer, P., Bauer, E.:
  „Datenbanksysteme: Theorie und Praxis mit Oracle und MySQL``, Pearson,
  2007 -- als pdf in ILIAS hochgeladen
\item
  Heuer, A., Saake, G., Sattler, K.-U., Grunert, H. \ldots: „
  Datenbanken Kompaktkurs``, MITP, 2020
\item
  Kemper, A., Eickler, A.: ``Datenbanksysteme -- Eine Einführung``. De
  Gruyter, 2015 mit Übungsbuch
\item
  Saake, G.; Sattler, K.-U.; Heuer, A.: „Datenbanken -- Konzepte und
  Sprachen``, mitp/bhv, 2018
\end{itemize}

\hypertarget{einfuxfchrung-in-betriebssysteme-und-rechnerarchitekturpathlabelmi-2017modulbeschreibungen-bachelorba_einfhrunginbetriebssystemeundrechnerarchitektur}{%
\chapter{Einführung in Betriebssysteme und
Rechnerarchitektur\label{/mi-2017/modulbeschreibungen-bachelor/BA_EinfhrunginBetriebssystemeundRechnerarchitektur}}\label{einfuxfchrung-in-betriebssysteme-und-rechnerarchitekturpathlabelmi-2017modulbeschreibungen-bachelorba_einfhrunginbetriebssystemeundrechnerarchitektur}}

\begin{modulHead}
\textbf{Modulverantwortlich}: Prof.~Dr.~Stefan
Karsch
\end{modulHead}
\begin{modulHead}
\textbf{Studiensemester}:
1
\end{modulHead}
\begin{modulHead}
\textbf{Sprache}:
deutsch
\end{modulHead}
\begin{modulHead}
\textbf{Kreditpunkte}:
5
\end{modulHead}
\begin{modulHead}
\textbf{Typ}:
Pflichtmodul
\end{modulHead}
\begin{modulHead}
\textbf{Prüfungsleistung}:
Schriftliche Prüfung
\end{modulHead}


\hypertarget{lehrformswspathlabelmi-2017modulbeschreibungen-bachelorba_einfhrunginbetriebssystemeundrechnerarchitektur}{%
\section*{Lehrform/SWS\label{/mi-2017/modulbeschreibungen-bachelor/BA_EinfhrunginBetriebssystemeundRechnerarchitektur}}\label{lehrformswspathlabelmi-2017modulbeschreibungen-bachelorba_einfhrunginbetriebssystemeundrechnerarchitektur}}

4 SWS: Vorlesung 2 SWS; Übung 2 SWS

\hypertarget{arbeitsaufwandpathlabelmi-2017modulbeschreibungen-bachelorba_einfhrunginbetriebssystemeundrechnerarchitektur}{%
\section*{Arbeitsaufwand\label{/mi-2017/modulbeschreibungen-bachelor/BA_EinfhrunginBetriebssystemeundRechnerarchitektur}}\label{arbeitsaufwandpathlabelmi-2017modulbeschreibungen-bachelorba_einfhrunginbetriebssystemeundrechnerarchitektur}}

Gesamtaufwand 150h, davon

\begin{itemize}
\tightlist
\item
  36h Vorlesung
\item
  36h Übung
\item
  78h Selbststudium
\end{itemize}

\hypertarget{angestrebte-lernergebnissepathlabelmi-2017modulbeschreibungen-bachelorba_einfhrunginbetriebssystemeundrechnerarchitektur}{%
\section*{Angestrebte
Lernergebnisse\label{/mi-2017/modulbeschreibungen-bachelor/BA_EinfhrunginBetriebssystemeundRechnerarchitektur}}\label{angestrebte-lernergebnissepathlabelmi-2017modulbeschreibungen-bachelorba_einfhrunginbetriebssystemeundrechnerarchitektur}}

Die Studierenden sollen die Basiskonzepte und Grundlagen der
Betriebssysteme und der Rechnerarchitektur kennen und verstehen, sowie
ein einheitliches konsistentes Begriffsgebäude zu, teilweise aus der
persönlichen Praxis bekannten, Sachverhalten der IT aufbauen und
anwenden können.

\hypertarget{inhaltpathlabelmi-2017modulbeschreibungen-bachelorba_einfhrunginbetriebssystemeundrechnerarchitektur}{%
\section*{Inhalt\label{/mi-2017/modulbeschreibungen-bachelor/BA_EinfhrunginBetriebssystemeundRechnerarchitektur}}\label{inhaltpathlabelmi-2017modulbeschreibungen-bachelorba_einfhrunginbetriebssystemeundrechnerarchitektur}}

\begin{itemize}
\tightlist
\item
  Betriebssysteme aus Nutzersicht: Dateisysteme, Parallele Prozesse,
  Sicherheit in Betriebssystemen
\item
  bei Rechnerkomponenten: grundlegende Architekturen, Darstellung von
  Daten, interne Bussysteme, Prozessoren, Festplatten,
  Peripherieschnittstellen, Parallelrechner
\end{itemize}

\hypertarget{literaturpathlabelmi-2017modulbeschreibungen-bachelorba_einfhrunginbetriebssystemeundrechnerarchitektur}{%
\section*{Literatur\label{/mi-2017/modulbeschreibungen-bachelor/BA_EinfhrunginBetriebssystemeundRechnerarchitektur}}\label{literaturpathlabelmi-2017modulbeschreibungen-bachelorba_einfhrunginbetriebssystemeundrechnerarchitektur}}

\begin{itemize}
\tightlist
\item
  Vorlesungsunterlagen: kommentierte Foliensammlung
\item
  Tanenbaum: „Rechnerarchitektur``
\item
  Tanenbaum: „Modern Operating Systems``
\end{itemize}

\hypertarget{einfuxfchrung-in-die-medieninformatikpathlabelmi-2017modulbeschreibungen-bachelorba_einfhrungindiemedieninformatik}{%
\chapter{Einführung in die
Medieninformatik\label{/mi-2017/modulbeschreibungen-bachelor/BA_EinfhrungindieMedieninformatik}}\label{einfuxfchrung-in-die-medieninformatikpathlabelmi-2017modulbeschreibungen-bachelorba_einfhrungindiemedieninformatik}}

\begin{modulHead}
\textbf{Modulverantwortlich}: Prof.~Christian Noss,
Prof.~Dr.~Mirjam Blümm, Prof.~Dr.~Irma Lindt, Prof.~Hans Kornacher,
Prof.~Dr.~Gerhard
Hartmann
\end{modulHead}
\begin{modulHead}
\textbf{Studiensemester}:
1
\end{modulHead}
\begin{modulHead}
\textbf{Sprache}:
deutsch
\end{modulHead}
\begin{modulHead}
\textbf{Kreditpunkte}:
5
\end{modulHead}
\begin{modulHead}
\textbf{Typ}:
Pflichtmodul
\end{modulHead}
\begin{modulHead}
\textbf{Prüfungsleistung}:
Projektpräsentation(30\%) und schriftliche
Ausarbeitung(70\%)
\end{modulHead}


\hypertarget{lehrformswspathlabelmi-2017modulbeschreibungen-bachelorba_einfhrungindiemedieninformatik}{%
\section*{Lehrform/SWS\label{/mi-2017/modulbeschreibungen-bachelor/BA_EinfhrungindieMedieninformatik}}\label{lehrformswspathlabelmi-2017modulbeschreibungen-bachelorba_einfhrungindiemedieninformatik}}

4 SWS: Seminar 3 SWS; Übung 1 SWS

Seminar mit eingebetteten Übungselementen und Projektarbeit.

\hypertarget{arbeitsaufwandpathlabelmi-2017modulbeschreibungen-bachelorba_einfhrungindiemedieninformatik}{%
\section*{Arbeitsaufwand\label{/mi-2017/modulbeschreibungen-bachelor/BA_EinfhrungindieMedieninformatik}}\label{arbeitsaufwandpathlabelmi-2017modulbeschreibungen-bachelorba_einfhrungindiemedieninformatik}}

Gesamtaufwand 150h, davon

\begin{itemize}
\tightlist
\item
  30h Seminar
\item
  10h Übung
\item
  40h Projekt
\item
  70h Selbststudium
\end{itemize}

\hypertarget{angestrebte-lernergebnissepathlabelmi-2017modulbeschreibungen-bachelorba_einfhrungindiemedieninformatik}{%
\section*{Angestrebte
Lernergebnisse\label{/mi-2017/modulbeschreibungen-bachelor/BA_EinfhrungindieMedieninformatik}}\label{angestrebte-lernergebnissepathlabelmi-2017modulbeschreibungen-bachelorba_einfhrungindiemedieninformatik}}

Die Studierenden können die inhaltlichen Ausrichtungen und die
Zielsetzungen der Lehr- und Anwendungsdisziplin Medieninformatik
benennen und gegenüber verwandten oder ähnlichen Disziplinen abgrenzen.

Die Studierenden kennen Grundkonzepte der Informatik (z.B.
Anforderungen) sowie audiovisueller und interaktiver Medientechnologien,
kennen architekturelle Alternativen interaktiver Systeme und kennen
Gestaltungsdimensionen für deren Informations- und Kommunikationsinhalte
und können diese Kenntnisse auf eine gegebene Problemstellung anwenden.

Die Studierenden sind sensibilisiert für Modellierungs- und
Entwicklungsaufgaben von medienbasierten Software-Systemen zur
Unterstützung menschlichen Handelns in betriebliche, sozialen und
privaten Kontexten.

Sie kennen grundlegende Konzepte, Prozesse/Verfahren und Modelle der
Medieninformatik und haben erste Projekterfahrungen gesammelt. Sie
können Systemkonzeptionen, zugehörige Modellierungen, Abwägungen und
Artefakte für ein Fachpublikum angemessen dokumentieren und mittels
verschiedener medialer Formen kommunizieren.

\hypertarget{inhaltpathlabelmi-2017modulbeschreibungen-bachelorba_einfhrungindiemedieninformatik}{%
\section*{Inhalt\label{/mi-2017/modulbeschreibungen-bachelor/BA_EinfhrungindieMedieninformatik}}\label{inhaltpathlabelmi-2017modulbeschreibungen-bachelorba_einfhrungindiemedieninformatik}}

Workshops zu grundlegenden projektrelevanten Themenfeldern (wie:
Datenmodellierung, Pseudo-Code, Kommunikation in verteilen medialen
Systeme, Visual Thinking, Storytelling, Anforderungen) und deren
Anwendung, illustriert anhand von Fallstudien.

Teambasiertes Projekt, welches ausgehend von Kontextszenarien eine (oder
mehrere) Problemstellung(en) umreißt, zu dem Lösungen konzipiert und
prototypisch umgesetzt, dokumentiert und einem Fachpublikum präsentiert
werden müssen.

\hypertarget{medienformenpathlabelmi-2017modulbeschreibungen-bachelorba_einfhrungindiemedieninformatik}{%
\section*{Medienformen\label{/mi-2017/modulbeschreibungen-bachelor/BA_EinfhrungindieMedieninformatik}}\label{medienformenpathlabelmi-2017modulbeschreibungen-bachelorba_einfhrungindiemedieninformatik}}

\begin{itemize}
\tightlist
\item
  Beamer-gestützte Vorlesungen (Folien in elektronischer Form)
\item
  Vorträge
\item
  verschiedene Präsentationsmaterialien (Whiteboard, Poster, etc.)
\item
  Einsatz von Bild- und Videobearbeitungssoftware
\item
  Umgang mit Kameras im Projektteil
\end{itemize}

\hypertarget{literaturpathlabelmi-2017modulbeschreibungen-bachelorba_einfhrungindiemedieninformatik}{%
\section*{Literatur\label{/mi-2017/modulbeschreibungen-bachelor/BA_EinfhrungindieMedieninformatik}}\label{literaturpathlabelmi-2017modulbeschreibungen-bachelorba_einfhrungindiemedieninformatik}}

\begin{itemize}
\tightlist
\item
  Michael Herczeg: Einführung in die Medieninformatik, Oldenbourg
  Verlag, 2006, ISBN: 3-486-581-031
\item
  Chris Rupp et al: Requirements-Engineering und -Management: Aus der
  Praxis von klassisch bis agil, Carl Hanser Verlag; 6-te Auflage, 2014,
  ISBN-10: 3446438939
\end{itemize}

\hypertarget{entwicklungsprojektpathlabelmi-2017modulbeschreibungen-bachelorba_entwicklungsprojekt}{%
\chapter{Entwicklungsprojekt\label{/mi-2017/modulbeschreibungen-bachelor/BA_Entwicklungsprojekt}}\label{entwicklungsprojektpathlabelmi-2017modulbeschreibungen-bachelorba_entwicklungsprojekt}}

\begin{modulHead}
\textbf{Modulverantwortlich}: Prof.~Dr.~Gerhard
Hartmann, Prof.~Hans Kornacher, Prof.~Dr.~Mirjam Blümm, Prof.~Christian
Noss
\end{modulHead}
\begin{modulHead}
\textbf{Studiensemester}:
5
\end{modulHead}
\begin{modulHead}
\textbf{Sprache}:
deutsch
\end{modulHead}
\begin{modulHead}
\textbf{Kreditpunkte}:
10
\end{modulHead}
\begin{modulHead}
\textbf{Typ}:
Pflichtmodul
\end{modulHead}
\begin{modulHead}
\textbf{Prüfungsleistung}:
Projekt, Projektdokumentation und
Meilensteinpräsentationen
\end{modulHead}


\hypertarget{lehrformswspathlabelmi-2017modulbeschreibungen-bachelorba_entwicklungsprojekt}{%
\section*{Lehrform/SWS\label{/mi-2017/modulbeschreibungen-bachelor/BA_Entwicklungsprojekt}}\label{lehrformswspathlabelmi-2017modulbeschreibungen-bachelorba_entwicklungsprojekt}}

2 SWS: Seminar 2 SWS; Projektarbeit

\hypertarget{arbeitsaufwandpathlabelmi-2017modulbeschreibungen-bachelorba_entwicklungsprojekt}{%
\section*{Arbeitsaufwand\label{/mi-2017/modulbeschreibungen-bachelor/BA_Entwicklungsprojekt}}\label{arbeitsaufwandpathlabelmi-2017modulbeschreibungen-bachelorba_entwicklungsprojekt}}

Gesamtaufwand 300h, davon

\begin{itemize}
\tightlist
\item
  36h Seminar
\item
  264h Selbststudium
\end{itemize}

\hypertarget{angestrebte-lernergebnissepathlabelmi-2017modulbeschreibungen-bachelorba_entwicklungsprojekt}{%
\section*{Angestrebte
Lernergebnisse\label{/mi-2017/modulbeschreibungen-bachelor/BA_Entwicklungsprojekt}}\label{angestrebte-lernergebnissepathlabelmi-2017modulbeschreibungen-bachelorba_entwicklungsprojekt}}

Die Studierenden sollen vertiefende Kenntnisse in die Methoden und
Techniken aus einem ausgewählten Modul aus den ersten vier Fachsemestern
des Studiums erlangen und diese in der Konzeption und prototypischen
Realisierung eines interaktiven Systems oder Mediums anwenden. Dadurch
sollen sie eigene Erfahrungen in der Projektabwicklung mit
Medieninformatik-spezifischen Fragestellungen und in der Teamarbeit
sammeln und eine reflektierend-kritische Haltung zu methodischen
Ansätzen und Entwicklungsmodellen entwickeln. Ziel ist es eine, mit
eigenen praktischen Erfahrungen fundierte Methodenkompetenz zu erlangen.

Die Studierenden sollen darüberhinaus lernen, die Vorgehensweise und die
Ergebnisse ihres Projektes in einem kritischen Diskurs vor einem
Fachpublikum zu vertreten, um in der Berufspraxis ihre Herangehensweise
und Projektergebnisse vertreten zu können.

\hypertarget{inhaltpathlabelmi-2017modulbeschreibungen-bachelorba_entwicklungsprojekt}{%
\section*{Inhalt\label{/mi-2017/modulbeschreibungen-bachelor/BA_Entwicklungsprojekt}}\label{inhaltpathlabelmi-2017modulbeschreibungen-bachelorba_entwicklungsprojekt}}

Die Projekte werden in Teams durchgeführt. Zunächst wird von den Teams
ein Modul aus den ersten vier Fachsemestern gewählt, welche die
fachliche Perspektive für das Entwicklungsprojekt bestimmen. In
Absprache mit den Lehrenden werden dann Projektziele festgelegt.

Aus dem Methoden- und Technikkatalog wird in Absprache mit den Lehrenden
eine Auswahl der einzusetzenden Entwicklungstechniken und -methoden
sowie der einzuhaltenden Entwicklungsmodelle getroffen und
Qualitätssicherungsmaßnahmen und das Prozessmanagement definiert.

Die Lehrenden bieten dann während der Projektdurchführung Beratung bzgl.
des adäquaten Einsatzes der gewählten Methoden und Techniken.
Zwischenstände des Projektes werden zu definierten Meilensteinen von den
Projektteams präsentiert. Die Präsentation der Projektergebnisse findet
in einem Plenum mit kritischer Diskussion der Methoden und Ergebnisse
statt.

\hypertarget{grundlagen-des-webpathlabelmi-2017modulbeschreibungen-bachelorba_grundlagen-des-web}{%
\chapter{Grundlagen des
Web\label{/mi-2017/modulbeschreibungen-bachelor/BA_Grundlagen-des-web}}\label{grundlagen-des-webpathlabelmi-2017modulbeschreibungen-bachelorba_grundlagen-des-web}}

\begin{modulHead}
\textbf{Modulverantwortlich}: Dirk
Breuer
\end{modulHead}
\begin{modulHead}
\textbf{Studiensemester}:
3
\end{modulHead}
\begin{modulHead}
\textbf{Sprache}:
deutsch
\end{modulHead}
\begin{modulHead}
\textbf{Kreditpunkte}:
5
\end{modulHead}
\begin{modulHead}
\textbf{Typ}:
Pflichtmodul
\end{modulHead}
\begin{modulHead}
\textbf{Prüfungsleistung}:
Mündliche Prüfung und Projektarbeit
\end{modulHead}


\hypertarget{kurzbeschreibungpathlabelmi-2017modulbeschreibungen-bachelorba_grundlagen-des-web}{%
\section*{Kurzbeschreibung\label{/mi-2017/modulbeschreibungen-bachelor/BA_Grundlagen-des-web}}\label{kurzbeschreibungpathlabelmi-2017modulbeschreibungen-bachelorba_grundlagen-des-web}}

In der Veranstaltung werden wesentliche Grundideen,
Interaktionsprinzipien, Contentarchitekturen und Sicherheitsmechanismen
eingeführt, die das Web als Medium konstituieren.

\hypertarget{lehrformswspathlabelmi-2017modulbeschreibungen-bachelorba_grundlagen-des-web}{%
\section*{Lehrform/SWS\label{/mi-2017/modulbeschreibungen-bachelor/BA_Grundlagen-des-web}}\label{lehrformswspathlabelmi-2017modulbeschreibungen-bachelorba_grundlagen-des-web}}

4 SWS: Seminar 2 SWS; Workshop 2 SWS

\hypertarget{arbeitsaufwandpathlabelmi-2017modulbeschreibungen-bachelorba_grundlagen-des-web}{%
\section*{Arbeitsaufwand\label{/mi-2017/modulbeschreibungen-bachelor/BA_Grundlagen-des-web}}\label{arbeitsaufwandpathlabelmi-2017modulbeschreibungen-bachelorba_grundlagen-des-web}}

Gesamtaufwand 150h, davon

\begin{itemize}
\tightlist
\item
  36h Vorlesung
\item
  36h Seminar
\item
  78h Selbststudium
\end{itemize}

\hypertarget{angestrebte-lernergebnissepathlabelmi-2017modulbeschreibungen-bachelorba_grundlagen-des-web}{%
\section*{Angestrebte
Lernergebnisse\label{/mi-2017/modulbeschreibungen-bachelor/BA_Grundlagen-des-web}}\label{angestrebte-lernergebnissepathlabelmi-2017modulbeschreibungen-bachelorba_grundlagen-des-web}}

In dem Modul sollen die Teilnehmerinnen und Teilnehmer wesentliche
Grundlagen des Web und aktuelle Entwicklungen im Web auf konzeptioneller
Ebene erfassen und diskutieren können und einige davon auf Ebene der
Programmierung umsetzen können. Das Ziel ist, dass die Studierenden

\begin{itemize}
\tightlist
\item
  wesentliche Grundideen, Interaktionsprinzipien, Contentarchitekturen
  und Sicherheitsmechanismen, die das Web als Medium konstituieren
  erklären können und
\item
  moderne Webanwendungen auf der Basis von Fachbegriffen analysieren und
  einordnen können, um kompetent am fachlichen Diskurs über
  Eigenschaften, Auswirkungen und Gestaltungsalternativen von Web
  Anwendungen teilnehmen zu können.
\item
  verteilte Web Anwendungen ggfs. nach einer Einarbeitung in konkrete
  Technologien oder Rahmenwerke als Proof-of-Concept realisieren
  (programmieren) können.
\end{itemize}

\hypertarget{inhaltpathlabelmi-2017modulbeschreibungen-bachelorba_grundlagen-des-web}{%
\section*{Inhalt\label{/mi-2017/modulbeschreibungen-bachelor/BA_Grundlagen-des-web}}\label{inhaltpathlabelmi-2017modulbeschreibungen-bachelorba_grundlagen-des-web}}

Im Grundlagenteil der Veranstaltung werden wesentliche Konzepte
vermittelt, die zur Konzeption, Diskussion und Realisierung von Diensten
im Web benötigt werden. Die Konzepte sind wichtig um als
Medieninformatiker bzw. Medieninformatikerin kompetent Aufgaben des
Berufsalltags lösen zu können und an Fachdiskussionen teilnehmen zu
können. Themen sind u.a.:

\begin{itemize}
\tightlist
\item
  Web Architektur des W3C
\item
  Offfenheit und Verwendung von Standards als Prinzip
\item
  Interaktionsformen: Synchrone Interaktion auf der Basis von REST,
  asynchrone Interaktion mit Publish/Subscribe
\item
  Fallstudien: Open Data, Social Coding
\item
  Ausgewählte Sicherheitsmechanismen im Web
\item
  Inhaltsarchitekturen: XML, JSON, Microformate, RDFa
\end{itemize}

Die Grundlagen werden nur zu einem geringen Teil durch
Seminarveranstaltungen vermittelt. Im Wesentlichen sollen sie durch

\begin{itemize}
\tightlist
\item
  das Erarbeiten von Lehrbuch Texten,
\item
  die Bearbeitung von Fragen und Aufgaben und
\item
  die Diskussion von Fragen und Lösungen sowohl in Kleingruppen als auch
  im Plenum
\end{itemize}

Das Ziel des Workshop ist die Entwicklung und das Deploynent eines
Webservice für ein selbstgewältes Problemszenario, der eine signifikante
Abnwendungslogik realisiert und seinerseits anwendungsbezogen einen
externen Web Service einbindet. Es soll keine Nutzerschnittstelle
entwickekt werden sondern ausschließlich ein REST oder/und PubSub API.

Im Kontext des Projektes sollen die zentralen Konzepte
`'Datenmodellierung'`,'`synchrone Interaktion'' und - soweit zeitlich
möglich - `'asynchrone Interaktion'' aus dem Grundlagenteil durch
praktische Umsetzung mit aktuellen Werkzeugen vertieft werden.

\hypertarget{medienformenpathlabelmi-2017modulbeschreibungen-bachelorba_grundlagen-des-web}{%
\section*{Medienformen\label{/mi-2017/modulbeschreibungen-bachelor/BA_Grundlagen-des-web}}\label{medienformenpathlabelmi-2017modulbeschreibungen-bachelorba_grundlagen-des-web}}

\begin{itemize}
\tightlist
\item
  Folienpräsentation
\item
  Auschnitte aus der Literatur als Leseaufgaben und Fallstudien
\end{itemize}

\hypertarget{literaturdas-vom-w3c-herausgegebene-dokument-uxfcber-die-architektur-des-webpathlabelmi-2017modulbeschreibungen-bachelorba_grundlagen-des-web}{%
\section*{LiteraturDas vom W3C herausgegebene Dokument über die
Architektur des
Web\label{/mi-2017/modulbeschreibungen-bachelor/BA_Grundlagen-des-web}}\label{literaturdas-vom-w3c-herausgegebene-dokument-uxfcber-die-architektur-des-webpathlabelmi-2017modulbeschreibungen-bachelorba_grundlagen-des-web}}

\begin{itemize}
\tightlist
\item
  Tilkov et al.: REST und HTTP, dpunkt.verlag 2015
\item
  Tanenbaum et al.: Distributed Systems, Pearson 2007
\item
  Randy Conolly, Richard Hoar: Fundamentals of Web Development, Pearson
  Publishing 2015
\item
  Hugh Taylor et al.: Event-Driven Architecture - How SOA Enables the
  Real-Time Enterprise, Addison-Wesley 2009
\item
  Webber: REST in Practice, OReilly 2011
\item
  Sam Newman: Building Micro Services, OReilly 2015
\item
  James Governor et al.: Web 2.0 Architectures, OReilly 2009
\item
  Rajkumar Buyya (ed.): Internet of Things: Principles and Paradigms,
  Morgan Kaufmann 2016
\end{itemize}

\hypertarget{kommunikationstechnik-und-netzepathlabelmi-2017modulbeschreibungen-bachelorba_kommunikationstechnikundnetze}{%
\chapter{Kommunikationstechnik und
Netze\label{/mi-2017/modulbeschreibungen-bachelor/BA_KommunikationstechnikundNetze}}\label{kommunikationstechnik-und-netzepathlabelmi-2017modulbeschreibungen-bachelorba_kommunikationstechnikundnetze}}

\begin{modulHead}
\textbf{Modulverantwortlich}: Prof.~Dr.~Hans L.
Stahl
\end{modulHead}
\begin{modulHead}
\textbf{Studiensemester}:
3
\end{modulHead}
\begin{modulHead}
\textbf{Sprache}:
deutsch
\end{modulHead}
\begin{modulHead}
\textbf{Kreditpunkte}:
5
\end{modulHead}
\begin{modulHead}
\textbf{Typ}:
Pflichtmodul
\end{modulHead}
\begin{modulHead}
\textbf{Prüfungsleistung}:
Schriftliche Prüfung, sowie erfolgreiche Teilnahme am Praktikum als
Prüfungsvorleistung
\end{modulHead}


\hypertarget{lehrformswspathlabelmi-2017modulbeschreibungen-bachelorba_kommunikationstechnikundnetze}{%
\section*{Lehrform/SWS\label{/mi-2017/modulbeschreibungen-bachelor/BA_KommunikationstechnikundNetze}}\label{lehrformswspathlabelmi-2017modulbeschreibungen-bachelorba_kommunikationstechnikundnetze}}

Vorlesung, Praktikum

\hypertarget{angestrebte-lernergebnissepathlabelmi-2017modulbeschreibungen-bachelorba_kommunikationstechnikundnetze}{%
\section*{Angestrebte
Lernergebnisse\label{/mi-2017/modulbeschreibungen-bachelor/BA_KommunikationstechnikundNetze}}\label{angestrebte-lernergebnissepathlabelmi-2017modulbeschreibungen-bachelorba_kommunikationstechnikundnetze}}

Die Studierenden sollen

\begin{itemize}
\tightlist
\item
  Prinzipien und Grundlagen von technischen Kommunikations­vor­gängen
  kennen lernen,
\item
  Protokolle als wesentliche Grundlage der Kommunikationstechnik im
  Detail verstehen (Internet-Protokolle, Multimedia-Protokolle,
  TK-Protokolle, Dienste)
\item
  Einsatz und Nutzung von Kommunikations­tech­nik praxistypisch kennen
  lernen,
\item
  in der Lage sein, selbstständig Netzstrukturen zu bewerten, Netze zu
  analysieren und zu konzipieren (unter Anwendung von
  Netz­analyse­werkzeugen und -methoden).
\end{itemize}

\hypertarget{inhaltpathlabelmi-2017modulbeschreibungen-bachelorba_kommunikationstechnikundnetze}{%
\section*{Inhalt\label{/mi-2017/modulbeschreibungen-bachelor/BA_KommunikationstechnikundNetze}}\label{inhaltpathlabelmi-2017modulbeschreibungen-bachelorba_kommunikationstechnikundnetze}}

Grundbegriffe und Grundlagen, Kommunikationssysteme (Modelle,
Grundbegriffe), Protokolle, Schnittstellen, Dienste, Architekturmodelle
(OSI-Referenzmodell, TCP/IP-Protokollfamilie), Standardisierung,
TCP/IP-Protokollfamilie als Grundlage des Internet, Schichtenmodell und
Protokolle im Detail, Adressierung, ausgewählte Anwendungen,
Klassifizierung von Netzen / Topologien / Technologien, Wegewahl /
Vermittlung / Routing, Vermittlungsprinzipien, Routing-Verfahren und~
Protokolle, Internet-spezifische Verfahren, Multimedia-Netze,
Dienstgüte, Internet-Telefonie, Realisierung von Multimedia-Netzen,
Netzsicherheit, grundlegende Begriffe der „IT-Sicherheit``, typische
Bedrohungen in Netzen, Beispielszenarien

\hypertarget{medienformenpathlabelmi-2017modulbeschreibungen-bachelorba_kommunikationstechnikundnetze}{%
\section*{Medienformen\label{/mi-2017/modulbeschreibungen-bachelor/BA_KommunikationstechnikundNetze}}\label{medienformenpathlabelmi-2017modulbeschreibungen-bachelorba_kommunikationstechnikundnetze}}

\begin{itemize}
\tightlist
\item
  Vorlesung im Hörsaal (PowerPoint und Beamer)
\item
  Praktikum an Rechnern des KTDS-Labors; Ressourcen:
  Netzanalysesoftware,div. Netzüberwachungssoftware, E-Mail-Server und
  -Clients, DNS-Server, ggf. weitereServer-Implementierungen
\end{itemize}

\hypertarget{literaturpathlabelmi-2017modulbeschreibungen-bachelorba_kommunikationstechnikundnetze}{%
\section*{Literatur\label{/mi-2017/modulbeschreibungen-bachelor/BA_KommunikationstechnikundNetze}}\label{literaturpathlabelmi-2017modulbeschreibungen-bachelorba_kommunikationstechnikundnetze}}

\begin{itemize}
\tightlist
\item
  Wird in der Veranstaltung bekannt gegeben
\end{itemize}

\hypertarget{medienrecht-medien-gesellschaftpathlabelmi-2017modulbeschreibungen-bachelorba_mug}{%
\chapter{Medienrecht, Medien \&
Gesellschaft\label{/mi-2017/modulbeschreibungen-bachelor/BA_MUG}}\label{medienrecht-medien-gesellschaftpathlabelmi-2017modulbeschreibungen-bachelorba_mug}}

\begin{modulHead}
\textbf{Modulverantwortlich}: Prof.~Dr.~Mario
Winter
\end{modulHead}
\begin{modulHead}
\textbf{Studiensemester}:
5
\end{modulHead}
\begin{modulHead}
\textbf{Sprache}:
deutsch
\end{modulHead}
\begin{modulHead}
\textbf{Kreditpunkte}:
5
\end{modulHead}
\begin{modulHead}
\textbf{Typ}:
Pflichtmodul
\end{modulHead}
\begin{modulHead}
\textbf{Prüfungsleistung}:
Präsentation im OpenSpace, Klausur und Klausur
\end{modulHead}


\hypertarget{lehrformswspathlabelmi-2017modulbeschreibungen-bachelorba_mug}{%
\section*{Lehrform/SWS\label{/mi-2017/modulbeschreibungen-bachelor/BA_MUG}}\label{lehrformswspathlabelmi-2017modulbeschreibungen-bachelorba_mug}}

4 SWS: Vorlesung 2 SWS; Übung 2 SWS

\hypertarget{arbeitsaufwandpathlabelmi-2017modulbeschreibungen-bachelorba_mug}{%
\section*{Arbeitsaufwand\label{/mi-2017/modulbeschreibungen-bachelor/BA_MUG}}\label{arbeitsaufwandpathlabelmi-2017modulbeschreibungen-bachelorba_mug}}

Gesamtaufwand: 150h, davon

\begin{itemize}
\tightlist
\item
  36h Vorlesung
\item
  36h Übung
\item
  78h Selbststudium
\end{itemize}

\hypertarget{angestrebte-lernergebnissepathlabelmi-2017modulbeschreibungen-bachelorba_mug}{%
\section*{Angestrebte
Lernergebnisse\label{/mi-2017/modulbeschreibungen-bachelor/BA_MUG}}\label{angestrebte-lernergebnissepathlabelmi-2017modulbeschreibungen-bachelorba_mug}}

Informatikerinnen und Informatiker analysieren und konstruieren
sozio-technische Systeme und entwickeln dabei semiotische Artefakte wie
z.B. Spezifikationen, Programme und Handbücher. Die entwickelten Systeme
bilden einerseits soziale Wirklichkeit in vielfältiger Form ab und
ändern andererseits diese Wirklichkeit durch ihren Einsatz.

Die Studierenden sollen befähigt werden

\begin{itemize}
\tightlist
\item
  ethische und rechtliche Aspekte des Einsatzes von Informatik-Systemen
  zu charakterisieren und
\item
  ein kritisches Bewusstsein für die aktuellen Fragen des
  wechselseitigen Einflusses von Informatik und Gesellschaft zu
  entwickeln sowie
\item
  die Grundbegriffe des deutschen Privatrechts zu verstehen und sich im
  dazugehörigen Gesetzeswerk zu orientieren,
\item
  um die unterschiedlichen Wechselwirkungen zwischen Informatik-Systemen
  und ihrem Einsatzumfeld erkennen und bewerten und insbesondere im
  Bereich des Vertragsrechts selbständige Lösungsvorschläge erarbeiten
  zu können.
\end{itemize}

\hypertarget{inhaltpathlabelmi-2017modulbeschreibungen-bachelorba_mug}{%
\section*{Inhalt\label{/mi-2017/modulbeschreibungen-bachelor/BA_MUG}}\label{inhaltpathlabelmi-2017modulbeschreibungen-bachelorba_mug}}

\hypertarget{informatik-und-gesellschaftpathlabelmi-2017modulbeschreibungen-bachelorba_mug}{%
\subsection*{Informatik und
Gesellschaft\label{/mi-2017/modulbeschreibungen-bachelor/BA_MUG}}\label{informatik-und-gesellschaftpathlabelmi-2017modulbeschreibungen-bachelorba_mug}}

Die Wechselwirkungen zwischen den von Informatikern entwickelten
Systemen und ihrem Einsatzumfeld werden in drei großen Themenblöcken
behandelt:

\begin{itemize}
\tightlist
\item
  Informatik und soziale Kontexte
\item
  Komplexität und Sicherheit in sozio-technischenen Systemen
\item
  Systemgestaltung und Verantwortung der Informatik.
\end{itemize}

Beispielhafte Inhalte:

\begin{itemize}
\tightlist
\item
  Geschichte der Informatik
\item
  Bildung und Wissenschaft
\item
  Wissenschaften und Gesellschaft
\item
  Digitale Medien und Internet
\item
  Datenschutz und Überwachungstechniken
\item
  Informatik und Gestaltung
\item
  partizipative Systemgestaltung
\item
  Open Source
\item
  Ethische Leitlinien für Informatiker
\item
  Normen und Standards
\item
  philosophische Aspekte der Informatik
\end{itemize}

\hypertarget{rechtpathlabelmi-2017modulbeschreibungen-bachelorba_mug}{%
\subsection*{Recht\label{/mi-2017/modulbeschreibungen-bachelor/BA_MUG}}\label{rechtpathlabelmi-2017modulbeschreibungen-bachelorba_mug}}

\begin{itemize}
\tightlist
\item
  Einführung in das deutsche Privatrecht, insbesondere in das BGB.
\item
  Schwerpunkt im Schuldrecht, hier insbesondere im Vertragsrecht.
\item
  Besondere Aspekte des Verbraucherschutzes und der inhaltlichen
  Gestaltung von Verträgen.
\item
  Im Allgemeinen Teil des BGB wird auf den Vertragsschluss, die
  Willenerklärung als rechtsgeschäftliches Gestaltungsmittel und die
  allgemeinen Anforderungen an die Vertragspartner eingegangen.
\end{itemize}

Klausur (60 Min.)

\hypertarget{medienformenpathlabelmi-2017modulbeschreibungen-bachelorba_mug}{%
\section*{Medienformen\label{/mi-2017/modulbeschreibungen-bachelor/BA_MUG}}\label{medienformenpathlabelmi-2017modulbeschreibungen-bachelorba_mug}}

Beamergestützte Vorträge

\hypertarget{literaturpathlabelmi-2017modulbeschreibungen-bachelorba_mug}{%
\section*{Literatur\label{/mi-2017/modulbeschreibungen-bachelor/BA_MUG}}\label{literaturpathlabelmi-2017modulbeschreibungen-bachelorba_mug}}

\hypertarget{iugpathlabelmi-2017modulbeschreibungen-bachelorba_mug}{%
\subsection*{IUG\label{/mi-2017/modulbeschreibungen-bachelor/BA_MUG}}\label{iugpathlabelmi-2017modulbeschreibungen-bachelorba_mug}}

\begin{itemize}
\tightlist
\item
  Sara Baase: A Gift of Fire. Social, Legal, and Ethical Issues in
  Computing. Prentice Hall, Upper Saddle River, 1997
\item
  A.F. Chalmers: Wege der Wissenschaft. 5. Aufl., Springer, Heidelberg,
  2001
\item
  D.M. Hester, P.J. Ford: Computers and Ethics in the Cyberage. Prentice
  Hall, Upper Saddle River, 2001
\item
  P. Gola, C. Klug: Grundzüge des Datenschutzrechts. C.H. Beck, 2003
\item
  M. Pierson, D. Seiler: Internet-Recht im Unternehmen.
  Beck-Rechtsberater im dtv, Deutscher Taschenbuch Verlag, München, 2002
\item
  \url{http://www.gi-ev.de} Arbeitskreis Informatik und Verantwortung,
  Ethische Leitlinien der GI
\item
  \url{http://www.bfd.bund.de} Der Bundesbeauftragte für den Datenschutz
\item
  \url{http://www.aktiv.org/DVD} Deutsche Vereinigung für Datenschutz
\item
  \url{http://www.big-brother-award.org} Überwachungsinformationen
\end{itemize}

\hypertarget{rechtpathlabelmi-2017modulbeschreibungen-bachelorba_mug-1}{%
\subsection*{Recht\label{/mi-2017/modulbeschreibungen-bachelor/BA_MUG}}\label{rechtpathlabelmi-2017modulbeschreibungen-bachelorba_mug-1}}

\begin{itemize}
\tightlist
\item
  Bürgerliches Gesetzbuch in der aktuellen Taschenbuchausgabe des dtv
\end{itemize}

\hypertarget{fakultativpathlabelmi-2017modulbeschreibungen-bachelorba_mug}{%
\subsection*{Fakultativ\label{/mi-2017/modulbeschreibungen-bachelor/BA_MUG}}\label{fakultativpathlabelmi-2017modulbeschreibungen-bachelorba_mug}}

\begin{itemize}
\tightlist
\item
  Eugen Klunziger, Einführung in das Bürgerliche Recht, Verlag Vahlen
\item
  Norbert Ullrich, Wirtschaftsrecht für Betriebswirte, Verlag Neue
  Wirtschaftsbriefe
\end{itemize}

\hypertarget{mathematik-1pathlabelmi-2017modulbeschreibungen-bachelorba_mathematik1}{%
\chapter{Mathematik
1\label{/mi-2017/modulbeschreibungen-bachelor/BA_Mathematik1}}\label{mathematik-1pathlabelmi-2017modulbeschreibungen-bachelorba_mathematik1}}

\begin{modulHead}
\textbf{Modulverantwortlich}: Prof.~Dr.~Wolfgang
Konen
\end{modulHead}
\begin{modulHead}
\textbf{Studiensemester}:
1
\end{modulHead}
\begin{modulHead}
\textbf{Sprache}:
deutsch
\end{modulHead}
\begin{modulHead}
\textbf{Kreditpunkte}:
7
\end{modulHead}
\begin{modulHead}
\textbf{Typ}:
Pflichtmodul
\end{modulHead}
\begin{modulHead}
\textbf{Prüfungsleistung}:
Schriftliche Prüfung, sowie erfolgreiche Teilnahme am Praktikum als
Prüfungsvorleistung
\end{modulHead}


Weitere Infos unter
\url{http://www.gm.fh-koeln.de/~konen/Mathe1-WS/index.htm}.

\hypertarget{lehrformswspathlabelmi-2017modulbeschreibungen-bachelorba_mathematik1}{%
\section*{Lehrform/SWS\label{/mi-2017/modulbeschreibungen-bachelor/BA_Mathematik1}}\label{lehrformswspathlabelmi-2017modulbeschreibungen-bachelorba_mathematik1}}

6 SWS: Vorlesung 3 SWS; Praktikum 1 SWS; Übung 2 SWS

\hypertarget{arbeitsaufwandpathlabelmi-2017modulbeschreibungen-bachelorba_mathematik1}{%
\section*{Arbeitsaufwand\label{/mi-2017/modulbeschreibungen-bachelor/BA_Mathematik1}}\label{arbeitsaufwandpathlabelmi-2017modulbeschreibungen-bachelorba_mathematik1}}

Gesamtaufwand 210h, davon

\begin{itemize}
\tightlist
\item
  54h Vorlesung
\item
  18h Praktikum
\item
  36h Übung
\item
  102h Selbststudium
\end{itemize}

\hypertarget{angestrebte-lernergebnissepathlabelmi-2017modulbeschreibungen-bachelorba_mathematik1}{%
\section*{Angestrebte
Lernergebnisse\label{/mi-2017/modulbeschreibungen-bachelor/BA_Mathematik1}}\label{angestrebte-lernergebnissepathlabelmi-2017modulbeschreibungen-bachelorba_mathematik1}}

Die Studierenden sollen die Fähigkeiten zur Analyse realer oder
geplanter Systeme entwickeln, indem sie praktische Aufgabenstellungen
aus dem Informatik-Umfeld in mathematische Strukturen abstrahieren und
lernen, selbstständig die Modellfindung und die Ergebnisbeurteilung
vorzunehmen. Dabei sollen die Anwendungsbezüge der Mathematik deutlich
werden, z.B. die Bedeutung funktionaler Beziehungen für kontinuierliche
Zusammenhänge, die lineare Algebra z.B als Grundlage der grafischen
Datenverarbeitung und die Analysis zur Verarbeitung von Signalen und zur
Lösung von mathematischen Modellen.

\hypertarget{inhaltpathlabelmi-2017modulbeschreibungen-bachelorba_mathematik1}{%
\section*{Inhalt\label{/mi-2017/modulbeschreibungen-bachelor/BA_Mathematik1}}\label{inhaltpathlabelmi-2017modulbeschreibungen-bachelorba_mathematik1}}

\begin{itemize}
\tightlist
\item
  Grundlagen
\item
  Folgen
\item
  Funktionen
\item
  Differenzialrechnung (1 Veränderliche)
\item
  Integralrechnung
\item
  Lineare Algebra
\end{itemize}

\hypertarget{literaturpathlabelmi-2017modulbeschreibungen-bachelorba_mathematik1}{%
\section*{Literatur\label{/mi-2017/modulbeschreibungen-bachelor/BA_Mathematik1}}\label{literaturpathlabelmi-2017modulbeschreibungen-bachelorba_mathematik1}}

\begin{itemize}
\tightlist
\item
  Skript unter
  \href{http://www.gm.fh-koeln.de/~konen}{www.gm.fh-koeln.de/\textasciitilde konen}
\item
  Teschl, Gerald und Teschl, Susanne: ``Mathematik für Informatiker'',
  Springer Verlag, 4. Auflage, 2013
\item
  Hartmann, Peter: ``Mathematik für Informatiker-Ein praxisbezogenes
  Lehrbuch'' Vieweg Verlag, 475 Seiten, 3. Auflage 2006
\item
  Papula, Lothar: ``Mathematik für Ingenieure und Naturwissenschaftler''
  Vieweg Verlag, 14. Auflage, 2014
\item
  Stingl, Mathematik für Fachhochschulen, Hanser 2003
\end{itemize}

\hypertarget{mathematik-2pathlabelmi-2017modulbeschreibungen-bachelorba_mathematik2}{%
\chapter{Mathematik
2\label{/mi-2017/modulbeschreibungen-bachelor/BA_Mathematik2}}\label{mathematik-2pathlabelmi-2017modulbeschreibungen-bachelorba_mathematik2}}

\begin{modulHead}
\textbf{Modulverantwortlich}: Prof.~Dr.~Wolfgang
Konen
\end{modulHead}
\begin{modulHead}
\textbf{Studiensemester}:
2
\end{modulHead}
\begin{modulHead}
\textbf{Sprache}:
deutsch
\end{modulHead}
\begin{modulHead}
\textbf{Kreditpunkte}:
8
\end{modulHead}
\begin{modulHead}
\textbf{Typ}:
Pflichtmodul
\end{modulHead}
\begin{modulHead}
\textbf{Prüfungsleistung}:
Schriftliche Prüfung, sowie erfolgreiche Teilnahme am Praktikum als
Prüfungsvorleistung
\end{modulHead}


\hypertarget{lehrformswspathlabelmi-2017modulbeschreibungen-bachelorba_mathematik2}{%
\section*{Lehrform/SWS\label{/mi-2017/modulbeschreibungen-bachelor/BA_Mathematik2}}\label{lehrformswspathlabelmi-2017modulbeschreibungen-bachelorba_mathematik2}}

6 SWS: Vorlesung 3 SWS; Praktikum 1 SWS; Übung 2 SWS

\hypertarget{arbeitsaufwandpathlabelmi-2017modulbeschreibungen-bachelorba_mathematik2}{%
\section*{Arbeitsaufwand\label{/mi-2017/modulbeschreibungen-bachelor/BA_Mathematik2}}\label{arbeitsaufwandpathlabelmi-2017modulbeschreibungen-bachelorba_mathematik2}}

Gesamtaufwand 240h, davon

\begin{itemize}
\tightlist
\item
  54h Vorlesung
\item
  18h Praktikum
\item
  36h Übung
\item
  132h Selbststudium
\end{itemize}

\hypertarget{angestrebte-lernergebnissepathlabelmi-2017modulbeschreibungen-bachelorba_mathematik2}{%
\section*{Angestrebte
Lernergebnisse\label{/mi-2017/modulbeschreibungen-bachelor/BA_Mathematik2}}\label{angestrebte-lernergebnissepathlabelmi-2017modulbeschreibungen-bachelorba_mathematik2}}

Die Studierenden sollen die Fähigkeiten zur Analyse realer oder
geplanter Systeme entwickeln, indem sie praktische Aufgabenstellungen
aus dem Informatik-Umfeld in mathematische Strukturen abstrahieren und
lernen, selbstständig die Modellfindung und die Ergebnisbeurteilung
vorzunehmen. Dabei sollen die Anwendungsbezüge der Mathematik deutlich
werden, z.B. die Beziehungen diskreter Strukturen wie der Graphen zu
vielfältigen grundlegenden Datenstrukturen, die Statistik zur
Deskription und Beurteilung von Beobachtungen und die Analysis zur
Verarbeitung von Signalen und zur Lösung von mathematischen Modellen.

\hypertarget{inhaltpathlabelmi-2017modulbeschreibungen-bachelorba_mathematik2}{%
\section*{Inhalt\label{/mi-2017/modulbeschreibungen-bachelor/BA_Mathematik2}}\label{inhaltpathlabelmi-2017modulbeschreibungen-bachelorba_mathematik2}}

\begin{itemize}
\tightlist
\item
  Mehrdimensionale Differenzialrechnung,
\item
  Graphentheorie,
\item
  Kombinatorik, Wahrscheinlichkeitsrechnung und Statistik,
\item
  Komplexe Zahlen,
\item
  Differentialgleichungen.
\end{itemize}

\hypertarget{literaturpathlabelmi-2017modulbeschreibungen-bachelorba_mathematik2}{%
\section*{Literatur\label{/mi-2017/modulbeschreibungen-bachelor/BA_Mathematik2}}\label{literaturpathlabelmi-2017modulbeschreibungen-bachelorba_mathematik2}}

\begin{itemize}
\tightlist
\item
  \begin{enumerate}
  \def\labelenumi{\alph{enumi}.}
  \setcounter{enumi}{18}
  \tightlist
  \item
    Literaturliste auf der Homepage
    \href{http://www.gm.fh-koeln.de/~konen}{www.gm.fh-koeln.de/\textasciitilde konen}
  \end{enumerate}
\item
  Skript unter
  \href{http://www.gm.fh-koeln.de/~konen/Mathe2-SS}{www.gm.fh-koeln.de/\textasciitilde konen/Mathe2-SS}
\end{itemize}

\hypertarget{mensch-computer-interaktionpathlabelmi-2017modulbeschreibungen-bachelorba_mensch-computer_interaktion}{%
\chapter{Mensch-Computer
Interaktion\label{/mi-2017/modulbeschreibungen-bachelor/BA_Mensch-Computer_Interaktion}}\label{mensch-computer-interaktionpathlabelmi-2017modulbeschreibungen-bachelorba_mensch-computer_interaktion}}

\begin{modulHead}
\textbf{Modulverantwortlich}: Robert
Gabriel
\end{modulHead}
\begin{modulHead}
\textbf{Studiensemester}:
2
\end{modulHead}
\begin{modulHead}
\textbf{Sprache}:
deutsch
\end{modulHead}
\begin{modulHead}
\textbf{Kreditpunkte}:
10
\end{modulHead}
\begin{modulHead}
\textbf{Typ}:
Pflichtmodul
\end{modulHead}
\begin{modulHead}
\textbf{Prüfungsleistung}:
individuelles Lernportfolio
\end{modulHead}


\hypertarget{lehrformswspathlabelmi-2017modulbeschreibungen-bachelorba_mensch-computer_interaktion}{%
\section*{Lehrform/SWS\label{/mi-2017/modulbeschreibungen-bachelor/BA_Mensch-Computer_Interaktion}}\label{lehrformswspathlabelmi-2017modulbeschreibungen-bachelorba_mensch-computer_interaktion}}

Vorlesung und Praktikum/Übung

\hypertarget{arbeitsaufwandpathlabelmi-2017modulbeschreibungen-bachelorba_mensch-computer_interaktion}{%
\section*{Arbeitsaufwand\label{/mi-2017/modulbeschreibungen-bachelor/BA_Mensch-Computer_Interaktion}}\label{arbeitsaufwandpathlabelmi-2017modulbeschreibungen-bachelorba_mensch-computer_interaktion}}

Gesamtaufwand 300h, davon

\begin{itemize}
\tightlist
\item
  65h Vorlesung
\item
  65h Praktikum/Übung
\item
  170h Selbststudium
\end{itemize}

\hypertarget{angestrebte-lernergebnissepathlabelmi-2017modulbeschreibungen-bachelorba_mensch-computer_interaktion}{%
\section*{Angestrebte
Lernergebnisse\label{/mi-2017/modulbeschreibungen-bachelor/BA_Mensch-Computer_Interaktion}}\label{angestrebte-lernergebnissepathlabelmi-2017modulbeschreibungen-bachelorba_mensch-computer_interaktion}}

\begin{itemize}
\tightlist
\item
  Die Studierenden erwerben Grundkenntnisse in kognitions-, arbeits- und
  organisations-psychologischen Grundkonzepten und können diese auf
  Problemstellungen im Kontext der Mensch-Computer Interaktion anwenden.
\item
  Die Studierenden kennen Modelle, Methoden, Arbeits- und
  Dokumentationstechniken der Mensch-Computer Interaktion, können sie
  anwenden, kritisch diskutieren und für konkrete Aktivitäten in
  Entwicklungsprojekten unter Abwägung der Alternativen auswählen.
\item
  Sie kennen relevante internationale Normen und Standards, können sie
  anwenden und erarbeitete Ergebnisse kritisch diskutieren und
  einordnen.
\item
  Sie kennen methodische Ansätze benutzer- oder benutzungsorientierter
  Entwicklungsprozesse und können diese systematisch und iterativ auf
  die Konzeption, Realisation, Evaluation und das Redesign von
  interaktiven Systemen anwenden.
\item
  Zudem kennen sie Konzepte und Vorgehensmodelle für die Integration von
  Software- und Usability Engineering in einem Gesamtprozess und können
  diese in Entwicklungsprojekten anwenden.
\item
  Die Studierenden erlangen die Fähigkeit zum fachlichen Diskurs.
\end{itemize}

\hypertarget{inhaltpathlabelmi-2017modulbeschreibungen-bachelorba_mensch-computer_interaktion}{%
\section*{Inhalt\label{/mi-2017/modulbeschreibungen-bachelor/BA_Mensch-Computer_Interaktion}}\label{inhaltpathlabelmi-2017modulbeschreibungen-bachelorba_mensch-computer_interaktion}}

\begin{itemize}
\tightlist
\item
  kognitionspsychologische Grundlagen
\item
  Benutzermodellierung
\item
  Tätigkeitsmodellierung
\item
  Spezifikationsformen für Nutzungskontexte
\item
  Spezifikation von Nutzungsanforderungen
\item
  Interaktionsmodelle
\item
  Interaktionsmodalitäten und --kodalitäten
\item
  Vorgehensmodelle (human-centered, usability-engineering,
  usage-centered design)
\item
  Design-Prinzipien, -Pattern, -Guidelines, -Styleguides
\item
  Prototyping und Sketching
\item
  Evaluation
\end{itemize}

\hypertarget{medienformenpathlabelmi-2017modulbeschreibungen-bachelorba_mensch-computer_interaktion}{%
\section*{Medienformen\label{/mi-2017/modulbeschreibungen-bachelor/BA_Mensch-Computer_Interaktion}}\label{medienformenpathlabelmi-2017modulbeschreibungen-bachelorba_mensch-computer_interaktion}}

\begin{itemize}
\tightlist
\item
  Beamergestützte Vorlesung
\item
  Case Studies
\item
  Lehrfilme
\end{itemize}

\hypertarget{literaturpathlabelmi-2017modulbeschreibungen-bachelorba_mensch-computer_interaktion}{%
\section*{Literatur\label{/mi-2017/modulbeschreibungen-bachelor/BA_Mensch-Computer_Interaktion}}\label{literaturpathlabelmi-2017modulbeschreibungen-bachelorba_mensch-computer_interaktion}}

\begin{itemize}
\tightlist
\item
  Dix, A.; Finlay, J.; Abowd, G. \& Beale, R.: Human-Computer
  Interaction. Harlow, Pearson, 2004 (3rd ed.),
\item
  Benyon, D., Turner, S. Turner, P. Designing Interactive Systems:
  People, Activities, Contexts, Technologies, Addison Wesley, 2005,
\item
  Anderson, J.R.: Kognitive Psychologie. Heidelberg, Springer, 2001 (3.
  Auflage).
\item
  Beyer H. \& Holtzblatt K.: Contextual Design: Defining
  Customer-Centered Systems. San Francisco Morgan Kaufmann, 1997.
\item
  Cockburn, A.: Writing Effective Use Cases. Boston, Addison-Wesley,
  2000.
\item
  Constantine, L.; Lockwood, L.: Software for Use, ACM Press, 1999.
\item
  Dumas, J.S. \& Redish, J.C.: A Practical Guide to Usability Testing.
  Exter, Intellect Books, 1999 (rev. edition).
\item
  Hacker, W.: Allgemeine Arbeitspsychologie. Bern, Huber, 1998.
\item
  Hackos, J. \& Redish, J.: User and Task Analysis for Interface Design.
  New York, Wiley, 1998.
\item
  Holtzblatt K.; Wendell, J.B. \& Wood, S.: Rapid Contextual Design. A
  How-to Guide to Key Techniques for User-Centered Design. San
  Francisco, Morgan Kaufmann, 2005.
\item
  Johnson, J.: GUI Bloopers. San Francisco, Morgan Kaufmann, 2000.
\item
  Kulak, D. \& Guiney, E.: Use Cases. Requirements in Context. Boston,
  Addison-Wesley, 2000.
\item
  Mayhew, D.: The Usability Engineering Lifecycle. A Practitioner´s
  Handbook for User Interface Design. San Francisco: Morgan Kaufmann,
  1999.
\item
  Nielsen, J. \& Mack, R.L. (eds.): Usability Inspection Methods.
  NewYork, Wiley, 1994.
\item
  Preece, J; Rogers, Y. \& Sharp, H.: Interaction Design. Beyond
  Human-Computer Interaction. NewYork, Wiley, 2002.
\item
  Rosson, M.B. \& Carroll, J.M.: Usability Engineering. Scenario-Based
  Development of Human-Computer Interaction. San Francisco, Morgan
  Kaufmann, 2002.
\item
  Snyder, C: Paper Prototyping. San Francisco, Morgan Kaufmann, 2003.
\item
  Ulich, E.: Arbeitspsychologie. Stuttgart, Schäffer-Poeschel, 2001
  (5.Auflage).
\end{itemize}

\hypertarget{mobile-computingpathlabelmi-2017modulbeschreibungen-bachelorba_mobile-computing}{%
\chapter{Mobile
Computing\label{/mi-2017/modulbeschreibungen-bachelor/BA_Mobile-Computing}}\label{mobile-computingpathlabelmi-2017modulbeschreibungen-bachelorba_mobile-computing}}

\begin{modulHead}
\textbf{Modulverantwortlich}: Prof.~Dr.~Matthias
Böhmer
\end{modulHead}
\begin{modulHead}
\textbf{Studiensemester}:
4
\end{modulHead}
\begin{modulHead}
\textbf{Sprache}:
deutsch
\end{modulHead}
\begin{modulHead}
\textbf{Kreditpunkte}:
5
\end{modulHead}
\begin{modulHead}
\textbf{Typ}:
Pflichtmodul
\end{modulHead}
\begin{modulHead}
\textbf{Prüfungsleistung}:
Mündliche Prüfung (30\%) und Projektarbeit (70\%)
\end{modulHead}


\hypertarget{kurzbeschreibungpathlabelmi-2017modulbeschreibungen-bachelorba_mobile-computing}{%
\section*{Kurzbeschreibung\label{/mi-2017/modulbeschreibungen-bachelor/BA_Mobile-Computing}}\label{kurzbeschreibungpathlabelmi-2017modulbeschreibungen-bachelorba_mobile-computing}}

In diesem Modul erfahren Studierende die Relevanz, Herausforderungen und
Techniken der Entwicklung mobiler Software. Sie können danach Apps für
Smartphones entwerfen, implementieren und managen. Das Modul befähigt
Studierende dazu, in weiteren Studienprojekten, der Abschlussarbeit oder
im Beruf eigene mobile Anwendungen zu realisieren. Um die Lernziele zu
erreichen werden Grundlagen und Konzepte in den Veranstaltungen studiert
und in Teams projektorientiert angewandt. Das Modul verfolgt einen
inkrementell-iterativen Ansatz von der Erstellung eines ersten
Prototypen, über die Implementierung des User Interface, der Auslagerung
von Operationen in den Hintergrund, der Speicherung von strukturierten
Daten und dem Management mobiler Software.

\hypertarget{lehrformswspathlabelmi-2017modulbeschreibungen-bachelorba_mobile-computing}{%
\section*{Lehrform/SWS\label{/mi-2017/modulbeschreibungen-bachelor/BA_Mobile-Computing}}\label{lehrformswspathlabelmi-2017modulbeschreibungen-bachelorba_mobile-computing}}

4 SWS: Vorlesung 2 SWS; Praktikum 2 SWS

\hypertarget{arbeitsaufwandpathlabelmi-2017modulbeschreibungen-bachelorba_mobile-computing}{%
\section*{Arbeitsaufwand\label{/mi-2017/modulbeschreibungen-bachelor/BA_Mobile-Computing}}\label{arbeitsaufwandpathlabelmi-2017modulbeschreibungen-bachelorba_mobile-computing}}

Gesamtaufwand 150h, davon

\begin{itemize}
\tightlist
\item
  26h Vorlesung
\item
  68h Praktikum und Projektarbeit
\item
  56h Selbststudium
\end{itemize}

\hypertarget{angestrebte-lernergebnissepathlabelmi-2017modulbeschreibungen-bachelorba_mobile-computing}{%
\section*{Angestrebte
Lernergebnisse\label{/mi-2017/modulbeschreibungen-bachelor/BA_Mobile-Computing}}\label{angestrebte-lernergebnissepathlabelmi-2017modulbeschreibungen-bachelorba_mobile-computing}}

Nach erfolgreicher Teilnahme am Modul können Studierende mobile
Anwendungen entwerfen und implementieren und dabei die Herausforderungen
von Mobilität bei der Gestaltung mobiler Medien sowie typische
nicht-funktionale Anforderungen an mobile Informationstechnologie
berücksichtigen, indem sie

\begin{itemize}
\tightlist
\item
  mobile Nutzungskontexte aus der Perspektive der
  Mensch-Computer-Interaktion analysieren,
\item
  mobiler Benutzungsschnittstellen daran angepasst gestalten,
\item
  Software für mobile Geräte in typischen Komponenten strukturieren und
  Architekturen entwerfen,
\item
  mobile Apps auf Basis aktueller Technologien, Frameworks und
  Entwicklungsumgebungen implementieren,
\item
  Nebenläufigkeit insbesondere vor dem Hintergrund interaktiver User
  Interfaces umsetzen,
\item
  Paradigmen für verteilte Architekturen als Basis für Kommunikation und
  Datenaustausch nutzen,
\item
  Herausforderungen hinsichtlich Kommunikation und Sicherheit kennen und
  adressieren,
\item
  sowie Mechanismen für das Deployment und Ansätze für die
  Monetarisierung im mobilen Ökosystemen nutzen.
\end{itemize}

Dies versetzt sie in die Lage, in weiteren Studienprojekten, der
Abschlussarbeit oder im Beruf mobile Anwendungen und Medien mit Blick
auf deren spezielle Nutzungskontexte zu konzipieren, zu entwerfen und zu
entwickeln.

\hypertarget{inhaltpathlabelmi-2017modulbeschreibungen-bachelorba_mobile-computing}{%
\section*{Inhalt\label{/mi-2017/modulbeschreibungen-bachelor/BA_Mobile-Computing}}\label{inhaltpathlabelmi-2017modulbeschreibungen-bachelorba_mobile-computing}}

App components \& architecture

\begin{itemize}
\tightlist
\item
  Types of apps
\item
  App components
\item
  Patterns MVC, MVP, MVVM
\item
  UI components
\item
  Declarative UIs
\item
  Navigation patterns
\item
  Background operations
\item
  Storage and databases
\item
  Concurrency and coroutines
\item
  Foreground and background processes
\end{itemize}

App Development

\begin{itemize}
\tightlist
\item
  Integrated development environment
\item
  Software development kit
\item
  Logging and debugging
\item
  Signing and versioning
\item
  Using libraries
\end{itemize}

Communication

\begin{itemize}
\tightlist
\item
  Networks and data transmission
\item
  Challenges and strategies
\item
  Req-res-based architecture
\item
  Event-based architecture
\item
  Security and compression
\item
  Backend requirements
\item
  Phone calls and SMS
\end{itemize}

Sensors \& gadgets

\begin{itemize}
\tightlist
\item
  Permissions and ethics
\item
  Sensors and sensor events
\item
  Using the observer pattern
\item
  Near field communication
\item
  Bluetooth and discovery
\item
  Speech input and output
\item
  Smartwatch and wearables
\end{itemize}

Management \& application usage

\begin{itemize}
\tightlist
\item
  Mobile human-computer-interaction
\item
  History and future
\item
  Mobile ecosystem
\item
  Deployment on app store
\item
  Feedback and tracking
\item
  Monetization
\item
  Smartphone lifecycle
\item
  Application lifecycle
\item
  Context-awareness
\item
  Usage patterns
\item
  Localization
\item
  Device management
\end{itemize}

\hypertarget{medienformenpathlabelmi-2017modulbeschreibungen-bachelorba_mobile-computing}{%
\section*{Medienformen\label{/mi-2017/modulbeschreibungen-bachelor/BA_Mobile-Computing}}\label{medienformenpathlabelmi-2017modulbeschreibungen-bachelorba_mobile-computing}}

Beamergestützte Vorträge, Rechnergestützte Workshops

\hypertarget{literaturpathlabelmi-2017modulbeschreibungen-bachelorba_mobile-computing}{%
\section*{Literatur\label{/mi-2017/modulbeschreibungen-bachelor/BA_Mobile-Computing}}\label{literaturpathlabelmi-2017modulbeschreibungen-bachelorba_mobile-computing}}

\begin{itemize}
\tightlist
\item
  Bollmann, Zeppenfeld: Mobile Computing. W3L Verlag, 2015.
\item
  Android Website: https://developer.android.com
\item
  Weitere Referenzen werden in Veranstaltung genannt
\end{itemize}

\hypertarget{paradigmen-der-programmierungpathlabelmi-2017modulbeschreibungen-bachelorba_paradigmen-der-programmierung}{%
\chapter{Paradigmen der
Programmierung\label{/mi-2017/modulbeschreibungen-bachelor/BA_Paradigmen-der-Programmierung}}\label{paradigmen-der-programmierungpathlabelmi-2017modulbeschreibungen-bachelorba_paradigmen-der-programmierung}}

\begin{modulHead}
\textbf{Modulverantwortlich}: Prof.~Dr.~Christian
Kohls
\end{modulHead}
\begin{modulHead}
\textbf{Studiensemester}:
3
\end{modulHead}
\begin{modulHead}
\textbf{Sprache}:
deutsch
\end{modulHead}
\begin{modulHead}
\textbf{Kreditpunkte}:
5
\end{modulHead}
\begin{modulHead}
\textbf{Typ}:
Pflichtmodul
\end{modulHead}
\begin{modulHead}
\textbf{Prüfungsleistung}:
Schriftliche Prüfung
\end{modulHead}


\hypertarget{lehrformswspathlabelmi-2017modulbeschreibungen-bachelorba_paradigmen-der-programmierung}{%
\section*{Lehrform/SWS\label{/mi-2017/modulbeschreibungen-bachelor/BA_Paradigmen-der-Programmierung}}\label{lehrformswspathlabelmi-2017modulbeschreibungen-bachelorba_paradigmen-der-programmierung}}

4 SWS: Vorlesung 2 SWS; Praktikum 1 SWS; Übung 1 SWS

\hypertarget{arbeitsaufwandpathlabelmi-2017modulbeschreibungen-bachelorba_paradigmen-der-programmierung}{%
\section*{Arbeitsaufwand\label{/mi-2017/modulbeschreibungen-bachelor/BA_Paradigmen-der-Programmierung}}\label{arbeitsaufwandpathlabelmi-2017modulbeschreibungen-bachelorba_paradigmen-der-programmierung}}

Gesamtaufwand 150h, davon

\begin{itemize}
\tightlist
\item
  36h Vorlesung
\item
  18h Praktikum
\item
  18h Übung
\item
  78h Selbststudium
\end{itemize}

\hypertarget{angestrebte-lernergebnissepathlabelmi-2017modulbeschreibungen-bachelorba_paradigmen-der-programmierung}{%
\section*{Angestrebte
Lernergebnisse\label{/mi-2017/modulbeschreibungen-bachelor/BA_Paradigmen-der-Programmierung}}\label{angestrebte-lernergebnissepathlabelmi-2017modulbeschreibungen-bachelorba_paradigmen-der-programmierung}}

Die Studierenden sollen unterschiedliche Programmierparadigmen verstehen
und anwenden können. Weiterhin sollen sie die Angemessenheit der
verschiedenen Programmierparadigmen für eine Aufgabenstellung einordnen
und bewerten können. Studierende sollen mithilfe von etablierten
Paradigmen und Entwurfsmustern in der Lage sein, synchrone und
asynchrone Programme zu konzipieren und ablaufsicher zu gestalten.

\hypertarget{inhaltpathlabelmi-2017modulbeschreibungen-bachelorba_paradigmen-der-programmierung}{%
\section*{Inhalt\label{/mi-2017/modulbeschreibungen-bachelor/BA_Paradigmen-der-Programmierung}}\label{inhaltpathlabelmi-2017modulbeschreibungen-bachelorba_paradigmen-der-programmierung}}

\begin{itemize}
\tightlist
\item
  Grundlagen von Programmiersprachen
\item
  Vergleich imperativer und deklarativer Paradigmen
\item
  prozedurale und objektorientierte Programmierung
\item
  funktionale Programmierung
\item
  Logikprogrammierung
\item
  Nebenläufigkeit
\item
  Entwurfsmuster
\end{itemize}

\hypertarget{medienformenpathlabelmi-2017modulbeschreibungen-bachelorba_paradigmen-der-programmierung}{%
\section*{Medienformen\label{/mi-2017/modulbeschreibungen-bachelor/BA_Paradigmen-der-Programmierung}}\label{medienformenpathlabelmi-2017modulbeschreibungen-bachelorba_paradigmen-der-programmierung}}

\begin{itemize}
\tightlist
\item
  Foliensammlung
\item
  Screencasts
\item
  Skript
\item
  Beispiellösungen
\end{itemize}

\hypertarget{literaturpathlabelmi-2017modulbeschreibungen-bachelorba_paradigmen-der-programmierung}{%
\section*{Literatur\label{/mi-2017/modulbeschreibungen-bachelor/BA_Paradigmen-der-Programmierung}}\label{literaturpathlabelmi-2017modulbeschreibungen-bachelorba_paradigmen-der-programmierung}}

\begin{itemize}
\tightlist
\item
  Abelson, Sussman, Struktur und Interpretation von Computer
  Programmen,Springer-Verlag 2001
\item
  W.F. Clocksin, C.S. Mellish, Programming in Prolog, Springer-Verlag
  2003
\item
  Gamma, E., Helm, R., Johnson, R., \& Vlissides, J. (2015). Design
  patterns: Entwurfsmuster als Elemente wiederverwendbarer
  objektorientierter Software. Frechen: Mitp.
\item
  Odersky, Spoon, Venners, Programming in Scala, Artima Press 2011
\item
  Goetz, B., Peierls, T., Bloch, J., Bowbeer, J., Holmes, D., Lea, D.
  (2006). Java-Concurrency in Practise. Addison Wesley.
\item
  Tate, B. A., \& Klicman, P. (2011). Sieben Wochen, sieben Sprachen:
  Verstehen Sie die modernen Sprachkonzepte. Sebastopol: O'Reilly.
\end{itemize}

\hypertarget{praxisprojektpathlabelmi-2017modulbeschreibungen-bachelorba_praxisprojekt}{%
\chapter{Praxisprojekt\label{/mi-2017/modulbeschreibungen-bachelor/BA_Praxisprojekt}}\label{praxisprojektpathlabelmi-2017modulbeschreibungen-bachelorba_praxisprojekt}}

\begin{modulHead}
\textbf{Modulverantwortlich}: alle Professor:innen
der Lehreinheit Informatik der
F10
\end{modulHead}
\begin{modulHead}
\textbf{Studiensemester}:
6
\end{modulHead}
\begin{modulHead}
\textbf{Sprache}:
deutsch
\end{modulHead}
\begin{modulHead}
\textbf{Kreditpunkte}:
15
\end{modulHead}
\begin{modulHead}
\textbf{Typ}:
Pflichtmodul
\end{modulHead}
\begin{modulHead}
\textbf{Prüfungsleistung}:
Schriftliche Ausarbeitung, Projektdokumentation
\end{modulHead}


\hypertarget{lehrformswspathlabelmi-2017modulbeschreibungen-bachelorba_praxisprojekt}{%
\section*{Lehrform/SWS\label{/mi-2017/modulbeschreibungen-bachelor/BA_Praxisprojekt}}\label{lehrformswspathlabelmi-2017modulbeschreibungen-bachelorba_praxisprojekt}}

Angeleitetes, eigenverantwortliches Arbeiten

\hypertarget{arbeitsaufwandpathlabelmi-2017modulbeschreibungen-bachelorba_praxisprojekt}{%
\section*{Arbeitsaufwand\label{/mi-2017/modulbeschreibungen-bachelor/BA_Praxisprojekt}}\label{arbeitsaufwandpathlabelmi-2017modulbeschreibungen-bachelorba_praxisprojekt}}

300 h Projektarbeit

\hypertarget{angestrebte-lernergebnissepathlabelmi-2017modulbeschreibungen-bachelorba_praxisprojekt}{%
\section*{Angestrebte
Lernergebnisse\label{/mi-2017/modulbeschreibungen-bachelor/BA_Praxisprojekt}}\label{angestrebte-lernergebnissepathlabelmi-2017modulbeschreibungen-bachelorba_praxisprojekt}}

Die Studierenden

\begin{itemize}
\tightlist
\item
  können Methoden und Techniken, die sie im Studium erlernt haben, in
  realitätsnahen Projekten weitgehend selbstständig anwenden
\item
  haben erste Erfahrungen mit der Selbststeuerung und proaktiven
  Kommunikation in einem Projekt mittlerer Größe und der Einordnung von
  Projektarbeit in betriebliche, gesellschaftliche und rechtliche
  Rahmenbedingungen gesammelt
\end{itemize}

\hypertarget{inhaltpathlabelmi-2017modulbeschreibungen-bachelorba_praxisprojekt}{%
\section*{Inhalt\label{/mi-2017/modulbeschreibungen-bachelor/BA_Praxisprojekt}}\label{inhaltpathlabelmi-2017modulbeschreibungen-bachelorba_praxisprojekt}}

Modulinhalte des ersten bis fünften Semesters anhand von realen
Anforderungen in einem praxisrelevanten Kontext anwenden und den
Studierenden durch die Betreuung des Dozenten an eine selbstständige
Projektdurchführung und Kommunikation heranführen. Das Praxisprojekt
kann entweder in einem Unternehmen oder in der Hochschule - dann
eingebettet in Forschungsprojekte - erfolgen.

\hypertarget{enthuxe4lt-folgende-teilmodulepathlabelmi-2017modulbeschreibungen-bachelorba_praxisprojekt}{%
\section*{Enthält folgende
Teilmodule:\label{/mi-2017/modulbeschreibungen-bachelor/BA_Praxisprojekt}}\label{enthuxe4lt-folgende-teilmodulepathlabelmi-2017modulbeschreibungen-bachelorba_praxisprojekt}}

\begin{itemize}
\tightlist
\item
  \hyperref[/mi-2017/modulbeschreibungen-bachelor/BA_Praxisprojektarbeit]{Praxisprojekt, praktischer Teil}
\item
  \hyperref[/mi-2017/modulbeschreibungen-bachelor/BA_Praxisprojektseminar]{Praxisprojekt, Seminarteil}
\end{itemize}

\hypertarget{praxisprojekt-praktischer-teilpathlabelmi-2017modulbeschreibungen-bachelorba_praxisprojektarbeit}{%
\chapter{Praxisprojekt, praktischer
Teil\label{/mi-2017/modulbeschreibungen-bachelor/BA_Praxisprojektarbeit}}\label{praxisprojekt-praktischer-teilpathlabelmi-2017modulbeschreibungen-bachelorba_praxisprojektarbeit}}

\begin{modulHead}
\textbf{Modulverantwortlich}: alle Professor:innen
der Lehreinheit Informatik der
F10
\end{modulHead}
\begin{modulHead}
\textbf{Studiensemester}:
6
\end{modulHead}
\begin{modulHead}
\textbf{Sprache}:
deutsch
\end{modulHead}
\begin{modulHead}
\textbf{Kreditpunkte}:
10.5
\end{modulHead}
\begin{modulHead}
\textbf{Typ}:
Teilmodul
\end{modulHead}
\begin{modulHead}
\textbf{Prüfungsleistung}:
Schriftliche Ausarbeitung, Projektdokumentation
\end{modulHead}


\hypertarget{lehrformswspathlabelmi-2017modulbeschreibungen-bachelorba_praxisprojektarbeit}{%
\section*{Lehrform/SWS\label{/mi-2017/modulbeschreibungen-bachelor/BA_Praxisprojektarbeit}}\label{lehrformswspathlabelmi-2017modulbeschreibungen-bachelorba_praxisprojektarbeit}}

Angeleitetes, eigenverantwortliches Arbeiten

\hypertarget{arbeitsaufwandpathlabelmi-2017modulbeschreibungen-bachelorba_praxisprojektarbeit}{%
\section*{Arbeitsaufwand\label{/mi-2017/modulbeschreibungen-bachelor/BA_Praxisprojektarbeit}}\label{arbeitsaufwandpathlabelmi-2017modulbeschreibungen-bachelorba_praxisprojektarbeit}}

315 h Projektarbeit

\hypertarget{angestrebte-lernergebnissepathlabelmi-2017modulbeschreibungen-bachelorba_praxisprojektarbeit}{%
\section*{Angestrebte
Lernergebnisse\label{/mi-2017/modulbeschreibungen-bachelor/BA_Praxisprojektarbeit}}\label{angestrebte-lernergebnissepathlabelmi-2017modulbeschreibungen-bachelorba_praxisprojektarbeit}}

Die Studierenden

\begin{itemize}
\tightlist
\item
  können Methoden und Techniken, die sie im Studium erlernt haben, in
  realitätsnahen Projekten weitgehend selbstständig anwenden
\item
  haben erste Erfahrungen mit der Selbststeuerung und proaktiven
  Kommunikation in einem Projekt mittlerer Größe und der Einordnung von
  Projektarbeit in betriebliche, gesellschaftliche und rechtliche
  Rahmenbedingungen gesammelt
\end{itemize}

\hypertarget{inhaltpathlabelmi-2017modulbeschreibungen-bachelorba_praxisprojektarbeit}{%
\section*{Inhalt\label{/mi-2017/modulbeschreibungen-bachelor/BA_Praxisprojektarbeit}}\label{inhaltpathlabelmi-2017modulbeschreibungen-bachelorba_praxisprojektarbeit}}

Modulinhalte des ersten bis fünften Semesters anhand von realen
Anforderungen in einem praxisrelevanten Kontext anwenden und den
Studierenden durch die Betreuung des Dozenten an eine selbstständige
Projektdurchführung und Kommunikation heranführen. Das Praxisprojekt
kann entweder in einem Unternehmen oder in der Hochschule - dann
eingebettet in Forschungsprojekte - erfolgen.

\hypertarget{teilmodul-vonpathlabelmi-2017modulbeschreibungen-bachelorba_praxisprojektarbeit}{%
\section*{Teilmodul
von:\label{/mi-2017/modulbeschreibungen-bachelor/BA_Praxisprojektarbeit}}\label{teilmodul-vonpathlabelmi-2017modulbeschreibungen-bachelorba_praxisprojektarbeit}}

\hyperref[/mi-2017/modulbeschreibungen-bachelor/BA_Praxisprojekt]{Praxisprojekt}

\hypertarget{praxisprojekt-seminarteilpathlabelmi-2017modulbeschreibungen-bachelorba_praxisprojektseminar}{%
\chapter{Praxisprojekt,
Seminarteil\label{/mi-2017/modulbeschreibungen-bachelor/BA_Praxisprojektseminar}}\label{praxisprojekt-seminarteilpathlabelmi-2017modulbeschreibungen-bachelorba_praxisprojektseminar}}

\begin{modulHead}
\textbf{Modulverantwortlich}: Prof.~Christian
Noss
\end{modulHead}
\begin{modulHead}
\textbf{Studiensemester}:
6
\end{modulHead}
\begin{modulHead}
\textbf{Sprache}:
deutsch
\end{modulHead}
\begin{modulHead}
\textbf{Kreditpunkte}:
4.5
\end{modulHead}
\begin{modulHead}
\textbf{Typ}:
Teilmodul
\end{modulHead}
\begin{modulHead}
\textbf{Prüfungsleistung}:
Seminarvortrag und Abstract zur Praxisprojektarbeit
\end{modulHead}


\hypertarget{lehrformswspathlabelmi-2017modulbeschreibungen-bachelorba_praxisprojektseminar}{%
\section*{Lehrform/SWS\label{/mi-2017/modulbeschreibungen-bachelor/BA_Praxisprojektseminar}}\label{lehrformswspathlabelmi-2017modulbeschreibungen-bachelorba_praxisprojektseminar}}

4 SWS: Seminar

\hypertarget{arbeitsaufwandpathlabelmi-2017modulbeschreibungen-bachelorba_praxisprojektseminar}{%
\section*{Arbeitsaufwand\label{/mi-2017/modulbeschreibungen-bachelor/BA_Praxisprojektseminar}}\label{arbeitsaufwandpathlabelmi-2017modulbeschreibungen-bachelorba_praxisprojektseminar}}

Gesamtaufwand 135 h, davon

\begin{itemize}
\tightlist
\item
  32 h Seminar
\item
  103 h Selbststudium
\end{itemize}

\hypertarget{angestrebte-lernergebnissepathlabelmi-2017modulbeschreibungen-bachelorba_praxisprojektseminar}{%
\section*{Angestrebte
Lernergebnisse\label{/mi-2017/modulbeschreibungen-bachelor/BA_Praxisprojektseminar}}\label{angestrebte-lernergebnissepathlabelmi-2017modulbeschreibungen-bachelorba_praxisprojektseminar}}

Die Studierenden

\begin{itemize}
\tightlist
\item
  kennen Techniken wissenschaftlichen Arbeitens und können diese
  anwenden
\item
  haben erste Erfahrungen mit aktiver Fachkommunikation gesammelt
\item
  gewinnen einen ersten Überblick über das Spektrum von aktuellen Themen
  in der Medieninformatik
\item
  können eigene Projektergebnisse vor einem Fachpublikum in Vortrag und
  Diskussion darstellen und verteidigen
\end{itemize}

\hypertarget{inhaltpathlabelmi-2017modulbeschreibungen-bachelorba_praxisprojektseminar}{%
\section*{Inhalt\label{/mi-2017/modulbeschreibungen-bachelor/BA_Praxisprojektseminar}}\label{inhaltpathlabelmi-2017modulbeschreibungen-bachelorba_praxisprojektseminar}}

Das Praxisprojektseminar besteht aus

\begin{itemize}
\tightlist
\item
  Veranstaltungen in denen Techniken wissenschaftlichen Arbeitens
  vermittelt werden,
\item
  Audits über den aktuellen Stand ihres Projektes,
\item
  Fachvorträgen von Studierenden über ihre Projektergebnisse.
\end{itemize}

\hypertarget{literaturpathlabelmi-2017modulbeschreibungen-bachelorba_praxisprojektseminar}{%
\section*{Literatur\label{/mi-2017/modulbeschreibungen-bachelor/BA_Praxisprojektseminar}}\label{literaturpathlabelmi-2017modulbeschreibungen-bachelorba_praxisprojektseminar}}

\begin{itemize}
\tightlist
\item
  M. Karmasin, R. Ribing: Die Gestaltung wissenschaftlicher Arbeiten,
  10. überarbeitete und aktualisierte Auflage. - Wien: Facultas: 2019
\end{itemize}

\hypertarget{teilmodul-vonpathlabelmi-2017modulbeschreibungen-bachelorba_praxisprojektseminar}{%
\section*{Teilmodul
von:\label{/mi-2017/modulbeschreibungen-bachelor/BA_Praxisprojektseminar}}\label{teilmodul-vonpathlabelmi-2017modulbeschreibungen-bachelorba_praxisprojektseminar}}

\hyperref[/mi-2017/modulbeschreibungen-bachelor/BA_Praxisprojekt]{Praxisprojekt}

\hypertarget{projektmanagementpathlabelmi-2017modulbeschreibungen-bachelorba_projektmanagement}{%
\chapter{Projektmanagement\label{/mi-2017/modulbeschreibungen-bachelor/BA_Projektmanagement}}\label{projektmanagementpathlabelmi-2017modulbeschreibungen-bachelorba_projektmanagement}}

\begin{modulHead}
\textbf{Modulverantwortlich}: Prof.~Dr.~Holger
Günther, Prof.~Dr.~Mario
Winter
\end{modulHead}
\begin{modulHead}
\textbf{Studiensemester}:
5
\end{modulHead}
\begin{modulHead}
\textbf{Sprache}:
deutsch
\end{modulHead}
\begin{modulHead}
\textbf{Kreditpunkte}:
5
\end{modulHead}
\begin{modulHead}
\textbf{Typ}:
Pflichtmodul
\end{modulHead}
\begin{modulHead}
\textbf{Prüfungsleistung}:
Projekt-Ausarbeitung (30\%), Vortrag (30\%), Schriftliche Prüfung
(40\%)
\end{modulHead}


\hypertarget{kurzbeschreibungpathlabelmi-2017modulbeschreibungen-bachelorba_projektmanagement}{%
\section*{Kurzbeschreibung\label{/mi-2017/modulbeschreibungen-bachelor/BA_Projektmanagement}}\label{kurzbeschreibungpathlabelmi-2017modulbeschreibungen-bachelorba_projektmanagement}}

Managementaspekte der professionellen Entwicklung großer Softwaresysteme

\hypertarget{lehrformswspathlabelmi-2017modulbeschreibungen-bachelorba_projektmanagement}{%
\section*{Lehrform/SWS\label{/mi-2017/modulbeschreibungen-bachelor/BA_Projektmanagement}}\label{lehrformswspathlabelmi-2017modulbeschreibungen-bachelorba_projektmanagement}}

4 SWS: Vorlesung 2 SWS, Übung 1 SWS, Praktikum 1 SWS; max. 6 Studierende
pro Praktikumsteam;

\hypertarget{arbeitsaufwandpathlabelmi-2017modulbeschreibungen-bachelorba_projektmanagement}{%
\section*{Arbeitsaufwand\label{/mi-2017/modulbeschreibungen-bachelor/BA_Projektmanagement}}\label{arbeitsaufwandpathlabelmi-2017modulbeschreibungen-bachelorba_projektmanagement}}

Gesamtaufwand 150h, davon

\begin{itemize}
\tightlist
\item
  36h Vorlesung
\item
  18h Übung
\item
  18h Praktikum
\item
  78h Selbststudium
\end{itemize}

\hypertarget{angestrebte-lernergebnissepathlabelmi-2017modulbeschreibungen-bachelorba_projektmanagement}{%
\section*{Angestrebte
Lernergebnisse\label{/mi-2017/modulbeschreibungen-bachelor/BA_Projektmanagement}}\label{angestrebte-lernergebnissepathlabelmi-2017modulbeschreibungen-bachelorba_projektmanagement}}

Die Studierenden sollen befähigt werden,

\begin{itemize}
\tightlist
\item
  die grundlegenden Aufgaben des Projektmanagements, insbesondere in
  IT-Projekten, zu charakterisieren und durchzuführen
\item
  die Projektmanagement-Methoden, -Techniken und -Werkzeuge
  zielgerichtet einzusetzen
\item
  die erforderlichen soziologischen und kommunikativen Aspekte zu
  berücksichtigen, um, mit dem Ziel einer menschengerechten und
  soziologisch fundierten Menschenführung, eine wirkliche und optimale
  Produktivität bei komplexen Projekten erreichen zu können.
\end{itemize}

\hypertarget{inhaltpathlabelmi-2017modulbeschreibungen-bachelorba_projektmanagement}{%
\section*{Inhalt\label{/mi-2017/modulbeschreibungen-bachelor/BA_Projektmanagement}}\label{inhaltpathlabelmi-2017modulbeschreibungen-bachelorba_projektmanagement}}

Das Modul befasst sich mit den Managementaspekten der professionellen
Entwicklung großer Softwaresysteme.

Der Vorlesungsteil des Moduls gliedert sich in folgende Kapitel:

\begin{itemize}
\tightlist
\item
  Überblick -- Warum Projektmanagement?
\item
  Teamarbeit und Menschenführung (Kommunikation und Führung)
\item
  Kosten/Nutzen-Analysen und Entscheidungstechniken
\item
  Projektorganisation und Projektplanung (Aufbauorganisation,
  Ablauforganisation, Prozessmodellierung, iterative und agile
  Vorgehensmodelle, Netzplantechnik)
\item
  detaillierte Aufwandsschätzung und Projektcontrolling (Function Point
  Analysis, COCOMO, Risikomanagement, Projektpräsentationen)
\item
  Inhalte von PM-BOK (Project Management - Body of Knowledge)
\item
  Zusammenfassung und Prüfungsvorbereitung;
\end{itemize}

Damit die Studierenden die vorgestellten Methoden und Techniken zum
Management von Softwareprojekten anwenden, sowie besser analysieren und
bewerten können, werden in Projekt-Teams die in der Vorlesung
vermittelten Inhalte anhand eines Fallbeispiels eingesetzt. Dazu bilden
die Teilnehmenden Teams zu jeweils 6 Studierenden. Im Projekt werden
folgende Bereiche vertieft:

\begin{itemize}
\tightlist
\item
  Kosten- Nutzenrechnung, Entscheidungstechniken
\item
  Aufbauorganisation
\item
  Aufwandsschätzung (Function-Point-Analyse, COCOMO);
\item
  Risikomanagement
\item
  Ablauf- und Ressourcenplanung (Netzplantechnik, Einsatz von
  PM-Software wie z.B. MS-Project)
\end{itemize}

\hypertarget{medienformenpathlabelmi-2017modulbeschreibungen-bachelorba_projektmanagement}{%
\section*{Medienformen\label{/mi-2017/modulbeschreibungen-bachelor/BA_Projektmanagement}}\label{medienformenpathlabelmi-2017modulbeschreibungen-bachelorba_projektmanagement}}

\begin{itemize}
\tightlist
\item
  Beamer-gestützte Vorlesungen (Folien in elektronischer Form im Netz);
\item
  Vertiefende Unterlagen sowie aktuelle Artikel (in elektronischer Form
  im Netz);
\item
  Projektarbeit in Kleingruppen, um die erlernten Methoden und Techniken
  einzuüben und zu vertiefen (Seminarraum, Rechnerlabor);
\end{itemize}

\hypertarget{literaturpathlabelmi-2017modulbeschreibungen-bachelorba_projektmanagement}{%
\section*{Literatur\label{/mi-2017/modulbeschreibungen-bachelor/BA_Projektmanagement}}\label{literaturpathlabelmi-2017modulbeschreibungen-bachelorba_projektmanagement}}

\begin{itemize}
\tightlist
\item
  A. Buhl: Grundkurs Projektmanagement. Carl Hanser Verlag, München,
  2004
\item
  H.W. Wieczorrek, P. Mertens: Management von IT-Projekten Von der
  Planung zur Realisierung. 4. Aufl., Springer, Heidelberg, 2011
\item
  C. Aichele, M. Schönberger: IT-Projektmanagement. Springer Vieweg,
  2014
\item
  A. Henrich: Management von Softwareprojekten. Oldenbourg Verlag,
  München, 2002
\item
  H. Kerzner: Projektmanagement -- Ein systemorientierter Ansatz.
  mitp-Verlag, Bonn, 2003
\item
  T. DeMarco: Der Termin. Hanser, München, 1998
\end{itemize}

\hypertarget{social-computing-projektpathlabelmi-2017modulbeschreibungen-bachelorba_sc_projekt}{%
\chapter{Social Computing
Projekt\label{/mi-2017/modulbeschreibungen-bachelor/BA_SC_Projekt}}\label{social-computing-projektpathlabelmi-2017modulbeschreibungen-bachelorba_sc_projekt}}

\begin{modulHead}
\textbf{Modulverantwortlich}: Prof.~Dr.~Christian
Kohls, Prof.~Dr.~Mirjam Blümm, Uwe
Müsse
\end{modulHead}
\begin{modulHead}
\textbf{Studiensemester}:
4
\end{modulHead}
\begin{modulHead}
\textbf{Sprache}:
deutsch
\end{modulHead}
\begin{modulHead}
\textbf{Kreditpunkte}:
5
\end{modulHead}
\begin{modulHead}
\textbf{Typ}:
Teilmodul
\end{modulHead}
\begin{modulHead}
\textbf{Prüfungsleistung}:
Projektarbeit mit Projektpräsentationsprüfung und Fachgespräch, sowie
schriftliche Ausarbeitung.
\end{modulHead}


\hypertarget{arbeitsaufwandpathlabelmi-2017modulbeschreibungen-bachelorba_sc_projekt}{%
\section*{Arbeitsaufwand\label{/mi-2017/modulbeschreibungen-bachelor/BA_SC_Projekt}}\label{arbeitsaufwandpathlabelmi-2017modulbeschreibungen-bachelorba_sc_projekt}}

150h Projekt

\hypertarget{angestrebte-lernergebnissepathlabelmi-2017modulbeschreibungen-bachelorba_sc_projekt}{%
\section*{Angestrebte
Lernergebnisse\label{/mi-2017/modulbeschreibungen-bachelor/BA_SC_Projekt}}\label{angestrebte-lernergebnissepathlabelmi-2017modulbeschreibungen-bachelorba_sc_projekt}}

Studierende sollen in der Lage sein, computergestützte Systeme nach
ethischen, politischen, sozialen und psychologischen Kriterien zu
bewerten, zu planen und umsetzen zu können.

Ziel ist es, soziale Innovation durch digitale Anwendungen entstehen zu
lassen. Neben den empirischen Methoden werden Designmethoden vermittelt,
sowohl auf der konzeptionellen als auch auf der softwaretechnischen
Implementierungsebene, um robuste, sichere und flexible Systeme zu
gestalten.

\hypertarget{teilmodul-vonpathlabelmi-2017modulbeschreibungen-bachelorba_sc_projekt}{%
\section*{Teilmodul
von:\label{/mi-2017/modulbeschreibungen-bachelor/BA_SC_Projekt}}\label{teilmodul-vonpathlabelmi-2017modulbeschreibungen-bachelorba_sc_projekt}}

\hyperref[/mi-2017/modulbeschreibungen-bachelor/BA_Vertiefung_SocialComputing]{Vertiefung – Social Computing}

\hypertarget{empirische-forschungsmethodenpathlabelmi-2017modulbeschreibungen-bachelorba_sc_empirische-forschungsmethoden}{%
\chapter{Empirische
Forschungsmethoden\label{/mi-2017/modulbeschreibungen-bachelor/BA_SC_empirische-forschungsmethoden}}\label{empirische-forschungsmethodenpathlabelmi-2017modulbeschreibungen-bachelorba_sc_empirische-forschungsmethoden}}

\begin{modulHead}
\textbf{Modulverantwortlich}: Prof.~Dr.~Mirjam
Blümm
\end{modulHead}
\begin{modulHead}
\textbf{Studiensemester}:
4
\end{modulHead}
\begin{modulHead}
\textbf{Sprache}:
deutsch
\end{modulHead}
\begin{modulHead}
\textbf{Kreditpunkte}:
5
\end{modulHead}
\begin{modulHead}
\textbf{Typ}:
Teilmodul
\end{modulHead}
\begin{modulHead}
\textbf{Prüfungsleistung}:
Lernportfolio und Projekt- und
Projektpräsentationsprüfung
\end{modulHead}


\hypertarget{lehrformswspathlabelmi-2017modulbeschreibungen-bachelorba_sc_empirische-forschungsmethoden}{%
\section*{Lehrform/SWS\label{/mi-2017/modulbeschreibungen-bachelor/BA_SC_empirische-forschungsmethoden}}\label{lehrformswspathlabelmi-2017modulbeschreibungen-bachelorba_sc_empirische-forschungsmethoden}}

50h Vorlesung, Seminar; 100h Selbstlernphase

\hypertarget{angestrebte-lernergebnissepathlabelmi-2017modulbeschreibungen-bachelorba_sc_empirische-forschungsmethoden}{%
\section*{Angestrebte
Lernergebnisse\label{/mi-2017/modulbeschreibungen-bachelor/BA_SC_empirische-forschungsmethoden}}\label{angestrebte-lernergebnissepathlabelmi-2017modulbeschreibungen-bachelorba_sc_empirische-forschungsmethoden}}

Die Studierenden sollen die unterschiedlichen Herangehensweisen
quantitativer und qualitativer Forschungsmethoden verstehen. Darüber
hinaus sollen ausgewählte quantitative und qualitative Methoden
angewendet werden können. Die Studierenden sollen in der Lage sein,
einfache Forschungsdesigns zu entwickeln und nach wissenschaftlichen
Standards durchzuführen.

\hypertarget{inhaltpathlabelmi-2017modulbeschreibungen-bachelorba_sc_empirische-forschungsmethoden}{%
\section*{Inhalt\label{/mi-2017/modulbeschreibungen-bachelor/BA_SC_empirische-forschungsmethoden}}\label{inhaltpathlabelmi-2017modulbeschreibungen-bachelorba_sc_empirische-forschungsmethoden}}

\begin{itemize}
\tightlist
\item
  Wissenschaftstheoretische Grundlagen
\item
  Induktion, Deduktion
\item
  Unterschied zwischen quantitativer und qualitativer Forschung
\item
  Interviews gestalten, durchführen und auswerten
\item
  Beobachtungsmethoden
\item
  Ethnographische Methoden
\item
  Hypothesengewinnung und Theoriebildung
\item
  Statistische Verfahren für quantitative Forschung
\item
  Aussagekraft der Ergebnisse (statistische Signifikanz, interne und
  externe Valididät)
\end{itemize}

\hypertarget{medienformenpathlabelmi-2017modulbeschreibungen-bachelorba_sc_empirische-forschungsmethoden}{%
\section*{Medienformen\label{/mi-2017/modulbeschreibungen-bachelor/BA_SC_empirische-forschungsmethoden}}\label{medienformenpathlabelmi-2017modulbeschreibungen-bachelorba_sc_empirische-forschungsmethoden}}

\begin{itemize}
\tightlist
\item
  Beamer-gestützte Vorlesungen (Folien in elektronischer Form)
\item
  Screencasts und Handouts
\item
  Beispielmedien
\end{itemize}

\hypertarget{literaturpathlabelmi-2017modulbeschreibungen-bachelorba_sc_empirische-forschungsmethoden}{%
\section*{Literatur\label{/mi-2017/modulbeschreibungen-bachelor/BA_SC_empirische-forschungsmethoden}}\label{literaturpathlabelmi-2017modulbeschreibungen-bachelorba_sc_empirische-forschungsmethoden}}

\begin{itemize}
\tightlist
\item
  DeKoven, B., \& MIT Press. (2013). The well-played game: A player's
  philosophy. Cambridge: The MIT Press.
\item
  Döring, N. \& Bortz, J. (2015). Forschungsmethoden und Evaluation: Für
  Human- und Sozialwissenschaftler. Berlin {[}u.a.{]}: Springer.
\item
  Flick, U. (2011). Qualitative Sozialforschung: Eine Einführung.
  Reinbek bei Hamburg: Rowohlt-Taschenbuch-Verl.
\item
  Fullerton, T., Swain, C., \& Hoffman, S. (2008). Game design workshop:
  A playcentric approach to creating innovative games. Amsterdam:
  Elsevier Morgan Kaufmann.
\item
  Kienle, A., Kunau, G. (2014). Informatik und Gesellschaft. Eine
  sozio-technische Perspektive. München: Oldenbourg Wissenschaftsverlag.
\item
  Koster, R. (2013). Theory of Fun for Game Design. Sebastopol:
  O'Reilly.
\item
  Popper, K. R. (1972). The logic of scientific discovery. London:
  Hutchinson.
\item
  Salen, K., \& Zimmerman, E. (2007). Rules of play: Game design
  fundamentals. Cambridge, Mass. {[}u.a.: The MIT Press.
\item
  Schnädelbach, H. (2002). Erkenntnistheorie zur Einführung. Zur
  Einführung, 268. Hamburg: Junius.
\item
  Westermann, R. (2000). Wissenschaftstheorie und Experimentalmethodik:
  Ein Lehrbuch zur psychologischen Methodenlehre. Göttingen {[}u.a.{]}:
  Hogrefe, Verl. für Psychologie.
\item
  Zweig, K. A., In Neuser, W., In Pipek, V., In Rohde, M., \& In
  Scholtes, I. (2014). Socioinformatics: The social impact of
  interactions between humans and IT.
\end{itemize}

\hypertarget{teilmodul-vonpathlabelmi-2017modulbeschreibungen-bachelorba_sc_empirische-forschungsmethoden}{%
\section*{Teilmodul
von:\label{/mi-2017/modulbeschreibungen-bachelor/BA_SC_empirische-forschungsmethoden}}\label{teilmodul-vonpathlabelmi-2017modulbeschreibungen-bachelorba_sc_empirische-forschungsmethoden}}

\hyperref[/mi-2017/modulbeschreibungen-bachelor/BA_Vertiefung_SocialComputing]{Vertiefung – Social Computing}

\hypertarget{gamificationpathlabelmi-2017modulbeschreibungen-bachelorba_sc_gamification}{%
\chapter{Gamification\label{/mi-2017/modulbeschreibungen-bachelor/BA_SC_gamification}}\label{gamificationpathlabelmi-2017modulbeschreibungen-bachelorba_sc_gamification}}

\begin{modulHead}
\textbf{Modulverantwortlich}: Uwe
Müsse
\end{modulHead}
\begin{modulHead}
\textbf{Studiensemester}:
4
\end{modulHead}
\begin{modulHead}
\textbf{Sprache}:
deutsch
\end{modulHead}
\begin{modulHead}
\textbf{Kreditpunkte}:
5
\end{modulHead}
\begin{modulHead}
\textbf{Typ}:
Teilmodul
\end{modulHead}
\begin{modulHead}
\textbf{Prüfungsleistung}:
Projektarbeit (mit Präsentation) und Projekt- und
Projektpräsentationsprüfung
\end{modulHead}


\hypertarget{lehrformswspathlabelmi-2017modulbeschreibungen-bachelorba_sc_gamification}{%
\section*{Lehrform/SWS\label{/mi-2017/modulbeschreibungen-bachelor/BA_SC_gamification}}\label{lehrformswspathlabelmi-2017modulbeschreibungen-bachelorba_sc_gamification}}

50h Vorlesung, Seminar; 100h Selbstlernphase

\hypertarget{angestrebte-lernergebnissepathlabelmi-2017modulbeschreibungen-bachelorba_sc_gamification}{%
\section*{Angestrebte
Lernergebnisse\label{/mi-2017/modulbeschreibungen-bachelor/BA_SC_gamification}}\label{angestrebte-lernergebnissepathlabelmi-2017modulbeschreibungen-bachelorba_sc_gamification}}

Die Studierenden sollen in der Lage sein, die Möglichkeiten und Grenzen
des Gamification-Ansatzes, sowohl für die analoge als auch die digitale
Welt, einordnen zu können. Die verschiedenen Stufen der Gamification
sollen verstanden und die Maßnahmen in Gestaltungsprozessen eingesetzt
werden. Die Studierenden sollen die psychologischen Grundlagen verstehen
und die gesellschaftlichen Implikationen bewerten können. Die Analyse
von Regeln und Prozessen und daraus abgeleitete Gamification-Maßnahmen
sowie die Entwicklung von Serious Games sollen praktisch umgesetzt
werden, z.B. für Lernanwendungen, Online-Communities oder soziale
Dienste.

\hypertarget{inhaltpathlabelmi-2017modulbeschreibungen-bachelorba_sc_gamification}{%
\section*{Inhalt\label{/mi-2017/modulbeschreibungen-bachelor/BA_SC_gamification}}\label{inhaltpathlabelmi-2017modulbeschreibungen-bachelorba_sc_gamification}}

\begin{itemize}
\tightlist
\item
  Grundelemente der Gamification
\item
  Stufen der Gamification
\item
  Ludifikation
\item
  Historische Grundlagen
\item
  Psychologische Grundlagen
\item
  Gesellschaftliche Einordnung
\item
  Einsatzgebiete verstehen und einordnen
\item
  Planung und Realisierung von Gamification
\item
  Gestaltungregeln
\item
  Serious Games
\end{itemize}

\hypertarget{medienformenpathlabelmi-2017modulbeschreibungen-bachelorba_sc_gamification}{%
\section*{Medienformen\label{/mi-2017/modulbeschreibungen-bachelor/BA_SC_gamification}}\label{medienformenpathlabelmi-2017modulbeschreibungen-bachelorba_sc_gamification}}

\begin{itemize}
\tightlist
\item
  Beamer-gestützte Vorlesungen (Folien in elektronischer Form)
\item
  Screencasts und Handouts
\item
  Beispielmedien
\item
  Arbeit im Innovationsraum mit digitalen Whiteboards, Spiel-Arcarde,
  Tablets und Gestaltungsmaterialien
\end{itemize}

\hypertarget{literaturpathlabelmi-2017modulbeschreibungen-bachelorba_sc_gamification}{%
\section*{Literatur\label{/mi-2017/modulbeschreibungen-bachelor/BA_SC_gamification}}\label{literaturpathlabelmi-2017modulbeschreibungen-bachelorba_sc_gamification}}

\begin{itemize}
\tightlist
\item
  DeKoven, B., \& MIT Press. (2013). The well-played game: A player's
  philosophy. Cambridge: The MIT Press.
\item
  Döring, N. \& Bortz, J. (2015). Forschungsmethoden und Evaluation: Für
  Human- und Sozialwissenschaftler. Berlin {[}u.a.{]}: Springer.
\item
  Flick, U. (2011). Qualitative Sozialforschung: Eine Einführung.
  Reinbek bei Hamburg: Rowohlt-Taschenbuch-Verl.
\item
  Fullerton, T., Swain, C., \& Hoffman, S. (2008). Game design workshop:
  A playcentric approach to creating innovative games. Amsterdam:
  Elsevier Morgan Kaufmann.
\item
  Kienle, A., Kunau, G. (2014). Informatik und Gesellschaft. Eine
  sozio-technische Perspektive. München: Oldenbourg Wissenschaftsverlag.
\item
  Koster, R. (2013). Theory of Fun for Game Design. Sebastopol:
  O'Reilly.
\item
  Popper, K. R. (1972). The logic of scientific discovery. London:
  Hutchinson.
\item
  Salen, K., \& Zimmerman, E. (2007). Rules of play: Game design
  fundamentals. Cambridge, Mass. {[}u.a.: The MIT Press.
\item
  Schnädelbach, H. (2002). Erkenntnistheorie zur Einführung. Zur
  Einführung, 268. Hamburg: Junius.
\item
  Westermann, R. (2000). Wissenschaftstheorie und Experimentalmethodik:
  Ein Lehrbuch zur psychologischen Methodenlehre. Göttingen {[}u.a.{]}:
  Hogrefe, Verl. für Psychologie.
\item
  Zweig, K. A., In Neuser, W., In Pipek, V., In Rohde, M., \& In
  Scholtes, I. (2014). Socioinformatics: The social impact of
  interactions between humans and IT.
\end{itemize}

\hypertarget{teilmodul-vonpathlabelmi-2017modulbeschreibungen-bachelorba_sc_gamification}{%
\section*{Teilmodul
von:\label{/mi-2017/modulbeschreibungen-bachelor/BA_SC_gamification}}\label{teilmodul-vonpathlabelmi-2017modulbeschreibungen-bachelorba_sc_gamification}}

\hyperref[/mi-2017/modulbeschreibungen-bachelor/BA_Vertiefung_SocialComputing]{Vertiefung – Social Computing}

\hypertarget{soziotechnische-systemepathlabelmi-2017modulbeschreibungen-bachelorba_sc_soziotechnische-systeme}{%
\chapter{Soziotechnische
Systeme\label{/mi-2017/modulbeschreibungen-bachelor/BA_SC_soziotechnische-systeme}}\label{soziotechnische-systemepathlabelmi-2017modulbeschreibungen-bachelorba_sc_soziotechnische-systeme}}

\begin{modulHead}
\textbf{Modulverantwortlich}: Prof.~Dr.~Christian
Kohls
\end{modulHead}
\begin{modulHead}
\textbf{Studiensemester}:
4
\end{modulHead}
\begin{modulHead}
\textbf{Sprache}:
deutsch
\end{modulHead}
\begin{modulHead}
\textbf{Kreditpunkte}:
5
\end{modulHead}
\begin{modulHead}
\textbf{Typ}:
Teilmodul
\end{modulHead}
\begin{modulHead}
\textbf{Prüfungsleistung}:
Lernportfolio und Projekt- und
Projektpräsentationsprüfung
\end{modulHead}


\hypertarget{lehrformswspathlabelmi-2017modulbeschreibungen-bachelorba_sc_soziotechnische-systeme}{%
\section*{Lehrform/SWS\label{/mi-2017/modulbeschreibungen-bachelor/BA_SC_soziotechnische-systeme}}\label{lehrformswspathlabelmi-2017modulbeschreibungen-bachelorba_sc_soziotechnische-systeme}}

50h Vorlesung, Seminar; 100h Selbstlernphase

\hypertarget{angestrebte-lernergebnissepathlabelmi-2017modulbeschreibungen-bachelorba_sc_soziotechnische-systeme}{%
\section*{Angestrebte
Lernergebnisse\label{/mi-2017/modulbeschreibungen-bachelor/BA_SC_soziotechnische-systeme}}\label{angestrebte-lernergebnissepathlabelmi-2017modulbeschreibungen-bachelorba_sc_soziotechnische-systeme}}

Die Studierenden sollen das komplexe Wechselspiel zwischen
informationstechnischen Systemen und gesellschaftlichen Normen und
Prozessen verstehen, analysieren und einordnen können. Sie sollen in der
Lage sein, Systeme nach ethischen, psychologischen und soziologischen
Kriterien zu gestalten und die Auswirkungen einschätzen zu können. Die
grundlegenden Modelle der verschiedenen Disziplinen sollen bekannt und
verstanden werden.

\hypertarget{inhaltpathlabelmi-2017modulbeschreibungen-bachelorba_sc_soziotechnische-systeme}{%
\section*{Inhalt\label{/mi-2017/modulbeschreibungen-bachelor/BA_SC_soziotechnische-systeme}}\label{inhaltpathlabelmi-2017modulbeschreibungen-bachelorba_sc_soziotechnische-systeme}}

\begin{itemize}
\tightlist
\item
  Modelle der Sozioinformatik
\item
  E-Learning
\item
  Gestaltungsprinzipien für soziotechnische Systeme
\item
  Technikgenese und Ko-Evolution
\item
  Computerethische Grundlagen
\item
  Psychologische Grundlagen
\item
  Soziologische Grundlagen
\item
  Digitale Technologien für soziale Dienste
\item
  Digitale Technologien in Organisationen
\item
  Digitale Technologien in der Gesellschaft
\item
  E-Goverment
\end{itemize}

\hypertarget{medienformenpathlabelmi-2017modulbeschreibungen-bachelorba_sc_soziotechnische-systeme}{%
\section*{Medienformen\label{/mi-2017/modulbeschreibungen-bachelor/BA_SC_soziotechnische-systeme}}\label{medienformenpathlabelmi-2017modulbeschreibungen-bachelorba_sc_soziotechnische-systeme}}

\begin{itemize}
\tightlist
\item
  Beamer-gestützte Vorlesungen (Folien in elektronischer Form)
\item
  Screencasts und Handouts
\item
  Beispielmedien
\item
  Arbeit im Innovationsraum mit digitalen Whiteboards, Spiel-Arcarde,
  Tablets und Gestaltungsmaterialien
\end{itemize}

\hypertarget{literaturpathlabelmi-2017modulbeschreibungen-bachelorba_sc_soziotechnische-systeme}{%
\section*{Literatur\label{/mi-2017/modulbeschreibungen-bachelor/BA_SC_soziotechnische-systeme}}\label{literaturpathlabelmi-2017modulbeschreibungen-bachelorba_sc_soziotechnische-systeme}}

\begin{itemize}
\tightlist
\item
  DeKoven, B., \& MIT Press. (2013). The well-played game: A player's
  philosophy. Cambridge: The MIT Press.
\item
  Döring, N. \& Bortz, J. (2015). Forschungsmethoden und Evaluation: Für
  Human- und Sozialwissenschaftler. Berlin {[}u.a.{]}: Springer.
\item
  Flick, U. (2011). Qualitative Sozialforschung: Eine Einführung.
  Reinbek bei Hamburg: Rowohlt-Taschenbuch-Verl.
\item
  Fullerton, T., Swain, C., \& Hoffman, S. (2008). Game design workshop:
  A playcentric approach to creating innovative games. Amsterdam:
  Elsevier Morgan Kaufmann.
\item
  Kienle, A., Kunau, G. (2014). Informatik und Gesellschaft. Eine
  sozio-technische Perspektive. München: Oldenbourg Wissenschaftsverlag.
\item
  Koster, R. (2013). Theory of Fun for Game Design. Sebastopol:
  O'Reilly.
\item
  Popper, K. R. (1972). The logic of scientific discovery. London:
  Hutchinson.
\item
  Salen, K., \& Zimmerman, E. (2007). Rules of play: Game design
  fundamentals. Cambridge, Mass. {[}u.a.: The MIT Press.
\item
  Schnädelbach, H. (2002). Erkenntnistheorie zur Einführung. Zur
  Einführung, 268. Hamburg: Junius.
\item
  Westermann, R. (2000). Wissenschaftstheorie und Experimentalmethodik:
  Ein Lehrbuch zur psychologischen Methodenlehre. Göttingen {[}u.a.{]}:
  Hogrefe, Verl. für Psychologie.
\item
  Zweig, K. A., In Neuser, W., In Pipek, V., In Rohde, M., \& In
  Scholtes, I. (2014). Socioinformatics: The social impact of
  interactions between humans and IT.
\end{itemize}

\hypertarget{teilmodul-vonpathlabelmi-2017modulbeschreibungen-bachelorba_sc_soziotechnische-systeme}{%
\section*{Teilmodul
von:\label{/mi-2017/modulbeschreibungen-bachelor/BA_SC_soziotechnische-systeme}}\label{teilmodul-vonpathlabelmi-2017modulbeschreibungen-bachelorba_sc_soziotechnische-systeme}}

\hyperref[/mi-2017/modulbeschreibungen-bachelor/BA_Vertiefung_SocialComputing]{Vertiefung – Social Computing}

\hypertarget{screendesignpathlabelmi-2017modulbeschreibungen-bachelorba_screendesign}{%
\chapter{Screendesign\label{/mi-2017/modulbeschreibungen-bachelor/BA_Screendesign}}\label{screendesignpathlabelmi-2017modulbeschreibungen-bachelorba_screendesign}}

\begin{modulHead}
\textbf{Modulverantwortlich}: Prof.~Christian
Noss
\end{modulHead}
\begin{modulHead}
\textbf{Studiensemester}:
3
\end{modulHead}
\begin{modulHead}
\textbf{Sprache}:
deutsch
\end{modulHead}
\begin{modulHead}
\textbf{Kreditpunkte}:
5
\end{modulHead}
\begin{modulHead}
\textbf{Typ}:
Pflichtmodul
\end{modulHead}
\begin{modulHead}
\textbf{Prüfungsleistung}:
Gestaltungsportfolio und Projekt- und
Projektpräsentationsprüfung
\end{modulHead}


\hypertarget{lehrformswspathlabelmi-2017modulbeschreibungen-bachelorba_screendesign}{%
\section*{Lehrform/SWS\label{/mi-2017/modulbeschreibungen-bachelor/BA_Screendesign}}\label{lehrformswspathlabelmi-2017modulbeschreibungen-bachelorba_screendesign}}

4 SWS: Vorlesung 1 SWS; Seminar 3 SWS

\hypertarget{arbeitsaufwandpathlabelmi-2017modulbeschreibungen-bachelorba_screendesign}{%
\section*{Arbeitsaufwand\label{/mi-2017/modulbeschreibungen-bachelor/BA_Screendesign}}\label{arbeitsaufwandpathlabelmi-2017modulbeschreibungen-bachelorba_screendesign}}

Gesamtaufwand 150h, davon

\begin{itemize}
\tightlist
\item
  15h Vorlesung
\item
  45h Seminar
\item
  80h Projektarbeit
\item
  10h Selbststudium
\end{itemize}

\hypertarget{angestrebte-lernergebnissepathlabelmi-2017modulbeschreibungen-bachelorba_screendesign}{%
\section*{Angestrebte
Lernergebnisse\label{/mi-2017/modulbeschreibungen-bachelor/BA_Screendesign}}\label{angestrebte-lernergebnissepathlabelmi-2017modulbeschreibungen-bachelorba_screendesign}}

Die Studierenden kennen wesentliche Begriffe der visuellen Kommunikation
und können diese anwenden um Briefings, Angebote oder Korrekturwünsche
im Design-Kontext zu verstehen oder zu verfassen.

Die Studierenden können Gestaltungslösungen und -kontexte analysieren,
argumentieren, diskutieren, dokumentieren und bewerten, um eigene
Lösungen innerhalb eines Gestaltungskontextes generieren zu können.

Die Studierenden können in einem gegebenen Gestaltungskontext, unter
Berücksichtigung von Gestaltungsregeln (Raster, Layout, Typographie,
etc.), eigene Gestaltungslösungen entwickeln, systematisch variieren und
argumentieren um gegebene funktionale und/oder kommunikative Ziele zu
adressieren.

\hypertarget{inhaltpathlabelmi-2017modulbeschreibungen-bachelorba_screendesign}{%
\section*{Inhalt\label{/mi-2017/modulbeschreibungen-bachelor/BA_Screendesign}}\label{inhaltpathlabelmi-2017modulbeschreibungen-bachelorba_screendesign}}

\hypertarget{vorlesungpathlabelmi-2017modulbeschreibungen-bachelorba_screendesign}{%
\subsection*{Vorlesung\label{/mi-2017/modulbeschreibungen-bachelor/BA_Screendesign}}\label{vorlesungpathlabelmi-2017modulbeschreibungen-bachelorba_screendesign}}

\begin{itemize}
\tightlist
\item
  Design Basics
\item
  Axis Map \& Semantisches Differential
\item
  Kommunikationsmodelle
\item
  Visuelle Wahrnehmung
\item
  Benutzerziele
\item
  Corporate Identity
\item
  Orientierung, Hierarchisierung, Reduktion
\item
  Räumlichkeit
\item
  Gestaltgesetze
\item
  Farbe, Kontraste
\item
  Typographie, Textsatz
\item
  Proportion
\item
  Ordnung, visuelle Struktur, Flow \& Transistion
\item
  Gestaltungsziele, Gestaltungsprozess
\end{itemize}

\hypertarget{seminarpathlabelmi-2017modulbeschreibungen-bachelorba_screendesign}{%
\subsection*{Seminar\label{/mi-2017/modulbeschreibungen-bachelor/BA_Screendesign}}\label{seminarpathlabelmi-2017modulbeschreibungen-bachelorba_screendesign}}

\begin{itemize}
\tightlist
\item
  Designprojekte strukturieren
\item
  Layoutentwicklung mit Wireframes
\item
  Layoutentwicklung für verschiedene Endgeräte
\item
  Flow \& Transition
\item
  Typographie \& Textsatz
\item
  Designkonzepte analysieren \& bewerten
\item
  Variantenbildung
\item
  Modularisierung, Interface Inventar aufbauen \& visualisieren
\end{itemize}

\hypertarget{studien-pruxfcfungsleistungenpathlabelmi-2017modulbeschreibungen-bachelorba_screendesign}{%
\section*{Studien-/Prüfungsleistungen\label{/mi-2017/modulbeschreibungen-bachelor/BA_Screendesign}}\label{studien-pruxfcfungsleistungenpathlabelmi-2017modulbeschreibungen-bachelorba_screendesign}}

Projekt und Projektpräsentationsprüfung.

\hypertarget{medienformenpathlabelmi-2017modulbeschreibungen-bachelorba_screendesign}{%
\section*{Medienformen\label{/mi-2017/modulbeschreibungen-bachelor/BA_Screendesign}}\label{medienformenpathlabelmi-2017modulbeschreibungen-bachelorba_screendesign}}

Beamergestützte Vorträge, Rechnergestützte Workshops

\hypertarget{literaturpathlabelmi-2017modulbeschreibungen-bachelorba_screendesign}{%
\section*{Literatur\label{/mi-2017/modulbeschreibungen-bachelor/BA_Screendesign}}\label{literaturpathlabelmi-2017modulbeschreibungen-bachelorba_screendesign}}

\begin{itemize}
\tightlist
\item
  Stapelkamp, Torsten: Informationsvisualisierung
\item
  Joachim Böhringer, Peter Bühler \& Patrick Schlaich: Kompendium der
  Mediengestaltung - Konzeption und Gestaltung für Digital- und
  Printmedien
\item
  Stapelkamp, Torsten: Screen- und Interfacedesign
\item
  Max Bollwage: Typografie kompakt
\item
  Kerstin Alexander: Kompendium der visuellen Information und
  Kommunikation
\item
  Maeda, John:Simplicity!: Die zehn Gesetze der Einfachheit
\item
  Lidwell, William; Holden, Kristina; Butler, Jill: Design: Die 100
  Prinzipien für erfolgreiche Gestaltung
\item
  Lewandowsky, Pina; Zeischegg, Francis: Visuelles Gestalten mit dem
  Computer
\item
  Koschembar, Frank: Grafik für Nicht-Grafiker
\end{itemize}

\hypertarget{softwaretechnikpathlabelmi-2017modulbeschreibungen-bachelorba_softwaretechnik}{%
\chapter{Softwaretechnik\label{/mi-2017/modulbeschreibungen-bachelor/BA_Softwaretechnik}}\label{softwaretechnikpathlabelmi-2017modulbeschreibungen-bachelorba_softwaretechnik}}

\begin{modulHead}
\textbf{Modulverantwortlich}: Prof.~Dr.~Mario
Winter
\end{modulHead}
\begin{modulHead}
\textbf{Studiensemester}:
4
\end{modulHead}
\begin{modulHead}
\textbf{Sprache}:
deutsch
\end{modulHead}
\begin{modulHead}
\textbf{Kreditpunkte}:
5
\end{modulHead}
\begin{modulHead}
\textbf{Typ}:
Pflichtmodul
\end{modulHead}
\begin{modulHead}
\textbf{Prüfungsleistung}:
Schriftliche Prüfung, sowie erfolgreiche Teilnahme am Praktikum als
Prüfungsvorleistung
\end{modulHead}


\hypertarget{kurzbeschreibungpathlabelmi-2017modulbeschreibungen-bachelorba_softwaretechnik}{%
\section*{Kurzbeschreibung\label{/mi-2017/modulbeschreibungen-bachelor/BA_Softwaretechnik}}\label{kurzbeschreibungpathlabelmi-2017modulbeschreibungen-bachelorba_softwaretechnik}}

Prinzipien, Methoden und Techniken der modellbasierten methodischen
objektorientierten Softwareentwicklung

\hypertarget{lehrformswspathlabelmi-2017modulbeschreibungen-bachelorba_softwaretechnik}{%
\section*{Lehrform/SWS\label{/mi-2017/modulbeschreibungen-bachelor/BA_Softwaretechnik}}\label{lehrformswspathlabelmi-2017modulbeschreibungen-bachelorba_softwaretechnik}}

4 SWS: Vorlesung 2 SWS; Praktikum 2 SWS

max. 15 Studierende/Praktikumsgruppe;

\hypertarget{arbeitsaufwandpathlabelmi-2017modulbeschreibungen-bachelorba_softwaretechnik}{%
\section*{Arbeitsaufwand\label{/mi-2017/modulbeschreibungen-bachelor/BA_Softwaretechnik}}\label{arbeitsaufwandpathlabelmi-2017modulbeschreibungen-bachelorba_softwaretechnik}}

Gesamtaufwand 150h, davon

\begin{itemize}
\tightlist
\item
  36h Vorlesung
\item
  36h Praktikum
\item
  78h Selbststudium
\end{itemize}

\hypertarget{angestrebte-lernergebnissepathlabelmi-2017modulbeschreibungen-bachelorba_softwaretechnik}{%
\section*{Angestrebte
Lernergebnisse\label{/mi-2017/modulbeschreibungen-bachelor/BA_Softwaretechnik}}\label{angestrebte-lernergebnissepathlabelmi-2017modulbeschreibungen-bachelorba_softwaretechnik}}

Die Studierenden sollen befähigt werden,

\begin{itemize}
\tightlist
\item
  zu abstrahieren, Modelle zu entwickeln, Unterschiede zwischen Modell
  und Realität zu beurteilen sowie
\item
  gegebene Modelle zu interpretieren, zu analysieren und zu bewerten,
\item
  um komplexe Systeme zu analysieren, im Team zu entwerfen und dabei im
  Rahmen methodischer Vorgehensweisen Techniken und Werkzeuge der
  objektorientierten Modellierung und Softwareentwicklung in den
  Aktivitäten Anforderungsermittlung, Softwarespezifizierung und Entwurf
  einzusetzen.
\end{itemize}

\hypertarget{inhaltpathlabelmi-2017modulbeschreibungen-bachelorba_softwaretechnik}{%
\section*{Inhalt\label{/mi-2017/modulbeschreibungen-bachelor/BA_Softwaretechnik}}\label{inhaltpathlabelmi-2017modulbeschreibungen-bachelorba_softwaretechnik}}

Die Vorlesung skizziert zunächst das Gesamtgebiet Softwaretechnik und
behandelt dann ausschließlich grundlegende „Informatikaspekte'' der
objektorientierten Softwareentwicklung. Als wesentliche Grundlage werden
die wichtigsten Elemente der Unified Modelling Language (UML)
vorgestellt und anhand kleinerer Beispiele erläutert. Danach werden
typische Aktivitäten der Softwareentwicklung besprochen, wobei die UML
als Modellierungssprache benutzt wird. Im Praktikum werden die Anwendung
der Modellierungselemente und die Durchführung der Aktivitäten in
Gruppenarbeit vertieft.

Das Modul gliedert sich in folgende Inhalte:

\begin{itemize}
\tightlist
\item
  (10\%) Softwareentwicklung im Überblick (Komplexität großer Software,
  Kernaktivitäten und unterstützende Aktivitäten);
\item
  (30\%) Die Modellierungssprache UML (Strukturmodellierung mit Objekt-
  und Klassendiagrammen, Funktionsmodellierung mit
  Anwendungsfalldiagrammen, Verhaltensmodellierung mit Sequenz-,
  Kommunikations- und Zustandsdiagrammen);
\item
  (50\%) Modellbasierte Softwareentwicklung (Anforderungsermittlung,
  Softwarespezifizierung und Architekturkonzeption, Entwurfskonzepte und
  Grobentwurf, Feinentwurf);
\item
  (10\%) Zusammenfassung und Ausblick (Modellgetriebene
  Softwareentwicklung);
\end{itemize}

\hypertarget{medienformenpathlabelmi-2017modulbeschreibungen-bachelorba_softwaretechnik}{%
\section*{Medienformen\label{/mi-2017/modulbeschreibungen-bachelor/BA_Softwaretechnik}}\label{medienformenpathlabelmi-2017modulbeschreibungen-bachelorba_softwaretechnik}}

\begin{itemize}
\tightlist
\item
  Flipped-Classroom mit Diskussion und Übungen als Einzel- und
  Kleinstgrupen
\item
  e-Vorlesungen (Video-Clips und Folien in elektronischer Form zum
  Selbststudium);
\item
  Vertiefende Materialien in elektronischer Form (z.B. SWEBOK)
\item
  Praktika in Kleingruppen, um die erlernten Modelle und Methoden
  einzuüben und zu vertiefen (Seminarraum, Rechnerlabor); In den
  Praktika werden Modellierungs- und Entwicklungswerkzeuge eingesetzt.
\end{itemize}

\hypertarget{literaturpathlabelmi-2017modulbeschreibungen-bachelorba_softwaretechnik}{%
\section*{Literatur\label{/mi-2017/modulbeschreibungen-bachelor/BA_Softwaretechnik}}\label{literaturpathlabelmi-2017modulbeschreibungen-bachelorba_softwaretechnik}}

\begin{itemize}
\tightlist
\item
  Helmut Balzert: Lehrbuch der Software-Technik Bd. I: Basiskonzepte und
  Requirements Engineering; Spektrum Akademischer Verlag, Heidelberg, 3.
  Aufl. 2009
\item
  Helmut Balzert: Lehrbuch der Software-Technik Bd. II: Entwurf,
  Implementierung, Installation und Betrieb; Spektrum Akademischer
  Verlag, Heidelberg, 3. Aufl. 2012
\item
  Helmut Balzert: Lehrbuch der Software-Technik Bd. III: Software
  Management; Spektrum Akademischer Verlag, Heidelberg, 2. Aufl. 2008
\item
  Martina Seidl et al.: UML@Classroom; dpunkt.Verlag, Heidelberg, 2012

  Unterlagen/Videos: \url{http://www.uml.ac.at/lernen}
\item
  Winter, M.: Methodische objektorientierte Softwareentwicklung.
  dpunkt.verlag, Heidelberg, 2005;
\item
  Chris Rupp et al.: UML 2 Glasklar. 4. Aufl., Carl Hanser Verlag,
  München, 2012
\item
  Jochen Ludewig, Horst Lichter: Software Engineering -- Grundlagen,
  Menschen, Prozesse, Techniken. 2. Aufl., dPunkt Verlag, Heidelberg,
  2011
\end{itemize}

\hypertarget{theoretische-informatik-1pathlabelmi-2017modulbeschreibungen-bachelorba_theoretischeinformatik1}{%
\chapter{Theoretische Informatik
1\label{/mi-2017/modulbeschreibungen-bachelor/BA_TheoretischeInformatik1}}\label{theoretische-informatik-1pathlabelmi-2017modulbeschreibungen-bachelorba_theoretischeinformatik1}}

\begin{modulHead}
\textbf{Modulverantwortlich}: Prof.~Dr.~Irma
Lindt
\end{modulHead}
\begin{modulHead}
\textbf{Studiensemester}:
1
\end{modulHead}
\begin{modulHead}
\textbf{Sprache}:
deutsch
\end{modulHead}
\begin{modulHead}
\textbf{Kreditpunkte}:
5
\end{modulHead}
\begin{modulHead}
\textbf{Typ}:
Pflichtmodul
\end{modulHead}
\begin{modulHead}
\textbf{Prüfungsleistung}:
Schriftliche Prüfung
\end{modulHead}


\hypertarget{lehrformswspathlabelmi-2017modulbeschreibungen-bachelorba_theoretischeinformatik1}{%
\section*{Lehrform/SWS\label{/mi-2017/modulbeschreibungen-bachelor/BA_TheoretischeInformatik1}}\label{lehrformswspathlabelmi-2017modulbeschreibungen-bachelorba_theoretischeinformatik1}}

4 SWS: Vorlesung 2 SWS; Übung 2 SWS

\hypertarget{arbeitsaufwandpathlabelmi-2017modulbeschreibungen-bachelorba_theoretischeinformatik1}{%
\section*{Arbeitsaufwand\label{/mi-2017/modulbeschreibungen-bachelor/BA_TheoretischeInformatik1}}\label{arbeitsaufwandpathlabelmi-2017modulbeschreibungen-bachelorba_theoretischeinformatik1}}

Gesamtaufwand 150h, davon

\begin{itemize}
\tightlist
\item
  36h Vorlesung
\item
  36h Übung
\item
  78h Selbstlernphase
\end{itemize}

\hypertarget{angestrebte-lernergebnissepathlabelmi-2017modulbeschreibungen-bachelorba_theoretischeinformatik1}{%
\section*{Angestrebte
Lernergebnisse\label{/mi-2017/modulbeschreibungen-bachelor/BA_TheoretischeInformatik1}}\label{angestrebte-lernergebnissepathlabelmi-2017modulbeschreibungen-bachelorba_theoretischeinformatik1}}

\begin{itemize}
\tightlist
\item
  Grundsätzliches Ziel des Kurses ist eine Einführung in die Begriffe,
  Methoden, Modelle und Arbeitsweise der Theoretischen Informatik anhand
  der ausgewählten Teilgebiete.
\item
  Dabei lernen die Studierenden Probleme und Sachverhalte zu
  abstrahieren und zu modellieren (etwa logische und algebraische
  Kalküle, graphentheoretische Notationen, formale Sprachen und
  Automaten sowie spezielle Kalküle wie Petri-Netze)
\item
  Die Studierenden erwerben fundierte Kenntnisse der grundlegenden
  Themengebiete und eine wesentliche Basis und Vorbereitung für
  Veranstaltungen in höheren Semestern des Studiums.
\item
  In verschiedenen Grundlagengebieten der Informatik lernen die
  Studierenden Verfahrensweisen kennen, um den algorithmischen Kern
  eines Problems zu identifizieren und können passende Algorithmen
  entwerfen (Automaten, Turing Maschinen, Logik). Dabei können Sie
  bekannte Problemstellungen im Anwendungskontext erkennen und sind mit
  den zugehörigen Lösungsmustern vertraut (Modellierung mittels
  Automaten, Petri-Netzen, Boolescher Algebra, etc.).
\item
  Aufgaben zu den Lehrinhalten (s.u.) werden in kleinen Gruppen
  (Teamarbeit) selbständig gelöst. Die Lösungen sollen in den
  Übungsstunden vorgetragen und der Lösungsweg den Kommilitonen hierbei
  erläutert werden.
\end{itemize}

\hypertarget{inhaltpathlabelmi-2017modulbeschreibungen-bachelorba_theoretischeinformatik1}{%
\section*{Inhalt\label{/mi-2017/modulbeschreibungen-bachelor/BA_TheoretischeInformatik1}}\label{inhaltpathlabelmi-2017modulbeschreibungen-bachelorba_theoretischeinformatik1}}

\begin{itemize}
\tightlist
\item
  Mengen
\item
  Relationen
\item
  Graphen
\item
  Zahlensysteme
\item
  Zahlendarstellung
\item
  Numerische Aspekte
\item
  Codierung, Informationstheorie
\item
  Boolesche Algebra
\item
  Schaltnetze und Schaltwerke
\item
  Aussagenlogik
\item
  Prädikatenlogik
\end{itemize}

\hypertarget{literaturpathlabelmi-2017modulbeschreibungen-bachelorba_theoretischeinformatik1}{%
\section*{Literatur\label{/mi-2017/modulbeschreibungen-bachelor/BA_TheoretischeInformatik1}}\label{literaturpathlabelmi-2017modulbeschreibungen-bachelorba_theoretischeinformatik1}}

\begin{itemize}
\tightlist
\item
  Hoffmann, D. (2011): Theoretische Informatik, 2. Auflage
\item
  Hedtstück, U. ( 2004 ): Einführung in die Theoretische Informatik.
  Oldenbourg, München.
\item
  Kelly, J. ( 2003 ): Logik. Pearson Studium, München.
\item
  Ehrig, H. et al.~(1999): Mathematisch-strukturelle Grundlagen der
  Informatik. Springer,~ Heidelberg.
\item
  Beuth, K. (1992): Digitaltechnik. 9.Aufl.Vogel, Würzburg.
\end{itemize}

\hypertarget{theoretische-informatik-2pathlabelmi-2017modulbeschreibungen-bachelorba_theoretischeinformatik2}{%
\chapter{Theoretische Informatik
2\label{/mi-2017/modulbeschreibungen-bachelor/BA_TheoretischeInformatik2}}\label{theoretische-informatik-2pathlabelmi-2017modulbeschreibungen-bachelorba_theoretischeinformatik2}}

\begin{modulHead}
\textbf{Modulverantwortlich}: Prof.~Dr.~Irma
Lindt
\end{modulHead}
\begin{modulHead}
\textbf{Studiensemester}:
2
\end{modulHead}
\begin{modulHead}
\textbf{Sprache}:
deutsch
\end{modulHead}
\begin{modulHead}
\textbf{Kreditpunkte}:
5
\end{modulHead}
\begin{modulHead}
\textbf{Typ}:
Pflichtmodul
\end{modulHead}
\begin{modulHead}
\textbf{Prüfungsleistung}:
Schriftliche Prüfung
\end{modulHead}


\hypertarget{lehrformswspathlabelmi-2017modulbeschreibungen-bachelorba_theoretischeinformatik2}{%
\section*{Lehrform/SWS\label{/mi-2017/modulbeschreibungen-bachelor/BA_TheoretischeInformatik2}}\label{lehrformswspathlabelmi-2017modulbeschreibungen-bachelorba_theoretischeinformatik2}}

4 SWS: Vorlesung 2 SWS; Übung 2 SWS

\hypertarget{arbeitsaufwandpathlabelmi-2017modulbeschreibungen-bachelorba_theoretischeinformatik2}{%
\section*{Arbeitsaufwand\label{/mi-2017/modulbeschreibungen-bachelor/BA_TheoretischeInformatik2}}\label{arbeitsaufwandpathlabelmi-2017modulbeschreibungen-bachelorba_theoretischeinformatik2}}

Gesamtaufwand 150h, davon

\begin{itemize}
\tightlist
\item
  36h Vorlesung
\item
  36h Übung
\item
  78h Selbstlernphase
\end{itemize}

\hypertarget{angestrebte-lernergebnissepathlabelmi-2017modulbeschreibungen-bachelorba_theoretischeinformatik2}{%
\section*{Angestrebte
Lernergebnisse\label{/mi-2017/modulbeschreibungen-bachelor/BA_TheoretischeInformatik2}}\label{angestrebte-lernergebnissepathlabelmi-2017modulbeschreibungen-bachelorba_theoretischeinformatik2}}

siehe Theoretische Informatik 1.

\hypertarget{inhaltpathlabelmi-2017modulbeschreibungen-bachelorba_theoretischeinformatik2}{%
\section*{Inhalt\label{/mi-2017/modulbeschreibungen-bachelor/BA_TheoretischeInformatik2}}\label{inhaltpathlabelmi-2017modulbeschreibungen-bachelorba_theoretischeinformatik2}}

\begin{itemize}
\tightlist
\item
  Reguläre (Typ-3) Sprachen: Endliche Automaten, Reguläre Ausdrücke;
  Typ3-Grammatiken, Zustandsübergangsdiagramme; Chomsky-Hierarchie
\item
  Modellierung sequentieller und paralleler (Ausgabe-) Prozesse:
  Endliche Maschinen / Automaten; Automatennetze, Petri-Netze, Zelluläre
  Automaten
\item
  Kontextfreie (Typ-2) Sprachen: Kontextfreie Grammatiken,
  Chomsky-Normalform; Kellerautomaten; Anwendungen (Ableitungs- und
  Syntaxbäume, Syntax von Programmiersprachen, Backus-Naur-Form)
\item
  Kontextsensitive (Typ-1) und rekursiv aufzählende (Typ-0) Sprachen:
  Grammatiken, Turingautomaten, Einführung in die Begriffe:
  Berechenbarkeit, Entscheidbarkeit und Komplexität
\end{itemize}

\hypertarget{literaturpathlabelmi-2017modulbeschreibungen-bachelorba_theoretischeinformatik2}{%
\section*{Literatur\label{/mi-2017/modulbeschreibungen-bachelor/BA_TheoretischeInformatik2}}\label{literaturpathlabelmi-2017modulbeschreibungen-bachelorba_theoretischeinformatik2}}

\begin{itemize}
\tightlist
\item
  Hoffmann, D. (2011): Theoretische Informatik, 2. Auflage
\item
  Vossen, G., Witt K. (2004): Grundlagen der Theoretischen Informatik
  mit Anwendungen.~3. Aufl.~ Vieweg \& Sohn, Braunschweig.
\item
  Hedtstück, U. ( 2004 ): Einführung in die Theoretische Informatik.
  Oldenbourg, München.
\item
  Asteroth, A., Baier, C. (2002) Theoretische Informatik. Pearson
  Studium München
\item
  Hopcroft, J. E.~ et al.~(2002): Einführung in die Automatentheorie,
  Formale Sprachen und Komplexitätstheorie. Pearson Studium, München.
\item
  Schöning, U. (1997): Theoretische Informatik - kurzgefaßt. 3. Aufl.
  Spektrum Akademischer Verlag, Heidelberg.
\end{itemize}

\hypertarget{audiovisuelle-medientechnikpathlabelmi-2017modulbeschreibungen-bachelorba_vc-audiovisuelle-medientechnik}{%
\chapter{Audiovisuelle
Medientechnik\label{/mi-2017/modulbeschreibungen-bachelor/BA_VC-audiovisuelle-medientechnik}}\label{audiovisuelle-medientechnikpathlabelmi-2017modulbeschreibungen-bachelorba_vc-audiovisuelle-medientechnik}}

\begin{modulHead}
\textbf{Modulverantwortlich}: Prof.~Hans
Kornacher
\end{modulHead}
\begin{modulHead}
\textbf{Studiensemester}:
4
\end{modulHead}
\begin{modulHead}
\textbf{Sprache}:
deutsch
\end{modulHead}
\begin{modulHead}
\textbf{Kreditpunkte}:
5
\end{modulHead}
\begin{modulHead}
\textbf{Typ}:
Teilmodul
\end{modulHead}
\begin{modulHead}
\textbf{Prüfungsleistung}:
Schriftliche Prüfung, in Ausnahmefällen mündliche
Online-Prüfung
\end{modulHead}


\hypertarget{lehrformswspathlabelmi-2017modulbeschreibungen-bachelorba_vc-audiovisuelle-medientechnik}{%
\section*{Lehrform/SWS\label{/mi-2017/modulbeschreibungen-bachelor/BA_VC-audiovisuelle-medientechnik}}\label{lehrformswspathlabelmi-2017modulbeschreibungen-bachelorba_vc-audiovisuelle-medientechnik}}

36h Vorlesung; 36h Praktikum / Projekt; 78h Selbstlernphase

\hypertarget{angestrebte-lernergebnissepathlabelmi-2017modulbeschreibungen-bachelorba_vc-audiovisuelle-medientechnik}{%
\section*{Angestrebte
Lernergebnisse\label{/mi-2017/modulbeschreibungen-bachelor/BA_VC-audiovisuelle-medientechnik}}\label{angestrebte-lernergebnissepathlabelmi-2017modulbeschreibungen-bachelorba_vc-audiovisuelle-medientechnik}}

Die Studierenden sollen durch dieses Modul dazu befähigt werden, auf
Basis der technischen Grundlagen der Video- und Fernsehtechnik
weitergehende Fragestellungen selbstständig zu erarbeiten und sich so
auch zukünftige technische Entwicklungen autonom erschließen zu können.

Neben der Entwicklung und Förderung dieser Fachkompetenz ist die
Initiierung der Methodenkompetenz eine wichtige Säule des
Vorlesungsmoduls. Unter Methodenkompetenz ist hier die
Selbstorganisation im Sinne von wissenschaftlicher Fragestellung an
einen Themenkomplex und ein strukturiertes Vorgehen in der Erarbeitung
eines Lösungsansatzes zu verstehen. Ziel ist es, das Wissen aus
verschiedenen Bereichen, wie Kerninformatik, Internet- und
Webtechnologien und benachbarten Wissenschaften mit der in diesem Modul
unterrichteten Medientechnologien zu kombinieren und in die
Medienproduktion zu integrieren.

Gerade der Umgang mit Technologien und Methoden aus der Film- und
Fernsehproduktion erweitert den Erfahrungshorizont der Studierenden über
den bekannten Themenbereich der Kerninformatik hinaus und legt ihnen
eine Einarbeitung in informatikfremde Sachverhalte und technologische
Problemstellungen und deren Lösungsmethoden nahe.

Pragmatisches Ziel ist es, in den unterschiedlichsten Berufsfeldern
audiovisueller Medien die Entwicklung und den Einsatz digitaler
Medientechnik zu beraten, zu planen, durchzuführen oder zu verantworten.

\hypertarget{inhaltpathlabelmi-2017modulbeschreibungen-bachelorba_vc-audiovisuelle-medientechnik}{%
\section*{Inhalt\label{/mi-2017/modulbeschreibungen-bachelor/BA_VC-audiovisuelle-medientechnik}}\label{inhaltpathlabelmi-2017modulbeschreibungen-bachelorba_vc-audiovisuelle-medientechnik}}

\begin{itemize}
\tightlist
\item
  Grundlagen der Fernsehtechnik
\item
  Digitale Fernsehtechnik
\item
  HD-Technik
\item
  Videodatenreduktion
\item
  Bildwandler
\item
  Das optische System der Videokamera
\item
  Signalverarbeitung in der Videokamera
\item
  Signalaufzeichnung
\item
  Elektroakustik
\item
  Bildwiedergabesysteme
\item
  Lichttechnik und Beleuchtung
\end{itemize}

\hypertarget{medienformenpathlabelmi-2017modulbeschreibungen-bachelorba_vc-audiovisuelle-medientechnik}{%
\section*{Medienformen\label{/mi-2017/modulbeschreibungen-bachelor/BA_VC-audiovisuelle-medientechnik}}\label{medienformenpathlabelmi-2017modulbeschreibungen-bachelorba_vc-audiovisuelle-medientechnik}}

\begin{itemize}
\tightlist
\item
  Vorlesungen mit Folienpräsentationen
\item
  Workshops zu Anwendungsprogrammen
\item
  Beispiele aus verschiedenen Medien: Filmbeispiele, Webvideos
\item
  Audiovisuelle Aufnahme- und Wiedergabegeräte
\item
  Lehrfilme und Video-Tutorials
\end{itemize}

\hypertarget{literaturpathlabelmi-2017modulbeschreibungen-bachelorba_vc-audiovisuelle-medientechnik}{%
\section*{Literatur\label{/mi-2017/modulbeschreibungen-bachelor/BA_VC-audiovisuelle-medientechnik}}\label{literaturpathlabelmi-2017modulbeschreibungen-bachelorba_vc-audiovisuelle-medientechnik}}

\begin{itemize}
\tightlist
\item
  Schmidt Ulrich, Professionelle Videotechnik, Springer-Verlag Berlin
  Heidelberg New York 2013, ISBN 978-3-642-38992-4
\item
  Johannes Webers, Film- und Fernsehtechnik, Franzis Verlag, Poing 2000,
  ISBN 3-7723-7116-7
\item
  Möllering, Slansky, Handbuch der professionellen Videoaufnahme Edition
  Filmwerkstatt, Essen, 1993, ISBN 3 - 9 802 581 - 3 - 0
\end{itemize}

\hypertarget{teilmodul-vonpathlabelmi-2017modulbeschreibungen-bachelorba_vc-audiovisuelle-medientechnik}{%
\section*{Teilmodul
von:\label{/mi-2017/modulbeschreibungen-bachelor/BA_VC-audiovisuelle-medientechnik}}\label{teilmodul-vonpathlabelmi-2017modulbeschreibungen-bachelorba_vc-audiovisuelle-medientechnik}}

\hyperref[/mi-2017/modulbeschreibungen-bachelor/BA_Vertiefung-Visual-Computing]{Vertiefung – Visual Computing}

\hypertarget{audiovisuelles-medienprojekt-2pathlabelmi-2017modulbeschreibungen-bachelorba_vc-audiovisuelles-medienprojekt-2}{%
\chapter{Audiovisuelles Medienprojekt
2\label{/mi-2017/modulbeschreibungen-bachelor/BA_VC-audiovisuelles-medienprojekt-2}}\label{audiovisuelles-medienprojekt-2pathlabelmi-2017modulbeschreibungen-bachelorba_vc-audiovisuelles-medienprojekt-2}}

\begin{modulHead}
\textbf{Modulverantwortlich}: Prof.~Hans
Kornacher
\end{modulHead}
\begin{modulHead}
\textbf{Studiensemester}:
4
\end{modulHead}
\begin{modulHead}
\textbf{Sprache}:
deutsch
\end{modulHead}
\begin{modulHead}
\textbf{Kreditpunkte}:
5
\end{modulHead}
\begin{modulHead}
\textbf{Typ}:
Teilmodul
\end{modulHead}
\begin{modulHead}
\textbf{Prüfungsleistung}:
Projektarbeit mit schriftlicher Ausarbeitung
\end{modulHead}


\hypertarget{lehrformswspathlabelmi-2017modulbeschreibungen-bachelorba_vc-audiovisuelles-medienprojekt-2}{%
\section*{Lehrform/SWS\label{/mi-2017/modulbeschreibungen-bachelor/BA_VC-audiovisuelles-medienprojekt-2}}\label{lehrformswspathlabelmi-2017modulbeschreibungen-bachelorba_vc-audiovisuelles-medienprojekt-2}}

36h Vorlesung; 36h Praktikum / Projekt; 78h Selbstlernphase

\hypertarget{angestrebte-lernergebnissepathlabelmi-2017modulbeschreibungen-bachelorba_vc-audiovisuelles-medienprojekt-2}{%
\section*{Angestrebte
Lernergebnisse\label{/mi-2017/modulbeschreibungen-bachelor/BA_VC-audiovisuelles-medienprojekt-2}}\label{angestrebte-lernergebnissepathlabelmi-2017modulbeschreibungen-bachelorba_vc-audiovisuelles-medienprojekt-2}}

Die praktische Umsetzung des Vorlesungsstoffes, die Kommunikation und
Zusammenarbeit im Team über Themenbereiche dieses Faches und die
Präsentation von eigenen Projekten und Untersuchungsergebnissen sind die
Lernziele des Moduls Audiovisuelles Medienprojekt 2. Neben dieser
formulierten Fachkompetenz, Methodenkompetenz und
Kommunikationskompetenz stehen gerade die sogenannten Softskills
Teamfähigkeit und Kommunikationsfähigkeit im Focus der Ausbildung in
diesem Modul.

Die Studierenden kennen über die grundlegenden Erzählformen
audiovisueller Medien hinaus spezielle Formate wie Spielfilm, Imagefilm
und Studioproduktion und haben dabei folgende Fertigkeiten: Sie können
eigene audiovisuelle Erzählformen auf der Basis dieser Erzählmuster
entwickeln und sind befähigt zur Analyse, zur Diskussion und zur
kritischen Betrachtung audiovisueller Medieninhalte.

Pragmatisches Ziel ist es, in den unterschiedlichsten Berufsfeldern
digitaler audiovisueller Medien die Entwicklung und den Einsatz
audiovisuellen Content zu beraten, zu planen, durchzuführen oder zu
verantworten.

\hypertarget{inhaltpathlabelmi-2017modulbeschreibungen-bachelorba_vc-audiovisuelles-medienprojekt-2}{%
\section*{Inhalt\label{/mi-2017/modulbeschreibungen-bachelor/BA_VC-audiovisuelles-medienprojekt-2}}\label{inhaltpathlabelmi-2017modulbeschreibungen-bachelorba_vc-audiovisuelles-medienprojekt-2}}

\begin{itemize}
\tightlist
\item
  Vertiefung der Video- und Audioaufnahmetechnik
\item
  Verschiedene Dramaturgiemodelle
\item
  Drehbuch, Auflösung, Storyboard
\item
  Schnitt und Montage
\item
  Medienproduktion in den Formaten Werbefilm, Imagefilm und
  Studioproduktion
\end{itemize}

\hypertarget{medienformenpathlabelmi-2017modulbeschreibungen-bachelorba_vc-audiovisuelles-medienprojekt-2}{%
\section*{Medienformen\label{/mi-2017/modulbeschreibungen-bachelor/BA_VC-audiovisuelles-medienprojekt-2}}\label{medienformenpathlabelmi-2017modulbeschreibungen-bachelorba_vc-audiovisuelles-medienprojekt-2}}

\begin{itemize}
\tightlist
\item
  Vorlesungen mit Folienpräsentationen
\item
  Workshops zu Anwendungsprogrammen
\item
  Beispiele aus verschiedenen Medien: Filmbeispiele, Webvideos
\item
  Audiovisuelle Aufnahme- und Wiedergabegeräte
\item
  Lehrfilme und Video-Tutorials
\end{itemize}

\hypertarget{literaturpathlabelmi-2017modulbeschreibungen-bachelorba_vc-audiovisuelles-medienprojekt-2}{%
\section*{Literatur\label{/mi-2017/modulbeschreibungen-bachelor/BA_VC-audiovisuelles-medienprojekt-2}}\label{literaturpathlabelmi-2017modulbeschreibungen-bachelorba_vc-audiovisuelles-medienprojekt-2}}

\begin{itemize}
\tightlist
\item
  James Monaco, Film verstehen, Rowolth Taschenbuch Verlag Hamburg,
  1980, ISBN 3-499-162717
\item
  Syd Field, Drehbuchschreiben für Film und Fernsehen, München 2003,
  ISBN 354836473X
\item
  Steven D. Katz, Die Richtige Einstellung, Zweitausendeins, Frankfurt
  a.M.1998,ISBN 3-86150-229-1
\item
  David Lewis Yewdall, Practical Art of Motion Picture Sound, Focal
  Press, USA 2003, ISBN 0-240-80525-9
\item
  Hans Kornacher \& Manfred Stross, Dokumentarisches Videofilmen,
  Augustus Verlag, Augsburg, 1992, ISBN 3-8043-5474-2
\item
  Hans Beller Hg., Handbuch der Filmmontage, München: TR-Verlagsunion,
  1993, ISBN 3-8058-2357-6
\item
  Karel Reisz, Gavin Millar, Geschichte und Technik der Filmmontage,
  München: Filmlandpresse, 1988, ISBN 3-88690-071-1
\item
  Chris Vogler, Die Reise des Drehbuchschreibens, Verlag Zweitausendeins
\item
  Wolfgang Lanzenberger, Michael Müller, Unternehmensfilme drehen:
  Business Movies im digitalen Zeitalter, ISBN 978-386764367
\end{itemize}

\hypertarget{teilmodul-vonpathlabelmi-2017modulbeschreibungen-bachelorba_vc-audiovisuelles-medienprojekt-2}{%
\section*{Teilmodul
von:\label{/mi-2017/modulbeschreibungen-bachelor/BA_VC-audiovisuelles-medienprojekt-2}}\label{teilmodul-vonpathlabelmi-2017modulbeschreibungen-bachelorba_vc-audiovisuelles-medienprojekt-2}}

\hyperref[/mi-2017/modulbeschreibungen-bachelor/BA_Vertiefung-Visual-Computing]{Vertiefung – Visual Computing}

\hypertarget{bildverarbeitung-und-computer-visionpathlabelmi-2017modulbeschreibungen-bachelorba_vc-bildverarbeitung-und-computer-vision}{%
\chapter{Bildverarbeitung und Computer
Vision\label{/mi-2017/modulbeschreibungen-bachelor/BA_VC-bildverarbeitung-und-computer-vision}}\label{bildverarbeitung-und-computer-visionpathlabelmi-2017modulbeschreibungen-bachelorba_vc-bildverarbeitung-und-computer-vision}}

\begin{modulHead}
\textbf{Modulverantwortlich}: Prof.~Dr.~Daniel
Gaida
\end{modulHead}
\begin{modulHead}
\textbf{Studiensemester}:
4
\end{modulHead}
\begin{modulHead}
\textbf{Sprache}:
deutsch
\end{modulHead}
\begin{modulHead}
\textbf{Kreditpunkte}:
5
\end{modulHead}
\begin{modulHead}
\textbf{Typ}:
Teilmodul
\end{modulHead}
\begin{modulHead}
\textbf{Prüfungsleistung}:
mündlicher Beitrag (Präsentation)
\end{modulHead}


\hypertarget{aufwandpathlabelmi-2017modulbeschreibungen-bachelorba_vc-bildverarbeitung-und-computer-vision}{%
\section*{Aufwand\label{/mi-2017/modulbeschreibungen-bachelor/BA_VC-bildverarbeitung-und-computer-vision}}\label{aufwandpathlabelmi-2017modulbeschreibungen-bachelorba_vc-bildverarbeitung-und-computer-vision}}

32h Vorlesung; 36h Projekt; 82h Selbstlernphase

\hypertarget{angestrebte-lernergebnissepathlabelmi-2017modulbeschreibungen-bachelorba_vc-bildverarbeitung-und-computer-vision}{%
\section*{Angestrebte
Lernergebnisse\label{/mi-2017/modulbeschreibungen-bachelor/BA_VC-bildverarbeitung-und-computer-vision}}\label{angestrebte-lernergebnissepathlabelmi-2017modulbeschreibungen-bachelorba_vc-bildverarbeitung-und-computer-vision}}

Die Studierenden können robuste Bildverarbeitungspipelines in Python
entwickeln, indem sie

\begin{itemize}
\tightlist
\item
  durch die Anwendung von Punkt- und Filteroperationen eine geeignete
  Bildvorverarbeitung implementieren, wie bspw. Kontrasterhöhung,
  Reduktion des Bildrauschens, Erhöhung der Bildschärfe und
  Kantendetektion,
\item
  markante Bildregionen in einem Bild identifizieren können und diese
  Bildregionen in anderen Bildern der gleichen Szene (evtl. aufgenommen
  aus einem anderen Blickwinkel, aus einer anderen Entfernung)
  wiederfinden und lokalisieren können,
\item
  Bilder transformieren (bspw. rotieren, skalieren, rektifizieren)
  können,
\item
  3-D Informationen der Szene durch Nutzung eines oder mehrerer Bilder
  der Szene bestimmen können, wie bspw. die Objektgröße, der
  Kameraentfernung, oder dem Abstand zwischen Objekten in der Szene,
\item
  in Bildsequenzen (Videos) Objekte tracken können,
\item
  sich in Teamarbeit in ein Bildverarbeitungsthema ihrer Wahl
  selbstständig ein- und dieses ausarbeiten können und den anderen
  Studierenden des Kurses in Form einer aktivierenden Lehreinheit
  präsentieren können,
\end{itemize}

um später zuverlässige Anwendungen im Bereich der Bildverarbeitung
realisieren und Teammitgliedern/Kunden erklären zu können, in denen
reale Bilder oder Bildsequenzen mit dem Ziel verarbeitet werden sollen,
unterschiedliche Fragestellungen anhand dieser Bilder beantworten zu
können.

\hypertarget{inhaltpathlabelmi-2017modulbeschreibungen-bachelorba_vc-bildverarbeitung-und-computer-vision}{%
\section*{Inhalt\label{/mi-2017/modulbeschreibungen-bachelor/BA_VC-bildverarbeitung-und-computer-vision}}\label{inhaltpathlabelmi-2017modulbeschreibungen-bachelorba_vc-bildverarbeitung-und-computer-vision}}

\begin{itemize}
\tightlist
\item
  Was ist das Ziel der Bildverarbeitung und von Computer Vision?
\item
  Wie entstehen Bilder und wie werden Bilder auf dem Computer
  dargestellt?
\item
  Wie verändert man den Kontrast und die Helligkeit eines Bildes?
\item
  Wie verringert man das Bildrauschen, erhöht die Bildschärfe,
  detektiert Kanten in Bildern und verringert die Bildgröße?
\item
  Umsetzung grundlegender Bildverarbeitung in Python durch Nutzung von
  scikit-image oder opencv2.
\item
  Themen, die durch Studierende ausgearbeitet und präsentiert werden
  können:

  \begin{itemize}
\tightlist
\item
    Wie erkennt man Linien und Kreise in Bildern?
  \item
    Wie findet man Bildregionen eines Bildes in einem anderen Bild
    wieder?
  \item
    Bilder transformieren durch Drehen, Schieben und Skalieren (bis hin
    zu nichtlinearer Transformation: bspw. Rektifizierung)
  \item
    Erstellung von Panorama Bildern
  \item
    Wie groß sind fotografierte Objekte in der realen Welt?
  \item
    Wie weit entfernt von der Kamera sind die Objekte in der Szene?
  \item
    Wo in der Szene befindet sich die Kamera?
  \item
    Tracking von Objekten in Videos
  \item
    Erkennung von Schnitten in Videos
  \item
    Wie funktioniert Bildstabilisierung in Soft- und Hardware?
  \item
    Bilder im Frequenzraum betrachten: Welchen Mehrwert hat das?
  \item
    Welche Bilddateiformate gibt es und wie funktioniert
    Bildkompression?
  \end{itemize}
\end{itemize}

\hypertarget{studien-pruxfcfungsleistungenpathlabelmi-2017modulbeschreibungen-bachelorba_vc-bildverarbeitung-und-computer-vision}{%
\section*{Studien-/Prüfungsleistungen\label{/mi-2017/modulbeschreibungen-bachelor/BA_VC-bildverarbeitung-und-computer-vision}}\label{studien-pruxfcfungsleistungenpathlabelmi-2017modulbeschreibungen-bachelorba_vc-bildverarbeitung-und-computer-vision}}

\begin{itemize}
\tightlist
\item
  mündlicher Beitrag, d.h. Präsentation inkl. Übung, Handout und
  Diskussion
\end{itemize}

\hypertarget{medienformenpathlabelmi-2017modulbeschreibungen-bachelorba_vc-bildverarbeitung-und-computer-vision}{%
\section*{Medienformen\label{/mi-2017/modulbeschreibungen-bachelor/BA_VC-bildverarbeitung-und-computer-vision}}\label{medienformenpathlabelmi-2017modulbeschreibungen-bachelorba_vc-bildverarbeitung-und-computer-vision}}

\begin{itemize}
\tightlist
\item
  Vorlesungen mit Folienpräsentationen
\item
  Übungen in Python mit scikit-image und/oder opencv2
\end{itemize}

\hypertarget{literaturpathlabelmi-2017modulbeschreibungen-bachelorba_vc-bildverarbeitung-und-computer-vision}{%
\section*{Literatur\label{/mi-2017/modulbeschreibungen-bachelor/BA_VC-bildverarbeitung-und-computer-vision}}\label{literaturpathlabelmi-2017modulbeschreibungen-bachelorba_vc-bildverarbeitung-und-computer-vision}}

\begin{itemize}
\tightlist
\item
  W. Burger, M. J. Burge, Digitale Bildverarbeitung -- Eine Einführung
  mit Java und ImageJ. eXamen.press, 3. Auflage, 2015.
\item
  Richard Szeliski, Computer Vision: Algorithms and Applications, 2nd
  ed., 2022
\end{itemize}

\hypertarget{teilmodul-vonpathlabelmi-2017modulbeschreibungen-bachelorba_vc-bildverarbeitung-und-computer-vision}{%
\section*{Teilmodul
von:\label{/mi-2017/modulbeschreibungen-bachelor/BA_VC-bildverarbeitung-und-computer-vision}}\label{teilmodul-vonpathlabelmi-2017modulbeschreibungen-bachelorba_vc-bildverarbeitung-und-computer-vision}}

\hyperref[/mi-2017/modulbeschreibungen-bachelor/BA_Vertiefung-Visual-Computing]{Vertiefung – Visual Computing}

\hypertarget{computergrafik-und-animationpathlabelmi-2017modulbeschreibungen-bachelorba_vc-computergrafik-und-animation}{%
\chapter{Computergrafik und
Animation\label{/mi-2017/modulbeschreibungen-bachelor/BA_VC-computergrafik-und-animation}}\label{computergrafik-und-animationpathlabelmi-2017modulbeschreibungen-bachelorba_vc-computergrafik-und-animation}}

\begin{modulHead}
\textbf{Modulverantwortlich}: Prof.~Dr.~Horst
Stenzel
\end{modulHead}
\begin{modulHead}
\textbf{Studiensemester}:
4
\end{modulHead}
\begin{modulHead}
\textbf{Sprache}:
deutsch
\end{modulHead}
\begin{modulHead}
\textbf{Kreditpunkte}:
5
\end{modulHead}
\begin{modulHead}
\textbf{Typ}:
Teilmodul
\end{modulHead}
\begin{modulHead}
\textbf{Prüfungsleistung}:
Durchführung eines Projektes, sowie erfolgreiche Teilnahme am Praktikum
als Prüfungsvorleistung
\end{modulHead}


\hypertarget{lehrformswspathlabelmi-2017modulbeschreibungen-bachelorba_vc-computergrafik-und-animation}{%
\section*{Lehrform/SWS\label{/mi-2017/modulbeschreibungen-bachelor/BA_VC-computergrafik-und-animation}}\label{lehrformswspathlabelmi-2017modulbeschreibungen-bachelorba_vc-computergrafik-und-animation}}

36h Vorlesung; 36h Praktikum / Projekt; 78h Selbstlernphase

\hypertarget{angestrebte-lernergebnissepathlabelmi-2017modulbeschreibungen-bachelorba_vc-computergrafik-und-animation}{%
\section*{Angestrebte
Lernergebnisse\label{/mi-2017/modulbeschreibungen-bachelor/BA_VC-computergrafik-und-animation}}\label{angestrebte-lernergebnissepathlabelmi-2017modulbeschreibungen-bachelorba_vc-computergrafik-und-animation}}

Die Grundlagen der zwei- und insbesondere der dreidimensionalen
Computergraphik und Animation stellen ein hervorragendes Paradigma zur
Vermittlung zentraler Inhalte und Kompetenzen der Medieninformatik dar.

Den Studierenden wird deutlich, wie der Bogen von den abstrakten,
geometrischen und algorithmischen Fakten zu den pragmatischen
Gegebenheiten der Computergraphik-Hardware gespannt ist.

Sie erkennen die Zusammenhänge zwischen Grundlagenvorlesungen
(Mathematik, Algorithmen, Programmierung) und der Gestaltung von
Schnittstellen und Oberflächen und werden so für die jeweiligen Inhalte
zusätzlich motiviert.

Dabei lernen Sie, im Kontext der Computergrafik, Verfahrensweisen
kennen, um den algorithmischen Kern eines Problems zu identifizieren und
können Algorithmen entwerfen, verifizieren und bzgl. ihres
Ressourcenbedarfs bewerten.

Sie erwerben die Fähigkeit, aktuelle technologische Entwicklungen im
Medieninformatik-Kontext zu bewerten und Trends einzuordnen.

Nach Abschluss des Moduls besitzen die Studierenden grundlegende
Kenntnisse über Architektur und Programmierung moderner Graphikhardware,
sowie deren Anwendung in konkreten Problemstellungen und
Anwendungskontexten.

Am Beispiel von OpenGL und der Rendering-Pipeline lernen die
Studierenden Problemstellungen im Anwendungskontext zu erkennen und sind
mit den zugehörigen Lösungsmustern durch praktische Programmierung
vertraut.

Das erlernte Wissen und die erlernten Kenntnisse in der Soft- und
Grafikhardware-Architektur ermöglicht es erfolgreichen Teilnehmern,
anschließend Echtzeit-Visualisierungen mit OpenGL zu implementieren und
somit mit einer modernen, plattformunabhängigen API umzugehen, die
flexibel an bestehende Anforderungen angepasst werden kann. Zudem haben
Sie die Fähigkeit hochparallele Algorithmen auf der Graphikkarte zu
entwerfen und auszuführen.

Dabei beherrschen die Studierenden nach Abschluss des Moduls die
Fähigkeit abstrakte Szenen- und Objektbeschreibungen zu erstellen und
darzustellen, sowie sich in vorhandenen Quelltext einzuarbeiten und
diesen sinnvoll weiter zu entwickeln.

Die Inhalte des Moduls befähigen die Studierenden die grundlegenden
Algorithmen und Datenstrukturen der Echtzeit-Computergraphik zu
beherrschen.

Die Studierenden können ihr erworbenes Können und Wissen zur
Implementierung einer eigenen Game/Visualisierungs-Engine einsetzen.
Dies zeigen Sie durch Umsetzung eines eigenen Projektes in Kleingruppen,
wo sie zusätzlich lernen mündlich überzeugend zu präsentieren,
abweichende Positionen zu erkennen und in eine sach- und
interessengerechte Lösung zu integrieren. Sie zeigen dadurch, dass Sie
in der Lage sind sich selbstständig neues Wissen anzueigenen und zu
erkennen, welches Wissen relevant ist, können mediengestalterische
Grundkompetenzen anwenden und besitzen aktive Vokabularien zur
Beschreibung und Realisierung angemessener Konzeptionen. Zudem können
sie die Realisationen bezüglich der Zielsetzungen kritisch diskutieren.

\hypertarget{inhaltpathlabelmi-2017modulbeschreibungen-bachelorba_vc-computergrafik-und-animation}{%
\section*{Inhalt\label{/mi-2017/modulbeschreibungen-bachelor/BA_VC-computergrafik-und-animation}}\label{inhaltpathlabelmi-2017modulbeschreibungen-bachelorba_vc-computergrafik-und-animation}}

\begin{itemize}
\tightlist
\item
  Graphikhardware,
\item
  OpenGL
\item
  Transformationen und homogene Koordinaten
\item
  Interpolation
\item
  Kameramodelle
\item
  Clipping
\item
  Shaderprogrammierung
\item
  Animation
\item
  Texturierung
\item
  Fortgeschrittene Algorithmen (Schatten, Reflexionen, Bump-, Normal-,
  Parallax-, Relief-Mapping, Globale Beleuchtung, Deferred Shading)
\item
  Perzeption
\item
  Grundlagen des Ray Tracings
\end{itemize}

\hypertarget{medienformenpathlabelmi-2017modulbeschreibungen-bachelorba_vc-computergrafik-und-animation}{%
\section*{Medienformen\label{/mi-2017/modulbeschreibungen-bachelor/BA_VC-computergrafik-und-animation}}\label{medienformenpathlabelmi-2017modulbeschreibungen-bachelorba_vc-computergrafik-und-animation}}

\begin{itemize}
\tightlist
\item
  Beamer-gestützte Vorlesungen
\item
  Rechnergestützte Workshops
\item
  Beispiele aus verschiedenen Medien in elektronischer Form:
  Filmbeispiele, Webvideos
\item
  Audiovisuelle Aufnahme- und Wiedergabegeräte
\item
  Interaktive Projektionsfläche
\item
  Lehrfilme
\end{itemize}

\hypertarget{literaturpathlabelmi-2017modulbeschreibungen-bachelorba_vc-computergrafik-und-animation}{%
\section*{Literatur\label{/mi-2017/modulbeschreibungen-bachelor/BA_VC-computergrafik-und-animation}}\label{literaturpathlabelmi-2017modulbeschreibungen-bachelorba_vc-computergrafik-und-animation}}

\begin{itemize}
\tightlist
\item
  Peter Shirley, Fundamentals of Computer Graphics, Peters, Wellesley
\item
  Andrew Woo, et al., OpenGL Programming Guide, Version 4.3,
  Addison-Wesley,
\item
  Tomas Akenine-Möller, Eric Haines, und Naty Hoffman, Real-Time
  Rendering, 3. Ausgabe, Peters, Wellesley
\item
  Randi J. Rost, John M. Kessenich, Barthold Lichtenbelt, OpenGL Shading
  Language, 2. Ausgabe, Addison-Wesley
\item
  Alan Watt, 3D Computer Graphics, Addison-Wesley
\item
  Frank Nielsen, Visual Computing, Charles River Media, 2005
\item
  James Foley, Andries Van Dam, et al., Computer Graphics : Principles
  and Practice, 2. Ausgabe, Addison-Wesley
\end{itemize}

\hypertarget{teilmodul-vonpathlabelmi-2017modulbeschreibungen-bachelorba_vc-computergrafik-und-animation}{%
\section*{Teilmodul
von:\label{/mi-2017/modulbeschreibungen-bachelor/BA_VC-computergrafik-und-animation}}\label{teilmodul-vonpathlabelmi-2017modulbeschreibungen-bachelorba_vc-computergrafik-und-animation}}

\hyperref[/mi-2017/modulbeschreibungen-bachelor/BA_Vertiefung-Visual-Computing]{Vertiefung – Visual Computing}

\hypertarget{visuelle-effekte-und-animationpathlabelmi-2017modulbeschreibungen-bachelorba_vc-visuelle-effekte-und-animation}{%
\chapter{Visuelle Effekte und
Animation\label{/mi-2017/modulbeschreibungen-bachelor/BA_VC-visuelle-effekte-und-animation}}\label{visuelle-effekte-und-animationpathlabelmi-2017modulbeschreibungen-bachelorba_vc-visuelle-effekte-und-animation}}

\begin{modulHead}
\textbf{Modulverantwortlich}: Prof.~Hans
Kornacher
\end{modulHead}
\begin{modulHead}
\textbf{Studiensemester}:
4
\end{modulHead}
\begin{modulHead}
\textbf{Sprache}:
deutsch
\end{modulHead}
\begin{modulHead}
\textbf{Kreditpunkte}:
5
\end{modulHead}
\begin{modulHead}
\textbf{Typ}:
Teilmodul
\end{modulHead}
\begin{modulHead}
\textbf{Prüfungsleistung}:
Projekt und schriftliche Ausarbeitung
\end{modulHead}


\hypertarget{lehrformswspathlabelmi-2017modulbeschreibungen-bachelorba_vc-visuelle-effekte-und-animation}{%
\section*{Lehrform/SWS\label{/mi-2017/modulbeschreibungen-bachelor/BA_VC-visuelle-effekte-und-animation}}\label{lehrformswspathlabelmi-2017modulbeschreibungen-bachelorba_vc-visuelle-effekte-und-animation}}

36h Vorlesung; 36h Praktikum / Projekt; 78h Selbstlernphase

\hypertarget{angestrebte-lernergebnissepathlabelmi-2017modulbeschreibungen-bachelorba_vc-visuelle-effekte-und-animation}{%
\section*{Angestrebte
Lernergebnisse\label{/mi-2017/modulbeschreibungen-bachelor/BA_VC-visuelle-effekte-und-animation}}\label{angestrebte-lernergebnissepathlabelmi-2017modulbeschreibungen-bachelorba_vc-visuelle-effekte-und-animation}}

Die Studierenden kennen die grundlegenden Produktionsschritte und
Abläufe einer Film- und TV-Produktion mit visuellen Effekten sowie die
in diesem Zusammenhang eingesetzten Softwaretools.

Sie haben die Fertigkeit, spezifische Fragestellungen der Umsetzung
visueller, computerbasierter Effekte und der damit zusammenhängenden
Bildbearbeitung zu bearbeiten und fallbezogene individuelle Lösungen zu
entwickeln.

Unter Entwicklungs- und Methodenkompetenz auf dem Gebiet der Visual
Effects ist die Fähigkeit zu verstehen, eigene und für den jeweiligen
Anwendungsfall auch eventuell neue Lösungsansätze zu entwickeln, bei
denen die unterschiedlichen Methoden der Visual Effects-Ausführung und
-Bearbeitung zum Einsatz kommen. Nachdem die Planung, Durchführung und
die Bearbeitung von Projekten auf dem Gebiet der Film- und TV-Produktion
mit visuellen Effekten in der Regel im kleinen Team erfolgt sind gerade
die Softskills der Teamkompetenz und der Organisationskompetenz von
großer Wichtigkeit in diesem Modul.

Berufsbilder, die von diesem Modul angesprochen werden, sind zum einen
in der Visual-Effects-spezifischen Softwareentwicklung, als auch im
Anwendungskontext zu finden: So zum Beispiel in der Planung,
Organisation, Durchführung und Verantwortung von VFX-Projekten.

\hypertarget{inhaltpathlabelmi-2017modulbeschreibungen-bachelorba_vc-visuelle-effekte-und-animation}{%
\section*{Inhalt\label{/mi-2017/modulbeschreibungen-bachelor/BA_VC-visuelle-effekte-und-animation}}\label{inhaltpathlabelmi-2017modulbeschreibungen-bachelorba_vc-visuelle-effekte-und-animation}}

\begin{itemize}
\tightlist
\item
  Storyboard
\item
  Kalkulation
\item
  Produktionabläufe
\item
  Keyverfahren mit Green- und Bluescreen
\item
  Compositing
\item
  Umgang mit Bild-/Videobearbeitungssoftware
\end{itemize}

\hypertarget{medienformenpathlabelmi-2017modulbeschreibungen-bachelorba_vc-visuelle-effekte-und-animation}{%
\section*{Medienformen\label{/mi-2017/modulbeschreibungen-bachelor/BA_VC-visuelle-effekte-und-animation}}\label{medienformenpathlabelmi-2017modulbeschreibungen-bachelorba_vc-visuelle-effekte-und-animation}}

\begin{itemize}
\tightlist
\item
  Vorlesungen mit Folienpräsentationen
\item
  Workshops zu Anwendungsprogrammen
\item
  Beispiele aus verschiedenen Medien: Filmbeispiele, Webvideos
\item
  Audiovisuelle Aufnahme- und Wiedergabegeräte
\item
  Lehrfilme und Video-Tutorials
\end{itemize}

\hypertarget{literaturpathlabelmi-2017modulbeschreibungen-bachelorba_vc-visuelle-effekte-und-animation}{%
\section*{Literatur\label{/mi-2017/modulbeschreibungen-bachelor/BA_VC-visuelle-effekte-und-animation}}\label{literaturpathlabelmi-2017modulbeschreibungen-bachelorba_vc-visuelle-effekte-und-animation}}

\begin{itemize}
\tightlist
\item
  Flückiger Barbara, Visual Effects: Filmbilder aus dem Computer
  (Zürcher Filmstudien), Schüren Verlag GmbH, 2008, ISBN 978-3894725181
\item
  Bertram Sascha, VFX (Praxis Film), UVK, 2005, ISBN 978-3896695154
\end{itemize}

\hypertarget{teilmodul-vonpathlabelmi-2017modulbeschreibungen-bachelorba_vc-visuelle-effekte-und-animation}{%
\section*{Teilmodul
von:\label{/mi-2017/modulbeschreibungen-bachelor/BA_VC-visuelle-effekte-und-animation}}\label{teilmodul-vonpathlabelmi-2017modulbeschreibungen-bachelorba_vc-visuelle-effekte-und-animation}}

\hyperref[/mi-2017/modulbeschreibungen-bachelor/BA_Vertiefung-Visual-Computing]{Vertiefung – Visual Computing}

\hypertarget{vertiefung-visual-computingpathlabelmi-2017modulbeschreibungen-bachelorba_vertiefung-visual-computing}{%
\chapter{Vertiefung --~Visual
Computing\label{/mi-2017/modulbeschreibungen-bachelor/BA_Vertiefung-Visual-Computing}}\label{vertiefung-visual-computingpathlabelmi-2017modulbeschreibungen-bachelorba_vertiefung-visual-computing}}

\begin{modulHead}
\textbf{Modulverantwortlich}: Prof.~Hans
Kornacher
\end{modulHead}
\begin{modulHead}
\textbf{Studiensemester}:
4
\end{modulHead}
\begin{modulHead}
\textbf{Sprache}:
deutsch
\end{modulHead}
\begin{modulHead}
\textbf{Kreditpunkte}:
20
\end{modulHead}
\begin{modulHead}
\textbf{Typ}:
Vertiefungsmodul
\end{modulHead}


\hypertarget{kurzbeschreibungpathlabelmi-2017modulbeschreibungen-bachelorba_vertiefung-visual-computing}{%
\section*{Kurzbeschreibung\label{/mi-2017/modulbeschreibungen-bachelor/BA_Vertiefung-Visual-Computing}}\label{kurzbeschreibungpathlabelmi-2017modulbeschreibungen-bachelorba_vertiefung-visual-computing}}

Das Modul „Visual Computing'' im Medieninformatik Bachelor beschäftigt
sich mit der Erzeugung und Verarbeitung visueller Informationen, sowohl
in realen als auch computergenerierten Szenarien.

Ziel dieses Moduls ist es den Studierenden eine fachlich fundierte,
praktische, sowie theoretische Grundlage im Umgang mit audiovisuellen
Medien zu geben. Dabei wird sowohl auf die technische Seite (technischen
Grundlagen der Video- und Fernsehtechnik) eingegangen als auch auf die
algorithmische (computergenerierte Bildsynthese, Gameentwicklung).

Das Modul ist aus vier Teilmodulen aufgebaut, von denen zwei
verpflichtend sind und zwei weitere aus einem Wahlkatalog gewählt werden
können.

Die beiden verpflichtenden Teilmodule, ``Audiovisuelle Medientechnik''
und ``Computergrafik und Animation'', schaffen ein Fundament, was es
erlaubt innerhalb der beiden verbliebenen Teilmodule, im Gesamtumfang
von 10 CP, tiefer in die jeweilige Spezialisierung einzutauchen. Dabei
gibt es grundsätzlich die Möglichkeit sich in Richtung Fernseh- und
Videoproduktion oder Gameentwicklung zu vertiefen.

Die Teilmodule werden nach Verfügbarkeit angeboten.

Die Teilmodule sind in der Regel projektbasiert aufgebaut, so dass
sowohl theoretischer Hintergrund als auch praxisnahes Wissen vermittelt
wird und zur Anwendung kommt.

\hypertarget{arbeitsaufwandpathlabelmi-2017modulbeschreibungen-bachelorba_vertiefung-visual-computing}{%
\section*{Arbeitsaufwand\label{/mi-2017/modulbeschreibungen-bachelor/BA_Vertiefung-Visual-Computing}}\label{arbeitsaufwandpathlabelmi-2017modulbeschreibungen-bachelorba_vertiefung-visual-computing}}

600h Gesamtaufwand

\hypertarget{angestrebte-lernergebnissepathlabelmi-2017modulbeschreibungen-bachelorba_vertiefung-visual-computing}{%
\section*{Angestrebte
Lernergebnisse\label{/mi-2017/modulbeschreibungen-bachelor/BA_Vertiefung-Visual-Computing}}\label{angestrebte-lernergebnissepathlabelmi-2017modulbeschreibungen-bachelorba_vertiefung-visual-computing}}

Je nach gewählten Teilmodulen entwickeln die Studierenden Fähigkeiten
zur selbstverantwortlichen Durchführung von Projekten im Bereich
Gamedevelopment, dreidimensionaler Darstellung virtueller Szenen, Film-
und Fernsehtechnik, sowie Visueller Effekte.

\hypertarget{enthuxe4lt-folgende-teilmodulepathlabelmi-2017modulbeschreibungen-bachelorba_vertiefung-visual-computing}{%
\section*{Enthält folgende
Teilmodule:\label{/mi-2017/modulbeschreibungen-bachelor/BA_Vertiefung-Visual-Computing}}\label{enthuxe4lt-folgende-teilmodulepathlabelmi-2017modulbeschreibungen-bachelorba_vertiefung-visual-computing}}

\begin{itemize}
\tightlist
\item
  \hyperref[/mi-2017/modulbeschreibungen-bachelor/BA_VC-audiovisuelle-medientechnik]{Audiovisuelle Medientechnik}
\item
  \hyperref[/mi-2017/modulbeschreibungen-bachelor/BA_VC-audiovisuelles-medienprojekt-2]{Audiovisuelles Medienprojekt 2}
\item
  \hyperref[/mi-2017/modulbeschreibungen-bachelor/BA_VC-bildverarbeitung-und-computer-vision]{Bildverarbeitung und Computer Vision}
\item
  \hyperref[/mi-2017/modulbeschreibungen-bachelor/BA_VC-computergrafik-und-animation]{Computergrafik und Animation}
\item
  \hyperref[/mi-2017/modulbeschreibungen-bachelor/BA_VC-visuelle-effekte-und-animation]{Visuelle Effekte und Animation}
\item
  \hyperref[/mi-2017/modulbeschreibungen-bachelor/BA_WPF-3D-MSD]{WPF 3D-Modellieren, -Scannen, -Drucken}
\end{itemize}

\hypertarget{vertiefung-web-developmentpathlabelmi-2017modulbeschreibungen-bachelorba_vertiefung-web_development}{%
\chapter{Vertiefung --~Web
Development\label{/mi-2017/modulbeschreibungen-bachelor/BA_Vertiefung-Web_Development}}\label{vertiefung-web-developmentpathlabelmi-2017modulbeschreibungen-bachelorba_vertiefung-web_development}}

\begin{modulHead}
\textbf{Modulverantwortlich}: Prof.~Christian
Noss
\end{modulHead}
\begin{modulHead}
\textbf{Studiensemester}:
4
\end{modulHead}
\begin{modulHead}
\textbf{Sprache}:
deutsch
\end{modulHead}
\begin{modulHead}
\textbf{Kreditpunkte}:
20
\end{modulHead}
\begin{modulHead}
\textbf{Typ}:
Vertiefungsmodul
\end{modulHead}


\hypertarget{kurzbeschreibungpathlabelmi-2017modulbeschreibungen-bachelorba_vertiefung-web_development}{%
\section*{Kurzbeschreibung\label{/mi-2017/modulbeschreibungen-bachelor/BA_Vertiefung-Web_Development}}\label{kurzbeschreibungpathlabelmi-2017modulbeschreibungen-bachelorba_vertiefung-web_development}}

Einführungen in Konzepte, Techniken und Arbeitsweisen der Web
Entwicklung.

\hypertarget{arbeitsaufwandpathlabelmi-2017modulbeschreibungen-bachelorba_vertiefung-web_development}{%
\section*{Arbeitsaufwand\label{/mi-2017/modulbeschreibungen-bachelor/BA_Vertiefung-Web_Development}}\label{arbeitsaufwandpathlabelmi-2017modulbeschreibungen-bachelorba_vertiefung-web_development}}

600h Gesamtaufwand

\hypertarget{angestrebte-lernergebnissepathlabelmi-2017modulbeschreibungen-bachelorba_vertiefung-web_development}{%
\section*{Angestrebte
Lernergebnisse\label{/mi-2017/modulbeschreibungen-bachelor/BA_Vertiefung-Web_Development}}\label{angestrebte-lernergebnissepathlabelmi-2017modulbeschreibungen-bachelorba_vertiefung-web_development}}

Die Studierenden

\begin{itemize}
\tightlist
\item
  kennen ausgewählte Methoden und Frameworks für die Web Entwicklung im
  Front-End, im Back-End und in vernetzten Geräten (IoT),
\item
  können eine Methoden und Technologiewahl für einen Projektkontext
  fachlich begründen,
\item
  können Frameworks und Methoden zur Realisierung von Proof-of-Concepts
  in einem Projektkontext einsetzen und
\item
  können die erzielten Ergebnisse fachlich, kritisch einordnen und
  diskutieren,
\item
  um kompetent in Web Entwicklungs Teams mitwirken zu können.
\end{itemize}

\hypertarget{enthuxe4lt-folgende-teilmodulepathlabelmi-2017modulbeschreibungen-bachelorba_vertiefung-web_development}{%
\section*{Enthält folgende
Teilmodule:\label{/mi-2017/modulbeschreibungen-bachelor/BA_Vertiefung-Web_Development}}\label{enthuxe4lt-folgende-teilmodulepathlabelmi-2017modulbeschreibungen-bachelorba_vertiefung-web_development}}

\begin{itemize}
\tightlist
\item
  \hyperref[/mi-2017/modulbeschreibungen-bachelor/BA_WD_Frameworks-daten-und-dienste]{Frameworks, Daten und Dienste im Web}
\item
  \hyperref[/mi-2017/modulbeschreibungen-bachelor/BA_WD_Frontend-Development]{Frontend Development}
\item
  \hyperref[/mi-2017/modulbeschreibungen-bachelor/BA_WD_Internet-of-things]{Internet of Things}
\item
  \hyperref[/mi-2017/modulbeschreibungen-bachelor/BA_WD_Praktische-IT-Sicherheit]{Praktische IT Sicherheit}
\end{itemize}

\hypertarget{vertiefung-social-computingpathlabelmi-2017modulbeschreibungen-bachelorba_vertiefung_socialcomputing}{%
\chapter{Vertiefung --~Social
Computing\label{/mi-2017/modulbeschreibungen-bachelor/BA_Vertiefung_SocialComputing}}\label{vertiefung-social-computingpathlabelmi-2017modulbeschreibungen-bachelorba_vertiefung_socialcomputing}}

\begin{modulHead}
\textbf{Modulverantwortlich}: Prof.~Dr.~Christian
Kohls, Prof.~Dr.~Mirjam
Blümm
\end{modulHead}
\begin{modulHead}
\textbf{Studiensemester}:
4
\end{modulHead}
\begin{modulHead}
\textbf{Sprache}:
deutsch
\end{modulHead}
\begin{modulHead}
\textbf{Kreditpunkte}:
20
\end{modulHead}
\begin{modulHead}
\textbf{Typ}:
Vertiefungsmodul
\end{modulHead}


\hypertarget{kurzbeschreibungpathlabelmi-2017modulbeschreibungen-bachelorba_vertiefung_socialcomputing}{%
\section*{Kurzbeschreibung\label{/mi-2017/modulbeschreibungen-bachelor/BA_Vertiefung_SocialComputing}}\label{kurzbeschreibungpathlabelmi-2017modulbeschreibungen-bachelorba_vertiefung_socialcomputing}}

In der Vertiefung „Social Computing'' werden die Wechselwirkungen
zwischen Gesellschaft und Informatik in den Mittelpunkt gestellt.
Rechnersysteme und Netzwerke werden von Menschen intentional gestaltet,
ausgerichtet an gesellschaftlichen Normen, Prozessen und Bedürfnissen.
Gleichzeitig beeinflussen IT-Systeme diese gesellschaftlichen Normen und
verändern Prozesse in allen Lebensbereichen. Die verantwortungsbewusste
Konzeption und Realisierung von soziotechnischen Systemen (z.B. Social
Software, Online Communities, e-Health, e-Government und e-Learning
Angebote) sowie die empirische Evaluation existierender Systeme sind
zentrale Ziele. Lösungen sollen unter ganzheitlichen Gesichtspunkten
entwickelt werden. Verschiedene Wertvorstellungen und Interessen
unterschiedlicher Stakeholder müssen identifiziert und berücksichtig
werden.

Das Modul verbindet daher Theorien, Modelle und Methodik der Human- und
Sozialwissenschaften mit anwendungsorientierter Informatik.

\hypertarget{arbeitsaufwandpathlabelmi-2017modulbeschreibungen-bachelorba_vertiefung_socialcomputing}{%
\section*{Arbeitsaufwand\label{/mi-2017/modulbeschreibungen-bachelor/BA_Vertiefung_SocialComputing}}\label{arbeitsaufwandpathlabelmi-2017modulbeschreibungen-bachelorba_vertiefung_socialcomputing}}

600h Gesamtaufwand

\hypertarget{angestrebte-lernergebnissepathlabelmi-2017modulbeschreibungen-bachelorba_vertiefung_socialcomputing}{%
\section*{Angestrebte
Lernergebnisse\label{/mi-2017/modulbeschreibungen-bachelor/BA_Vertiefung_SocialComputing}}\label{angestrebte-lernergebnissepathlabelmi-2017modulbeschreibungen-bachelorba_vertiefung_socialcomputing}}

Studierende sollen in der Lage sein, computergestützte Systeme nach
ethischen, politischen, sozialen und psychologischen Kriterien zu
bewerten, zu planen und umsetzen zu können.

Ziel ist es, soziale Innovation durch digitale Anwendungen entstehen zu
lassen. Neben den empirischen Methoden werden Designmethoden vermittelt,
sowohl auf der konzeptionellen als auch auf der softwaretechnischen
Implementierungsebene, um robuste, sichere und flexible Systeme zu
gestalten.

\hypertarget{enthuxe4lt-folgende-teilmodulepathlabelmi-2017modulbeschreibungen-bachelorba_vertiefung_socialcomputing}{%
\section*{Enthält folgende
Teilmodule:\label{/mi-2017/modulbeschreibungen-bachelor/BA_Vertiefung_SocialComputing}}\label{enthuxe4lt-folgende-teilmodulepathlabelmi-2017modulbeschreibungen-bachelorba_vertiefung_socialcomputing}}

\begin{itemize}
\tightlist
\item
  \hyperref[/mi-2017/modulbeschreibungen-bachelor/BA_SC_Projekt]{Social Computing Projekt}
\item
  \hyperref[/mi-2017/modulbeschreibungen-bachelor/BA_SC_empirische-forschungsmethoden]{Empirische Forschungsmethoden}
\item
  \hyperref[/mi-2017/modulbeschreibungen-bachelor/BA_SC_gamification]{Gamification}
\item
  \hyperref[/mi-2017/modulbeschreibungen-bachelor/BA_SC_soziotechnische-systeme]{Soziotechnische Systeme}
\end{itemize}

\hypertarget{frameworks-daten-und-dienste-im-webpathlabelmi-2017modulbeschreibungen-bachelorba_wd_frameworks-daten-und-dienste}{%
\chapter{Frameworks, Daten und Dienste im
Web\label{/mi-2017/modulbeschreibungen-bachelor/BA_WD_Frameworks-daten-und-dienste}}\label{frameworks-daten-und-dienste-im-webpathlabelmi-2017modulbeschreibungen-bachelorba_wd_frameworks-daten-und-dienste}}

\begin{modulHead}
\textbf{Modulverantwortlich}: Dirk
Breuer
\end{modulHead}
\begin{modulHead}
\textbf{Studiensemester}:
4
\end{modulHead}
\begin{modulHead}
\textbf{Sprache}:
deutsch
\end{modulHead}
\begin{modulHead}
\textbf{Kreditpunkte}:
5
\end{modulHead}
\begin{modulHead}
\textbf{Typ}:
Teilmodul
\end{modulHead}
\begin{modulHead}
\textbf{Prüfungsleistung}:
Mündliche Prüfung und Projektarbeit
\end{modulHead}


\hypertarget{aufwandpathlabelmi-2017modulbeschreibungen-bachelorba_wd_frameworks-daten-und-dienste}{%
\section*{Aufwand\label{/mi-2017/modulbeschreibungen-bachelor/BA_WD_Frameworks-daten-und-dienste}}\label{aufwandpathlabelmi-2017modulbeschreibungen-bachelorba_wd_frameworks-daten-und-dienste}}

50h Vorlesung, Seminar; 100h Selbstlernphase

\hypertarget{angestrebte-lernergebnissepathlabelmi-2017modulbeschreibungen-bachelorba_wd_frameworks-daten-und-dienste}{%
\section*{Angestrebte
Lernergebnisse\label{/mi-2017/modulbeschreibungen-bachelor/BA_WD_Frameworks-daten-und-dienste}}\label{angestrebte-lernergebnissepathlabelmi-2017modulbeschreibungen-bachelorba_wd_frameworks-daten-und-dienste}}

Die Studentinnen und Studenten kennen

\begin{itemize}
\tightlist
\item
  wesentliche Frameworks, Dienste und Werkzeuge für die serverseitige
  Entwicklung von Web Anwendungen
\item
  können ausgewählte Frameworks, Dienste und Tools in einem
  Projektkontext anwenden.
\end{itemize}

Die Kompetenz zur systematischen Entwicklung von Systemen in einem
arbeitsteiligen Team wird eingeübt und vertieft. Kenntnisse aus den
anderen Modulen der Vertiefung werden vertieft und verknüpft und im
Rahmen eines konkreten Projektauftrags angewendet.

Die Studierenden sind in der Lage ein Projektbriefing zu durchdringen
und daraus einen Projektauftrag abzuleiten und diesen im Team
abzuarbeiten.

Den Teilnehmern steht eine Auswahl an Techniken und Frameworks zur
Verfügung, aus dem sie die passenden Ansätze begründet auswählen und
anwenden können.

Die StudentenInnen sind in der Lage eine komplexe Anwendung im Web über
mehrere Endgeräte hinweg zu planen, zu realisieren und zu dokumentieren.

\hypertarget{inhaltpathlabelmi-2017modulbeschreibungen-bachelorba_wd_frameworks-daten-und-dienste}{%
\section*{Inhalt\label{/mi-2017/modulbeschreibungen-bachelor/BA_WD_Frameworks-daten-und-dienste}}\label{inhaltpathlabelmi-2017modulbeschreibungen-bachelorba_wd_frameworks-daten-und-dienste}}

\begin{itemize}
\tightlist
\item
  NodeJS
\item
  Services im Web: Amazon WS (AWS), Google Firebase
\item
  NoSQL Datenbanken
\item
  Web Analyse: Piwik,
\end{itemize}

Ausgewählte Tools sollen tiefgreifend erarbeitet werden und in einem
Projektkontext angewendet werden. Dies erfolgt in der Regel in dem
begleitenden Projekt

\hypertarget{medienformenpathlabelmi-2017modulbeschreibungen-bachelorba_wd_frameworks-daten-und-dienste}{%
\section*{Medienformen\label{/mi-2017/modulbeschreibungen-bachelor/BA_WD_Frameworks-daten-und-dienste}}\label{medienformenpathlabelmi-2017modulbeschreibungen-bachelorba_wd_frameworks-daten-und-dienste}}

Beamergestützte Vorträge, Rechnergestützte Workshops

\hypertarget{literaturpathlabelmi-2017modulbeschreibungen-bachelorba_wd_frameworks-daten-und-dienste}{%
\section*{Literatur\label{/mi-2017/modulbeschreibungen-bachelor/BA_WD_Frameworks-daten-und-dienste}}\label{literaturpathlabelmi-2017modulbeschreibungen-bachelorba_wd_frameworks-daten-und-dienste}}

\begin{itemize}
\tightlist
\item
  Tilkov et al.: REST und HTTP- Entwicklung und Integration nach dem
  Architekturstil des Web, dpunkt.verlag 2015
\item
  Watkin: Practical XMPP, Packt Publishing 2016
\item
  Saint-Andre: XMPP: THe Definitive Guide, OReilly 2009
\item
  Roy: RabbitMQ in Depth, Manning 2016
\item
  Newman: Building Microservices: Designing fine-grained systems,
  OReilly 2015
\end{itemize}

\hypertarget{teilmodul-vonpathlabelmi-2017modulbeschreibungen-bachelorba_wd_frameworks-daten-und-dienste}{%
\section*{Teilmodul
von:\label{/mi-2017/modulbeschreibungen-bachelor/BA_WD_Frameworks-daten-und-dienste}}\label{teilmodul-vonpathlabelmi-2017modulbeschreibungen-bachelorba_wd_frameworks-daten-und-dienste}}

\hyperref[/mi-2017/modulbeschreibungen-bachelor/BA_Vertiefung-Web_Development]{Vertiefung – Web Development}

\hypertarget{frontend-developmentpathlabelmi-2017modulbeschreibungen-bachelorba_wd_frontend-development}{%
\chapter{Frontend
Development\label{/mi-2017/modulbeschreibungen-bachelor/BA_WD_Frontend-Development}}\label{frontend-developmentpathlabelmi-2017modulbeschreibungen-bachelorba_wd_frontend-development}}

\begin{modulHead}
\textbf{Modulverantwortlich}: Prof.~Christian
Noss
\end{modulHead}
\begin{modulHead}
\textbf{Studiensemester}:
4
\end{modulHead}
\begin{modulHead}
\textbf{Sprache}:
deutsch
\end{modulHead}
\begin{modulHead}
\textbf{Kreditpunkte}:
5
\end{modulHead}
\begin{modulHead}
\textbf{Typ}:
Teilmodul
\end{modulHead}
\begin{modulHead}
\textbf{Prüfungsleistung}:
Schriftliche Prüfung
\end{modulHead}


\hypertarget{aufwandpathlabelmi-2017modulbeschreibungen-bachelorba_wd_frontend-development}{%
\section*{Aufwand\label{/mi-2017/modulbeschreibungen-bachelor/BA_WD_Frontend-Development}}\label{aufwandpathlabelmi-2017modulbeschreibungen-bachelorba_wd_frontend-development}}

60h Vorlesung/ Seminar; 90h Selbstlernphase

\hypertarget{angestrebte-lernergebnissepathlabelmi-2017modulbeschreibungen-bachelorba_wd_frontend-development}{%
\section*{Angestrebte
Lernergebnisse\label{/mi-2017/modulbeschreibungen-bachelor/BA_WD_Frontend-Development}}\label{angestrebte-lernergebnissepathlabelmi-2017modulbeschreibungen-bachelorba_wd_frontend-development}}

Die Studierenden kennen wesentliche Konzepte und Technologien des
Web-Frontend Developments und können diese anwenden, um eigenständig im
Team Web-Frontends zu konzipieren, realisieren und optimieren.

Die Studierenden sind in der Lage ein gegebenes Gestaltungskonzept zu
verstehen und zu erweitern, um dies als Web-Frontend umzusetzen.

Die Studierenden kennen Web-Frontend Frameworks und sind in der Lage
diese kritisch zu beurteilen und auf Basis der Anforderungen eines
konkreten Projekts das optimale Framework Set zu konfektionieren und die
Auswahl zu begründen.

Die Studierenden kennen das Zusammenspiel von server- und clientseitigen
Komponenten im Bereich des Webs und können Web-Frontends konzipieren und
realisieren, die mit serverseitigen Komponenten und Diensten möglichst
optimal zusammen arbeiten. Sie können außerdem, bezogen auf eine
konkrete Aufgabenstellung, abwägen, welche Funktionalitäten clientseitig
und welche serverseitig gelöst werden sollten.

\hypertarget{inhaltpathlabelmi-2017modulbeschreibungen-bachelorba_wd_frontend-development}{%
\section*{Inhalt\label{/mi-2017/modulbeschreibungen-bachelor/BA_WD_Frontend-Development}}\label{inhaltpathlabelmi-2017modulbeschreibungen-bachelorba_wd_frontend-development}}

\begin{itemize}
\tightlist
\item
  Web Basics: HTML, CSS, Javascript
\item
  CSS: Komplexe Layouts \& Responsivität
\item
  Javascript: Dynamische Anwendungen
\item
  Media Types
\item
  CSS Frameworks
\item
  CSS Preprozessoren
\item
  Javascript Frameworks
\item
  Performance
\item
  Microdata, Internationalisierung, SEO, Barrierefreiheit
\end{itemize}

\hypertarget{studien-pruxfcfungsleistungenpathlabelmi-2017modulbeschreibungen-bachelorba_wd_frontend-development}{%
\section*{Studien-/Prüfungsleistungen\label{/mi-2017/modulbeschreibungen-bachelor/BA_WD_Frontend-Development}}\label{studien-pruxfcfungsleistungenpathlabelmi-2017modulbeschreibungen-bachelorba_wd_frontend-development}}

Projektarbeit mit Projektpräsentationsprüfung und Fachgespräch.

\hypertarget{medienformenpathlabelmi-2017modulbeschreibungen-bachelorba_wd_frontend-development}{%
\section*{Medienformen\label{/mi-2017/modulbeschreibungen-bachelor/BA_WD_Frontend-Development}}\label{medienformenpathlabelmi-2017modulbeschreibungen-bachelorba_wd_frontend-development}}

Beamergestützte Vorträge, Rechnergestützte Workshops

\hypertarget{literaturpathlabelmi-2017modulbeschreibungen-bachelorba_wd_frontend-development}{%
\section*{Literatur\label{/mi-2017/modulbeschreibungen-bachelor/BA_WD_Frontend-Development}}\label{literaturpathlabelmi-2017modulbeschreibungen-bachelorba_wd_frontend-development}}

\begin{itemize}
\tightlist
\item
  Randy Connolly, Ricardo Hoar: Fundamentals of Web Development
\end{itemize}

\hypertarget{teilmodul-vonpathlabelmi-2017modulbeschreibungen-bachelorba_wd_frontend-development}{%
\section*{Teilmodul
von:\label{/mi-2017/modulbeschreibungen-bachelor/BA_WD_Frontend-Development}}\label{teilmodul-vonpathlabelmi-2017modulbeschreibungen-bachelorba_wd_frontend-development}}

\hyperref[/mi-2017/modulbeschreibungen-bachelor/BA_Vertiefung-Web_Development]{Vertiefung – Web Development}

\hypertarget{internet-of-thingspathlabelmi-2017modulbeschreibungen-bachelorba_wd_internet-of-things}{%
\chapter{Internet of
Things\label{/mi-2017/modulbeschreibungen-bachelor/BA_WD_Internet-of-things}}\label{internet-of-thingspathlabelmi-2017modulbeschreibungen-bachelorba_wd_internet-of-things}}

\begin{modulHead}
\textbf{Modulverantwortlich}: Prof.~Dr.~Matthias
Böhmer
\end{modulHead}
\begin{modulHead}
\textbf{Studiensemester}:
4
\end{modulHead}
\begin{modulHead}
\textbf{Sprache}:
deutsch
\end{modulHead}
\begin{modulHead}
\textbf{Kreditpunkte}:
5
\end{modulHead}
\begin{modulHead}
\textbf{Typ}:
Teilmodul
\end{modulHead}
\begin{modulHead}
\textbf{Prüfungsleistung}:
Seminarvortrag (30\%) und Projektarbeit (70\%)
\end{modulHead}


\hypertarget{aufwandpathlabelmi-2017modulbeschreibungen-bachelorba_wd_internet-of-things}{%
\section*{Aufwand\label{/mi-2017/modulbeschreibungen-bachelor/BA_WD_Internet-of-things}}\label{aufwandpathlabelmi-2017modulbeschreibungen-bachelorba_wd_internet-of-things}}

50h Vorlesung, Seminar; 100h Selbstlernphase

\hypertarget{angestrebte-lernergebnissepathlabelmi-2017modulbeschreibungen-bachelorba_wd_internet-of-things}{%
\section*{Angestrebte
Lernergebnisse\label{/mi-2017/modulbeschreibungen-bachelor/BA_WD_Internet-of-things}}\label{angestrebte-lernergebnissepathlabelmi-2017modulbeschreibungen-bachelorba_wd_internet-of-things}}

In diesem Modul lernen die Teilnehmer das Gebiet Internet of Things
kennen. Dabei liegt ein besonderer Fokus auf der Bedeutung des Web für
Applikationen jenseits eines Browsers. Immer mehr Alltagsgegenstände
werden mit Technologien angereichert, die eine Dienste-Bereitstellung
oder Dienst-Nutzung über das Web ermöglichen (beispielsweise das Steuern
von Gegenständen oder das Erfassen von Sensordaten). In diesem Modul
werden relevante Konzepte und aktuelle Technologien für das Internet der
Dinge diskutiert und in prototypischen Anwendungen erprobt.

Studierende können nach diesem Modul selbstständig verteilte Anwendungen
für das Internet of Things konzipieren und realisieren, die ihre
physikalische Umgebung wahrnehmen und verändern und mit Web-Komponenten
kommunizieren, indem sie

\begin{itemize}
\tightlist
\item
  Sensoren und Aktoren zur Messung und Veränderung der Umwelt auswählen,
\item
  hardwarenahe Software für Mikrocontroller und Einplatinencomputer
  entwickeln,
\item
  einschlägige Architekturen diskutieren und eine System-Architektur
  entwerfen,
\item
  geeignete Protokolle zur Vernetzung im Internet of Things kennen und
  nutzen,
\item
  relevante Technologien evaluieren und für eigene Implementierungen
  bewerten,
\item
  sowie Prototyping als Entwicklungsansatz im IoT einsetzen,
\end{itemize}

um später Anwendungen und Produkte zur realisieren, bei denen digitale
und dingliche Welten im Web miteinander wechselwirken.

\hypertarget{internet-of-thingspathlabelmi-2017modulbeschreibungen-bachelorba_wd_internet-of-things-1}{%
\subsection*{Internet of
Things\label{/mi-2017/modulbeschreibungen-bachelor/BA_WD_Internet-of-things}}\label{internet-of-thingspathlabelmi-2017modulbeschreibungen-bachelorba_wd_internet-of-things-1}}

\begin{itemize}
\tightlist
\item
  Physical Computing
\item
  Prototyping und Retrofitting
\item
  Hardware (bspw. RaspberryPi und Arduino)
\item
  Sensoren und Aktoren
\item
  Frameworks (bspw. NodeRed und Johnny Five)
\item
  Architekturen und Protokolle (bspw. event-basierte Architekturen und
  MQTT)
\item
  Mobile Web- und Smartphone-Sensoren (bspw. GPS, Beacons)
\end{itemize}

\hypertarget{medienformenpathlabelmi-2017modulbeschreibungen-bachelorba_wd_internet-of-things}{%
\section*{Medienformen\label{/mi-2017/modulbeschreibungen-bachelor/BA_WD_Internet-of-things}}\label{medienformenpathlabelmi-2017modulbeschreibungen-bachelorba_wd_internet-of-things}}

Beamergestützte Vorträge, Rechnergestützte Workshops

\hypertarget{teilmodul-vonpathlabelmi-2017modulbeschreibungen-bachelorba_wd_internet-of-things}{%
\section*{Teilmodul
von:\label{/mi-2017/modulbeschreibungen-bachelor/BA_WD_Internet-of-things}}\label{teilmodul-vonpathlabelmi-2017modulbeschreibungen-bachelorba_wd_internet-of-things}}

\hyperref[/mi-2017/modulbeschreibungen-bachelor/BA_Vertiefung-Web_Development]{Vertiefung – Web Development}

\hypertarget{praktische-it-sicherheitpathlabelmi-2017modulbeschreibungen-bachelorba_wd_praktische-it-sicherheit}{%
\chapter{Praktische IT
Sicherheit\label{/mi-2017/modulbeschreibungen-bachelor/BA_WD_Praktische-IT-Sicherheit}}\label{praktische-it-sicherheitpathlabelmi-2017modulbeschreibungen-bachelorba_wd_praktische-it-sicherheit}}

\begin{modulHead}
\textbf{Modulverantwortlich}: Prof.~Dr.~Stefan
Karsch
\end{modulHead}
\begin{modulHead}
\textbf{Studiensemester}:
4
\end{modulHead}
\begin{modulHead}
\textbf{Sprache}:
deutsch
\end{modulHead}
\begin{modulHead}
\textbf{Kreditpunkte}:
5
\end{modulHead}
\begin{modulHead}
\textbf{Typ}:
Teilmodul
\end{modulHead}
\begin{modulHead}
\textbf{Prüfungsleistung}:
Seminarvortrag
\end{modulHead}


\hypertarget{aufwandpathlabelmi-2017modulbeschreibungen-bachelorba_wd_praktische-it-sicherheit}{%
\section*{Aufwand\label{/mi-2017/modulbeschreibungen-bachelor/BA_WD_Praktische-IT-Sicherheit}}\label{aufwandpathlabelmi-2017modulbeschreibungen-bachelorba_wd_praktische-it-sicherheit}}

50h Vorlesung, Seminar; 100h Selbstlernphase

\hypertarget{teilmodul-vonpathlabelmi-2017modulbeschreibungen-bachelorba_wd_praktische-it-sicherheit}{%
\section*{Teilmodul
von:\label{/mi-2017/modulbeschreibungen-bachelor/BA_WD_Praktische-IT-Sicherheit}}\label{teilmodul-vonpathlabelmi-2017modulbeschreibungen-bachelorba_wd_praktische-it-sicherheit}}

\hyperref[/mi-2017/modulbeschreibungen-bachelor/BA_Vertiefung-Web_Development]{Vertiefung – Web Development}

\hypertarget{wpf-3d-modellieren--scannen--druckenpathlabelmi-2017modulbeschreibungen-bachelorba_wpf-3d-msd}{%
\chapter{WPF 3D-Modellieren, -Scannen,
-Drucken\label{/mi-2017/modulbeschreibungen-bachelor/BA_WPF-3D-MSD}}\label{wpf-3d-modellieren--scannen--druckenpathlabelmi-2017modulbeschreibungen-bachelorba_wpf-3d-msd}}

\begin{modulHead}
\textbf{Modulverantwortlich}: Prof.~Dr.~Horst
Stenzel
\end{modulHead}
\begin{modulHead}
\textbf{Studiensemester}:
4+
\end{modulHead}
\begin{modulHead}
\textbf{Sprache}:
deutsch
\end{modulHead}
\begin{modulHead}
\textbf{Kreditpunkte}:
5
\end{modulHead}
\begin{modulHead}
\textbf{Typ}:
Teilmodul
\end{modulHead}
\begin{modulHead}
\textbf{Prüfungsleistung}:
Projekt u. Lernportfolio
\end{modulHead}


\hypertarget{lehrformswspathlabelmi-2017modulbeschreibungen-bachelorba_wpf-3d-msd}{%
\section*{Lehrform/SWS\label{/mi-2017/modulbeschreibungen-bachelor/BA_WPF-3D-MSD}}\label{lehrformswspathlabelmi-2017modulbeschreibungen-bachelorba_wpf-3d-msd}}

70h Vorlesung und Selbstudium; 80h Projekt

\hypertarget{angestrebte-lernergebnissepathlabelmi-2017modulbeschreibungen-bachelorba_wpf-3d-msd}{%
\section*{Angestrebte
Lernergebnisse\label{/mi-2017/modulbeschreibungen-bachelor/BA_WPF-3D-MSD}}\label{angestrebte-lernergebnissepathlabelmi-2017modulbeschreibungen-bachelorba_wpf-3d-msd}}

Die Studierenden sollen

\begin{itemize}
\tightlist
\item
  die Grundlagen und unterschiedliche Verfahren zur Erstellung von
  3D-Modellen sowie die Grundlagen einer 3D-Pipeline kennenlernen.
\item
  anwendungsorientiert modellieren. Dazu gehört es Anforderungen zu
  ermitteln und die Einschränkungen des angestrebten Verwendungszwecks
  zu verstehen und korrekt zu nutzen. Beispiele sind das
  3D-Druck-gerechte Modellieren und die Animationserstellung.
\item
  Unterschiedliche Druck- wie auch Scanverfahren verstehen und anwenden
  können.
\end{itemize}

\hypertarget{inhaltpathlabelmi-2017modulbeschreibungen-bachelorba_wpf-3d-msd}{%
\section*{Inhalt\label{/mi-2017/modulbeschreibungen-bachelor/BA_WPF-3D-MSD}}\label{inhaltpathlabelmi-2017modulbeschreibungen-bachelorba_wpf-3d-msd}}

\begin{itemize}
\tightlist
\item
  3D-Modellierungstechniken
\item
  Einführung in Rendering-Methoden
\item
  3D-Modell-Arten
\item
  Dateiformate für 3D-Daten
\item
  3D-Scanverfahren
\item
  Photogrammetrie
\item
  Softwarelösungen
\item
  Kontextorientiertes Modellieren und Anwendungsbeschränkungen
\item
  3D-Druckarten und Architekturen
\end{itemize}

\hypertarget{medienformenpathlabelmi-2017modulbeschreibungen-bachelorba_wpf-3d-msd}{%
\section*{Medienformen\label{/mi-2017/modulbeschreibungen-bachelor/BA_WPF-3D-MSD}}\label{medienformenpathlabelmi-2017modulbeschreibungen-bachelorba_wpf-3d-msd}}

\begin{itemize}
\tightlist
\item
  Screencasts
\item
  Beispielmedien
\end{itemize}

\hypertarget{teilmodul-vonpathlabelmi-2017modulbeschreibungen-bachelorba_wpf-3d-msd}{%
\section*{Teilmodul
von:\label{/mi-2017/modulbeschreibungen-bachelor/BA_WPF-3D-MSD}}\label{teilmodul-vonpathlabelmi-2017modulbeschreibungen-bachelorba_wpf-3d-msd}}

\hyperref[/mi-2017/modulbeschreibungen-bachelor/BA_Vertiefung-Visual-Computing]{Vertiefung – Visual Computing}

\hypertarget{wahlpflichtmodulpathlabelmi-2017modulbeschreibungen-bachelorba_wpf}{%
\chapter{Wahlpflichtmodul\label{/mi-2017/modulbeschreibungen-bachelor/BA_WPF}}\label{wahlpflichtmodulpathlabelmi-2017modulbeschreibungen-bachelorba_wpf}}

\begin{modulHead}
\textbf{Modulverantwortlich}: alle Professor:innen
der Lehreinheit Informatik der
F10
\end{modulHead}
\begin{modulHead}
\textbf{Studiensemester}:
5
\end{modulHead}
\begin{modulHead}
\textbf{Sprache}:
deutsch
\end{modulHead}
\begin{modulHead}
\textbf{Kreditpunkte}:
5
\end{modulHead}
\begin{modulHead}
\textbf{Typ}:
Pflichtmodul
\end{modulHead}
\begin{modulHead}
\textbf{Prüfungsleistung}:
abhängig vom jeweiligen WPF
\end{modulHead}


\hypertarget{arbeitsaufwandpathlabelmi-2017modulbeschreibungen-bachelorba_wpf}{%
\section*{Arbeitsaufwand\label{/mi-2017/modulbeschreibungen-bachelor/BA_WPF}}\label{arbeitsaufwandpathlabelmi-2017modulbeschreibungen-bachelorba_wpf}}

150 Stunden

\hypertarget{angestrebte-lernergebnissepathlabelmi-2017modulbeschreibungen-bachelorba_wpf}{%
\section*{Angestrebte
Lernergebnisse\label{/mi-2017/modulbeschreibungen-bachelor/BA_WPF}}\label{angestrebte-lernergebnissepathlabelmi-2017modulbeschreibungen-bachelorba_wpf}}

Fachliche Vertiefung oder Verbreiterung, nach persönlichem Interesse. Es
kann eines der Module aus dem Katalog aller Module der Informatik
Bachelorstudiengänge gewählt werden. Auch Pflichtmodule anderer
Informatik Studiengänge am Campus können als Wahlpflichtmodule in der
Medieninformatik belegt werden.
